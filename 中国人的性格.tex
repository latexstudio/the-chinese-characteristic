% !Mode:: "TeX:UTF-8"%確保文檔utf-8編碼

\documentclass[12pt,oneside]{book}
\newlength{\textpt}
\setlength{\textpt}{12pt}

\usepackage{myconfig}
\usepackage{mytitle}



\begin{document}
\frontmatter

\titlea{中国人的性格}
\author{亚瑟·亨·史密斯}
\authorinfo{作者:美国公理会来华传教士。译者:乐爱国,张华玉。}
\editor{德山书生}
\email{a358003542@gmail.com}
\editorinfo{编者:德山书生,湖南常德人氏。}
\version{0.1}
\titleLA

\addchtoc{前言}
\chapter*{前言}
\begin{common-format}
\section*{简介}
来到这里的都是会思考的智慧人士,所以多的话我就不说了。下面简单介绍一下作者的一些信息,内容主要来自维基百科词条:明恩溥,更多信息请查看维基百科词条:Arthur Henderson Smith。

Arthur Henderson Smith(1845年7月18日-1932年8月31日)(中文名:明恩溥(pǔ)) 美国公理会来华传教士。

1872年受美国公理会派遣来华,先后居住于天津、山东等地,兼任上海《字林西报》通讯员。1880年,明恩溥在山东省西北部的恩县庞庄建立传教工作。1905年辞去教职,留居通州写作。1926年返回美国。他在华生活54年,熟悉下层人民生活,热爱中国,是最早向美国总统老罗斯福建议退还中国庚子赔款的人。著有多部关于中国的书籍,有《中国人的性格》(Chinese Characteristics)(1899年出版)、《中国乡村生活》、《今日的中国和美国》等。《中国人的性格》一书曾被鲁迅向国人郑重推荐。

顺便提一下豆瓣词条将清华大学和协和医院等都和作者扯上关系还有待考证。


\section*{推进}
 在谈论政治的时候我们需要分清两个问题,一个是人的本性,一个是政府的本性。政府的本性是由政治制度决定的,而所治下的人的本性更加的复杂。人的本性是可以改变的,但只限于小部分的改变;政府的本性也是可以改变的,而且相对于国民本性的改变并不见的是一件相对困难的事情。人们通常谈论政治更多的侧重在政治制度的改变的,其实就好像两个矛盾的事物,如果只关注于阴而不关注于阳,阴就算暂时增长了过会儿也会被阳遏制住,人的本性和政府的本性就是这两个矛盾的方面。

严格意义上讲,人的本性是无所谓好和坏的,人,在这社会中,都有着不同的本性,有的奸诈如鼠,有的老实正直等等等等。这个话题是如此的古老啊,远在柏拉图的时代,在理想国一书中 ,苏格拉底和他的好基友们就在严肃地谈论着这个问题,为什么这个社会总是坏人得利,好人总没好报。

人类历史的政治运动一直都在继续,好人和坏人,或者说不同的人总是存在着的,很难说从古到今是不是好人的基因变多了还是变少了。 乐观的人说也许没变或者好人,善良的人的基因是变多了,悲观的人甚至认为现在坏人的基因应该变得更多了,因为自古以来法制道德等等各种监督措施都不完善,当然是坏人得利,温顺的绵羊被奴隶或者被淘汰掉。

即使主张人做自利的事情会增进社会福利的亚当斯密也不敢保证,这样的情况是实现所有人最大福利和社会最大福利的情况。公地危机仍然存在,可以这么说,那些做了坏事不受惩罚的,就好比在一块公地上乱丢垃圾。

法律只管大事,法律是在道德的基础之上的,道德是那么一个古怪的东西。就是如果大家的所作所为曝光如阳光之下,那么什么是好什么是坏基本上还是能够如同公理一般大家给出一个答案。这里首先有个问题,并不是所有的真相总会水落石出,所以这个世界上做伪证的撒谎的演习的比比皆是,从无关紧要的戏谑到罪大恶极的都有。

我谈到这里就是想说这一点,中国人的本性和中国人的政府是两个非常紧密联系着的事物。从历史来看,中国人的本性可能这么多年来变动的很少,而正是这种中国人的本性的稳定性造成了中国人的政府不管换成那种形式,其本质都变化很少的原因。

并不存在完美的国民本性和完美的政治制度,而中国人的本性和中国人的政府是相容相洽同时又矛盾重重的。比如说中国人本性喜好撒谎不诚实,彼此之间缺乏信任等等,难道中国政府不也是如此吗?他们也是谎话连篇,好面子好演戏。

正如之前说的,很难说哪一种事物的本性是不好的,但我们可以说某一种事物的本性是待圆满待改善待改进的,而这改进的目的就是为了使得这个事物更加长久,具体到这里就是中国人的本性和中国政府的本性(政治制度)的改进是彼此互动,相互协同作用的,而目的就有一个:\textbf{国富民强}。

当我们承认中国人和中国政府还存在很大的改进空间,中国还不是完全的国富民强,我们还有很多问题需要解决,那么我们的谈话可以继续。

如果你们如同五毛一般,掩耳盗铃,混淆黑白,说中国已经是世界第一强国的,中国一切都是好的,中国人的本性是最好的,不需要改进了。(恩,听起来满满的形而上学和绝对论的味道),那么我们没必要继续谈论下去了,建议你们找个精神病院住下来,别碍了大家的思考就行。

如果我们承认中国人的本性和中国人的政府才存在很多问题,还有很大的改进空间,那么我们的谈论继续下去。

当然可能有的五毛会这么说,中国已经足够的好了,虽然中国目前不是最好的,但随着时间推移他会自然变得最好,不需要我们进行思考和做些什么了。这论调和之前那五毛的论调其实都是差不多的,我把其称之为轻度精神病,好在我们还有康复医院可以让这一类人住下来。 让他们待上一段时间,然后中国就彻底完美彻底好了,然后再把他们放出来,转移到精神病院,恩,我觉得这样对大家都挺好的。

那么剩下的人,和我们的谈论继续下去,是清楚事情不会自然变好,天上不会掉馅饼,还需要我们主动认知主动思考主动做些改变,那么事情才会变得更好,是吗?

那么剩下的人我们的旅途继续吧。

%这里空一行。
\end{common-format}


\addchtoc{目录}
\setcounter{tocdepth}{2}
\tableofcontents

\begin{common-format}
\mainmatter

\chapter{导言}
人们总是希望见证人所说的都是真话,没有半点含糊。许多略知中国人的见证人,虽然能说出一些真实的东西,但他们中并没有多少人能如实地叙述;更不用说要他们讲出全部真情。任何个人,无论他的知识面多宽,都不可能真正了解中国人的全部真实情况。因而,本书必须面对来自3种不同观点的异议。 

首先,或许有人会说,要把所知道的有关中国人的特性如实地转述给他人,那只是白费力气。乔治·温格罗夫·库克先生是1857至1858年期间伦敦《泰晤士报》的一位专门采访中国的记者;他像所有到中国去的作家一样,有机会亲眼目睹在各种环境下的中国人,并可以借助于那些很具资历的人物的看法去了解中国人。然而,库克先生在他所出版的书信集的前言中却承认他对中国人特性的描述是失败的,并为此表示歉意。他说:“在这些书信里,有关中国人特性的作品,我写得并不够精彩,这是很大的疏忽。这本是一个最具诱惑力且最能施展才华的题目;机智的假定,深刻的概括,自信的断言,都可以在其中得到展示。那些吹毛求疵的批评家们肯定会由于我没有从这种机会中有所收获而鄙视我。事实上,我写过几位中华民族中非常优秀的人物;但不幸的是,在我把眼前的这些人物置于笔下时,他们曾有过的粗俗的言行却与我的初衷相违背。为了追求真实,我烧了好几封长信。此外,我还经常就这种事与最著名的汉学家坦诚地进行交谈,结果发现,他们与我一样,都认为要形成有关中国人特性的整体概念是不可能的。当然,这些困难只有那些真正了解中国人的汉学家们才会遇到。一位精明的作家,可以在完全撇开主题的情况下,轻易地作出两个客观真实、头头是道但相互对立的分析。也许有朝一日,我们获得了那些必要知识,能够对中国人的自相矛盾之处两方面都做出恰如其分的分析。而现在,我能做的就是避免严格的精确的描述,用他们最突出的特性来描述之。” 

在过去的30年中,中国人已经使自己成为国际事务中的一个重要的角色。他们被看做是不可征服的和给人一种如此神秘的感觉。的确,除了在中国,任何其他国家的人都不可能真正了解中国人;在不少人的印象中,中国人是根本无法理解的矛盾体。不过也没有明显的理由,毕竟我们已经同中国打了几百年的交道,为什么真正地了解中国人就和其他复杂的现象一样难呢。 

“其次,对本书更为严重的反对意见是笔者并不完全具备写此书的资格。一个在中国生活了22年的人,并不能完全保证他有能力写出有关中国人特性的书,正如一个在银矿里埋头苦干了22年的人,并不足以证明他有资格撰写出有关冶金学或金本位制的论文。中国是一个巨大的整体;一个还未考察过它的一半以上省份且只是在其中两个省居住过的人,当然没有资格对这整个国家作出概括。本书的这些文章最初只是为上海的《华北每日新闻》所准备的,并没有考虑更广泛的传播。然而,其中的一些论题不仅在中国,而且在英国、美国和加拿大都引起了极大的兴趣,笔者这才应读者的要求将文章汇编成册。 

第三种反对意见,来自某些人,认为要阐发某些看法,特别是涉及中国人的道德特征,会产生误解和不公正。 

然而,人们应当记住,印象并不可能像统计数字那样丝毫不差。它们更像是照片的底片,没有哪两张是相同的,但每一张都反映了某部分真实,而为其他底片所没有观察到。拍照用的胶片不同,透镜不同,显影剂又不同,其结果当然也就不同。 

许多久居中国的人,他们对这个国家的了解完全要比笔者多得多。他们所表达的看法实质上是与笔者相一致的。还有人认为,在某些部位加一些亮丽的色彩会给过于单调的画面增添真实性,这些人的看法同样值得尊重。正是考虑到这些十分正确的意见,笔者对原文作了全面的修订。由于出版的紧迫,原来所讨论的中国人的特性有三分之一被省略了,当然,最重要的部分仍然保留,并新写了“知足常乐”一章。 

对于中国人所具有的并且表现出来的一切好的品性,我们没有任何正当的理由不予以赞美;同时,又不能陷于某种先验的思维框架,抬高他们的实际的道德品行而超过公正的应得的那部分——这种做法的危害性并不亚于那种不分青红皂白的指责。这使我们联想起撒克里(Thackeray)。曾经有人问他,在他的小说中,怎么会好人总是傻瓜,坏人却是聪明。对此,这位伟大的讽刺作家回答说,他是无心的。一幅表现橡树的木刻画,要求观察者从橡树的轮廓中分辨出拿破仑抱臂低头站在圣·海伦娜岛的样子。长时间地这样看,往往一无所获,而且,这样做显然也是不对的;但是,一经他人指点,看画者在看画时就不可避免地要看到拿破仑的样子。同样,在中国,许多事起初并没有出现,但却能被看出,而一旦被看出,就难以忘却。 

正如限制性从句不能取代概括性的主句,本书的文章并不是对整个中国的概括,也不是诸多外国人所见所闻的集萃,这一些必须提请读者们注意。这些文章仅仅是一位观察者对自己的印象所作的描述,只是许多“中国人特性”中的一部分。它们并不构成一幅中国民众的肖像图,而更像是观察者根据自己的所见所闻用炭笔对中国民众的某些特性所作的简略素描。它们只是组成一条光线,而无数的光线交织在一些,才能形成一道白光。它们也可能被视作归纳研究,来自笔者及其他各种人的个别经验在这里得到汇集。正是为了做到这一点,围绕着各个主题列举了大量的具体事例。 

米多斯先生(Meadow)是众多研究中国及中国人的作家中最富哲理性的。他认为,一个人对外国民族特性有了正确的看法并希望把这种看法转达给他人时,其最佳方式莫过于把所有有关的笔记都交给他细读。在这些笔记中,详细记录了大量引人注目的事件,特别是那些非同一般的事件,并附有当地人对该事件的说明。 

从大量的此类事件中推出一般性结论。推出的结论可以被怀疑或否定,但所列举的个别事例,只要是真实的,就不能仅仅由于某种原因而被搁在一边。任何有关中国人特性的理论最终都必须对这些事例作出分析。 

将中国人与盎格鲁-撒克逊人进行比较是十分困难的。试图从事这一研究的人会强烈地感受到这一点。同样,许多看起来属于中国人“特性”的东西,实际上只是东方人的特性,待会儿就会看到这种情况;至于对不对,每位读者都可以根据自己的切身经验予以判断。 

据说,当今与中国人交往、了解他们的社会生活,有三条途径:研究他们的小说、民谣和戏剧。这些信息来源无疑有其价值,但似乎还有第四条途径,那就是研究中国人在他们家乡的家庭生活,这一途径比前三者相加起来还有价值,但它并不向所有研究中国和中国人的作家开放。正如在农村比在城市更能明白一个区域的地形。在农村也更易于了解人的特性。一个外国人在中国的城市住上10年,他所获得的有关中国人家庭生活的知识,或许并不如在中国农村住12个月所获得的知识多。除了研究中国人的家庭生活,我们还必须把乡村看做是中国社会生活的基本单元,本书的这些文章正是以中国农村为立足点而写成的。这些文章并不是为了表达一个传教士的观点,而是一个没有任何主观偏见的观察者对所见所闻的如实报告。正是由于这一点,本书没有涉及所谓用基督教改善中国人特性的问题,也不作出中国人需要基督教的假定。但是,假如他们的特性中存在着严重的缺陷,那么,这些缺陷如何纠正就是一个很值得研究的问题。 

如前所述,“中国问题”己远不是一个国家的问题,而是国际性的问题。\uwave{完全有理由相信,到20世纪,这一问题将是更为紧迫的问题。中国人是人类中相当大的一部分,如何改善他们是每一个希望人类美好的人不可能不感兴趣的问题}。如果我们导出的结论是正确的话,那么这些结论将会得到一系列曾被过多忽视的论据的支持;如果这些结论错了,无论怎样支持,都不攻自破。 

埃尔金勋爵(Elgin)对上海商界的那次答问,虽然已过多年,但他的话至今仍是正确和中肯的。他说:“当阻碍自由进入这个国家的障碍被搬开时,西方基督教文化将发现自己所面对的不是野蛮,而是一个古老的文明在很多方面都表现出衰退和有缺陷,但在另一些方面又不能不使我们抱以同情和尊敬。在将要出现的竞争中,基督教文化要想在这个具有怀疑态度且又足智多谋的民族中开辟道路,就要使人们明白,进入天堂的信仰比不离尘世的信仰更能保证公众和个人具有良好的道德品行。”


\chapter{保全面子}
乍一看来,把全人类所共有的“脸面”当做中国人的特性,可能太不合理了。但是,中国人所讲的“脸面”不仅仅指头的前面部分,它是具有多种复杂含义的名词,其意思比我们所能描述的或者所能理解的还要多。 

为了理解“脸面”的意思,哪怕是作不完整的理解,我们也必须指出,中国人具有非常强的爱演戏的本能。戏剧几乎可以说是唯一通行全国的娱乐活动。中国人对戏剧演出的热情,犹如英国人爱好体育、西班牙人爱好斗牛。只要略加鼓动,任何中国人都会有模有样地扮演起某出戏中的某个角色。他会摆出演戏的姿势,鞠躬行礼,下跪叩拜。对于西方人来说,这种情形似乎多余,或是可笑。中国人是用戏剧化的语言进行思维。当要为自己辩解时,他面对两三人的讲话就像是面对众人。他大声地说道:“我是当着你们的面说的,你,还有你,你们都在场。”如果他心情愉快,他会说自己是光荣地“离开舞台”;如果心情不好,他会说自己没有脸“退出舞台”。所有这些,如果明白了,就会发现与实际毫不相干。这根本不是事实的问题,而是形式的问题。一场戏中,在适当的时候,以适当的方式,说了一句漂亮话,这场戏就会赢得喝彩。我们并非要进入幕后,因为那将会糟踏世界上的所有好戏。在一切复杂的生活关系中,完全依据戏剧化的样式而行动,那就会有“面子”。在他们表演时,不理他们,小看他们,喝倒彩,他们就“丢面子”。一旦正确理解了“面子”所包含的意思,人们就会发现,“面子”这个词本身是打开中国人许多最重要特性之锁的钥匙。

值得一提的是,如何做到有“面子”,其技巧和造诣往往是西方人所望尘莫及的。西方人经常是忘记了其中戏剧的因素,而走进与事实无关的领域中。在西方人看来,中国人的“面子”就像是南海岛的戒律,一种潜在的,不可否认的力量;只是“面子”可以反复无常,不可简化为规则,并且只是按照公共的情理而取消或变更。在这一点上,中国人与西方人必须承认存在着差异,因为他们决不可能用同样的眼光看待同样的事情。在调解各个村庄之间常有的无休止的争吵时,“和事佬”必须认真考虑“面子”的平衡,就像过去欧洲政治家考虑权力平衡一样。在这种情况下,目的不在于执行公正的原则,而是按照适当的比例,对所有有关的“面子”进行分配。执行公正的原则,对一个东方人来说,即使从道理上讲有这种愿望,但实际上是不可能的。就是在诉讼的裁决中、按比例分配“面子”的事也常有发生,这使得相当大比例的裁决在所谓不分胜负的比赛中不了了之。 

送人一份丰厚的礼物,算是“给面子”。但是,如果礼物是个人所送,受礼人应当只接受其中的一部分,统统拒绝是很少见的,或者根本不可能。一些渴望保住面子的事例很能说明问题。因错而受到指责是“失面子”,所以,无论证据多么充分,也要为了保住面子,予以否认。一个网球不见了,被一位苦力拣了去;即使有证据,他也会不客气地予以否认,然后走到丢失球的地方,找到了球(球是从他的长袖里掉出来的),并且还说:“这是你“丢”的球。”一位侍女把客人的削铅笔刀藏在她主人的房里,后来,她又在台布下把刀子找了出来,并且还谎称是她找到的。这样,“面子”保住了,一位仆人不小心丢失了主人的一件东西,他知道必须赔偿,或者被扣掉相当数目的工钱,于是,他提出了辞职,并傲慢地说:“用来赔偿那支银匙子的钱,我不要了。”这样,“面子”没受到损伤。一个人放了一笔债,他知道他已不可能收回;于是他找到欠债人,给予严厉的威胁,以表明他知道该怎么做。虽然他没有收回债款,但他保住了“面子”,并以此来保证他将来不会再遇到此类事。一位仆人失职或拒绝做某些份内之事,当他知道主人打算解雇他时,他会故意再度犯规,并自己提出辞职,以保住他的“面子”。 

宁死也要保住面子,这对我们来说,似乎并没有多大的吸引力。但我们听说,中国的地方官享有一种特殊的恩惠,这就是被杀头时准许穿戴官袍,以保住他的“面子”。


\begin{quotation}
中国人总是在群居,或者现实生活环境也不断强调着群体这个事实。于是中国人从小到大要学会的技能就是如何通过撒谎来保全自己的隐私权,如何通过演戏保全自己的面子和如何通过欺骗等等各种手段来获得自己的利益。你可以把这种技能称之为人际交往的能力——一种通过戏剧的方式来生活的能力。这种戏剧最重要的部分不是你内心真的如何如何,而是认清某个角色具体的剧本地位设定,然后与之配合并按照一套行为法则(常见的戏剧行为模式)进行下去。

这套社会游戏规则最核心的关键词就是“地位”。你需要认清和你打交道的任何人的地位,好给权势者、尊贵者以应有的尊重,当然心情不好的时候也可以给那些微不足道的人物一顿颜色和脾气,不过那是细枝末节。最关键的是审时度势,保全自己,苟言欢笑,虚伪逢迎等等。这套技能并不是说欧美那边不需要,只是不似中国在历史上如此久的塑造着这个民族,这套礼仪其实和内心的礼貌修养完全没有关系,它就是一套社会生存法则,现实到用不好就要在社会上遭大殃背大运。

中国人个体的内心世界是通过读书而不是社交活动建立起来的,这一类人常被称作读书人。在中国很多这样的读书人是太善于此道的人,而这一类人在中国古代社会激烈的社会竞争中被没有被淘汰掉,否则中国可能并不能发展出一种像样的文明出来。在中国古代社会形成了一种奇特的途径来使用着这些人才,因为这些人大多正直勇敢诚实,而且多智慧通达之人。中国古代社会常常出现诸侯多个地方势力混战的局面,而一方诸侯要在这场混战中获胜和生存下来,就必须学会招纳和使用这些人才。这是很奇特的一种现象,那就是中国文明似乎是由两个社会拼凑出来的,而我们在回顾中国历史也会发现这个现象,那就是中国文明严格意义上分成两种文化:一种是治世文化,一种是乱世文化。

科举制度近似弥补了这种治世文化和乱世文化之间的鸿沟,并在更深的层次影响了这个民族的文化基因。但那都是读书人的事, 而之于面子和礼仪,永远都是中国底层民众苟活生存的不二法则。
\end{quotation}


\chapter{节俭持家}
“节俭”这个词表示持家的原则,特别是指家庭的收支关系。按照我们的理解,“节俭”这一词包括3个不同方面的含义:节制花销,制止浪费,用少花钱多办事的方式调节收支。无论从这3个方面的哪一方面来讲,中国人都算得上是杰出的节俭能手。 

到中国旅行,最初的印象之一是民众的饮食相当简单。众多的人口似乎全依赖于品种很少的食物,如稻米、各种豆子、谷子、青菜和鱼。这些再加上其他一些食物,就是亿万人主要的食品,也许逢年过节遇到特别的事情,再增加一点点肉。 

在西方各国,人们想方设法为生活非常贫困的人提供廉价且富有营养的食品,那么,你们一定也很想知道这个不容置疑的事实,那就是在中国平常的年份里,每个成年人每天花不到两分钱就完全能够得到足够量的生活食品。在灾荒的月份里,成千上万的人更是靠每天不到一分半钱的生活费来维持生命。这意味着中国人做菜做饭的水平是很高的。虽然在外国人看来,中国人的食物很少且不精,有些还淡而无味,甚至倒胃口,但又必须承认,中国人在准备饭菜方面是超一流的烹饪大师。在这一方面,温格罗夫·库克(Wingrove Cooke)先生把中国人列于法国人之下、英国人(也许还包括美国人)之上。中国人是否应该排在哪一个国家之下,我们不敢说得像库克先生那样确定,但他们在某些国家之上,这是无可争辩的。我在以前的一些小文章中已经说过,即使从生理学家的观点看,中国人对主食的选择也算得上是很高明的。中国人的食物原料简单,制成品却是花样繁多,其烹饪技术之完美,即使是对中国烹饪术一无所知的人也会有所听闻。 

另一件以往没引起我们注意却是很有意义的事实是,中国人在做饭菜时很少浪费,一切都尽可能做到物尽其用。在普通的中国人家庭,即使是每顿饭后的剩余饭菜,除了很少一些不值钱的之外,都留待下次再用。为了说明这一事实,只需看一看中国人的狗和猫的身体状况。这些家养动物靠人们的剩菜剩饭艰难地“活着”,一直挣扎在“死亡线上”。在新兴的国家中,浪费是众所周知的。我们猜想,像美国这样日子好过的国家每天所浪费的东西有可能够亚洲6000万人过上比较好的生活。我们确实希望看到这些剩余的东西不被浪费,而能使更多的人胖一些起来,正如许多中国人“吃饱”后,仍把剩余的饭菜整理出来以备下次再用一样,即使是杯中的剩茶也要倒回茶壶,以备下次热过后再喝。 

有一个时时处处都会引起我们注意的事实是,中国人对吃的东西并不像西方人那样过于讲究。什么东西都迟早会成为他们的网中之鱼,不能吃的东西几乎没有。中国北方,普遍使用马、骡、牛和驴干活,许多地区还使用骆驼。但我们看到,所有这些牲口只要一死,甭管是意外之死、老死或病死,一般都吃掉。无疑,这在我们一些读者看来,显然过于节俭了。在中国,牲口死了,把它吃掉,被认为是理所当然的,用不着大惊小怪,即使牲口偶尔死于像胸膜肺炎一类的传染病,也还是这样做。不过,在他们看来,这类病畜肉要比死于其他疾病的病畜肉更差,因些出售的价格也较便宜,但还是都卖了,都被吃了。人们清楚地知道,这类病畜肉进入人的体内,会产生某种疾病,但还是要花较少的钱去吃这种肉去冒险,无非是贪图便宜。当然,应当说,这是不常见的。死狗,死猫,也像死马、死骡、死驴一样被吃掉。我们曾不止一次地亲身了解村民故意用毒药把狗毒死后煮了吃的事。其中有一次,有人提议去问问外国医生,吃了这种肉可能产生怎样的后果,但狗“已下锅”。曾吃过这种肉而没得病的人不可能放弃这顿美餐,结果,在这次狼吞虎咽之后,居然仍是平安无事! 

有关中国人节俭的另一个事例,也与做饭有关,这就是为了有效地使用燃料在饭锅的制作上下了很多的功夫。在中国,燃料缺乏因而十分宝贵,一般有树叶,庄稼的根茎,这种燃料只有一把火就没了。为了适应这种情况,锅底要做得尽可能薄,因此用起来要非常小心。顺便说一下拾柴的过程,这也可反映中国人节俭到了极点。每个小孩,即使他还不会做其他什么事,但至少能拾柴。秋冬时节,到处是拾柴大军,他们手持竹耙,连一根干草也不留下。孩子们进入树林,用木棒打落秋叶,好像是在打落成熟的栗子,甚至树叶还在随风飘落时,一些心急的拾柴人就开始去“抓”了。 

所有的中国家庭主妇都知道如何最大限度地使用她手中的布料。她的衣服并不像西方国家同龄女性所穿的那样,在装饰和款式上过于浪费,而是尽可能地省时、省工和省料。在外国人看来是一片小得不能再小的布料,在中国妇女手里同样能派上用场,这是议会中“家庭经济”的女作家们做梦也想不到的。在一处不能用的,在另一处肯定能用,即使是一些碎布头,还可用来粘合成鞋底。伦敦和纽约的慈善家把自己不再穿的衣服送人,并不切实际地希望接受者不要因而成为单纯依靠救济过活的人。这样做弊多利少。但是,无论是谁把同样的物品送给中国人,尽管他们使用的布料和穿着风格与我们根本不同,但也可有理由相信,这些特殊的物品会得到充分的利用,即使一丁点儿也不会被丢弃,而与其他布料配在一起。 

中国人经常为朋友题词,题完词的纸被缝在一块绸布上。用线缝而不用浆糊粘贴,为的是让受赠者以后可以方便地更换绸布上的字,使他拥有一块可长期使用的绸布! 
 
中国人的节俭还可表现在小商贩的买卖中,对他们来说,再小的东西都会引起注意。例如,一个小商贩能准确地知道各种火柴盒中的根数,知道每盒火柴的蝇头小利。 

中国人的账簿用过后,每一片都还用的着,或是糊窗子,或是做纸灯笼。 

中国人处处节俭,就连确实必需的食品也尽量节俭。在他们看来,这没有什么不对,而认为是理所当然。B. C. 亨利博士在《十字架和龙》一书中给出一个很好的例子:三个轿夫抬着他走了5小时,23英里路,然后,轿夫们又回广州,去吃别人为他们提供的早餐。吃早餐前走了46英里的路,其中一半还是抬着轿,只是为了节省5分钱! 

还有一个例子。两个轿夫抬着轿子走了35英里路,然后撑船回去,自早晨6点起就什么东西也没吃过,舍不得花3分钱买两大碗米饭。后来,船搁浅了,直到次日下午2点才到广州。这些人已是27小时没吃东西了,负重走了35英里路,其中抬着亨利博士去广州走了15英里,还要再加上他的行李! 

对西方人来说,中国人节俭的做法确实很难苟同,但是,我们不得不承认这些做法完全是出于纯朴的天性。在这个国家的部分地区,特别是(说起来很怪)在北方,男孩女孩像是在伊甸园里,光着身子到处乱跑。也许对他们来说,这样更舒服一些,但主要还是为了节俭。中国人使用的独轮车相当大部分转起来咯吱咯吱地作响,只要加几滴油,车子就不会这样响了,但没人会这样做,因为对“心平气和”的人来说,咯吱咯吱的响声要比油更便宜。 

一位日本人侨居国外,他的特别要求是每天有热水洗澡,这是习惯。中国人也有澡堂,但绝大多数人根本就没去过,甚至连见都没见过。一位外国妇人看见一位中国母亲用笤帚掸去她孩子身上的尘土,好奇地问:“你每天都给你的孩子洗澡吗?”这位中国母亲回答说:“自他生下来就还没洗过。”对一般中国人来说,肥皂零售商就是把“比污垢更便宜”的字样贴在橱窗上,也不会引起他们的兴趣。 

中国人肯定把外国人看做是“浪费肥皂的人”,正如意大利人看英国人一样。在中国,洗衣服时所用的肥皂当然是少得不能再少了,洗过的衣服,与我们所说的清洁标准无疑还有距离。但是我们必须承认,这样做完全是为了节俭,因为许多中国人与我们一样,尽管生活条件非常不利,但仍喜爱干净,其中有些人还值得我们效仿。 

正是由于节俭的本性,要买任何现成的工具一般是不可能的。你可买到部分“半成品”,然后自己加工组装。自己加工总比买现成的便宜,正因为所有人都这么想,现成货当然也就买不到了。 

我们曾说到过许多中国人节俭的事例。比如,在普通的房子里两个房间的隔墙上开一个洞,洞中摆一盏几乎是不用花钱的小油灯,用以驱散两个房间的黑暗。在中国,诸如纺织、制陶、冶炼、工艺制作一类的小作坊里,也可以看到这种情形。这类作坊在我们看来,并不像是在表现中国人的节俭,更多的是在表现某种才能。中国人原本可以设计出更好的劳作方式,但是却没有人去改善它。他们似乎能够白手起家干一切工作,他们的产品,无论简单或是复杂,一般都有这样的特点。还有,他们的炼铁炉建在一个小院子里,总共就是那么一点点大,像是建一个大炉灶,一个小时就能用砖砌好,而且是长期地用下去,不花钱。 

在中国,说明中国人节俭的最好、最典型的事例莫过于对大量谷物贡品的管理,宁可说是完全缺乏管理。在中国,每年有大量的谷物贡品运到北京,这些贡品从天津起运至通州卸货。令“谷物交换”商吃惊的是,装卸、称量和搬运这堆如山的稻谷不是用机械,而只是靠一帮苦力。一些计量谷物的斗和数量不定的苇席,仅此而已。席子铺在地上,然后倒出谷物,称量,装袋,运走,最后收起席子,剩下的又仅仅是泥岸! 

在美国烟草种植园,最沉重的开支之一是建一个又长又精致的棚子,用来晾烟叶。而在中国种烟草的农村根本就没有这桩花费。晾烟叶的棚子是用茅草搭盖的,用过后,这些茅草与其他草一样仍是很好的燃料。烟叶摘下时,结实的叶柄依然留着,于是用草绳扎住叶柄,这样烟叶都连在草绳上,然后再把它们挂起来,正像衣服挂在绳子上一样。这样做真是再简便而有效不过了! 

每一居住在中国细心观察的人都能再补充一些有关中国社会生活的事例,但是,或许没有比以下更典型的事例:一位中国老妇人,步履瞒珊地走着,一打听才知道,她是去亲戚家,为的是死的时候能离祖坟近些,好少花一些抬棺材的费用。

\begin{quotation}
严格来说节俭并不能称之为中国人的一种天性, 有一个更好更合理的解释,那就是“穷”。乱世自不必说,中国普通民众即使在所谓的盛世也被各级官僚等等盘扣剥夺的差不多了,只是艰难苟活,不得不节俭持家。作为反驳,中国的那些富人们可是一点都不节俭的,奢侈浪费起来恐怕欧美有钱人都望尘莫及吧。
\end{quotation}

\chapter{勤劳刻苦}
勤劳是指习惯于勤奋地干任何工作——始终专心工作。在当今世界,勤劳是最值得高度赞美的美德之一,是永远受到尊敬的美德。 

一个民族的勤劳,大致可以由3个方面来衡量:长度、广度和厚度;换句话说,有两个外延和一个内涵。所谓长度,是指勤劳所持续的时间;广度是指真正可以算做勤劳者的人数;所谓内涵,是指“习惯于勤奋”和“始终专心工作”的能量。这3个因素综合起来形成一个整体。一般说来,偶尔到中国的旅行者与长期定居的侨民,他们所获得的印象是不尽相同的,但是他们都确实相信中国人的勤奋。初到中国的人,他们对中国人的第一印象是,这个民族的人正在履行约翰·卫斯理的格言:“全力以赴,始终如一”,在中国,懒汉是不常见的。每个人似乎都在忙什么。当然,也有不少富人,尽管他们在全部人口中所占的比例微不足道,他们可以不干事而过着富裕的生活。然而,他们的生活并不是外国人在表面上所看到的那种平庸。在中国,富人仍然像他在贫穷时那样专注于他的事业,一般不会放弃。 

中国人把自己分为:知识分子、农民、工人和商人。让我们分别看一看每一阶层所表现的勤劳。 

西方人很难接受像中国那样的教育模式。总体上的弊病是显而易见的,但仍有一点却总是引起人们的注意——这就是只为勤奋,不求回报。有钱人走后门买学位很可能会挫伤读书人的热情,但是官职买卖却不会。各省都在抱怨,每一职位的合格考生远远多于空缺的职位。所有各级考场,从最低的到最高的,都人满为患,经常是一万多人竞考一个职位。只要我们想想中国的读书人为了进入这样的考场而花费的心血,就会生动地感受到他们的勤奋。《三字经》中所提及的传统读书人的勤奋,借萤火虫的光亮读书,把书本固定在耕牛角上边犁田边读书,至今仍被中国各地成千上万的人所努力效仿。在许多情况下,不少人一获得初步的成功就开始放弃往日的勤奋,但是中国人根本不把此类人当做读书人,而把读书人这一荣耀的称呼留给那些在充满荆棘的狭窄小路上不断奋斗直至功成名就的人。除了中国,我们又能在什么地方可以看到祖孙三代为了谋取同样的职位参加同样的考试,经过同样长时间的不屈不挠的努力,终于同样都是在80岁时获得盼望已久的荣誉? 

1889年春天,北京的《邸报》上披露了各种材料,其中有关于省级考试中老年考生的事。福州总督报告说,福州秋季考试中,有9位超过80岁和2位超过90岁的考生通过规定的考试,他们的考试文章,结构严谨,文字书写有力、准确。他说,这些老年考生中秀才后已过60年,在此期间已参加了3次晋级考试,如果第四次再不成功,当授予名誉头衔。河南总督以同样的方式报告说,有13位超过80岁和1位超过90岁的考生,他们全部“通过为期9天的严格考试,文章精练,并没有表现出暮年的痕迹”。但更令人吃惊的是安徽省,那里有35位考生超过80岁,18位超过90岁!还有哪一个国家会有这种奇观呢? 

如果说中国的知识分子是始终勤奋的一族,那么农民的勤奋则并不亚于他们。农民的劳作如家务一样,没完没了。所有北方各省,除了冬至前后有一段相当短时间的空闲外,一年到头就根本没有闲的时候,要做的事很多。无疑,其他各国的农民也多少有点类似,但是,中国农民的勤劳是其他民族所难以比拟的。 

农民是这样,雇农更是这样,雇农们长期过着极度贫困的日子,在无尽的折磨中度过一生。正如农民细心照料他的每一棵白菜,仔细清除各种害虫,雇农也同样要照料好他的工作,以便能填饱肚子,养家活口。那些需要出远门的人,往往是半夜起身赶路;他们说这是习惯,无论何时,在路上都可看见手拿叉子,肩背筐子的农民在拾粪。当没有其他事可做时,这是一桩不变的、永远做不完的事。 

经常可以看到有些人为了养家活口寻找两份不同的工作以互相衔接。天津的船夫在河水封冻没事干时,就拉冰撬,搞搬运,赚点小钱。同样,一些地区的农民在农闲时,都是在编制帽子,这种帽子还是大宗出口产品。中国妇女往往是不停地手纳鞋底,几乎看不到她们闲着;即使是在村口聊天也是如此;除此之外,她们或许还搓棉花,以备纺纱。总之,她们从不偷懒。

商人及其雇员完全可称得上是一个不知疲倦地工作的阶层,商店职员的生活,即使是在西方,也不是清闲的,但中国店员要更加忙碌,他们的工作永远没个头。他们几乎没有节假日,任务繁重,只是在精神麻木时才可稍停片刻。 

中国的店铺开市很早,收市很晚。簿记制度采用一种复式簿记法,非常细致,使得账房为了记录收支和平衡账目而常常忙到深夜。店员们在无事可做时,就坐下来挑捡收进来的铜钱,看看有没有值钱的铜板。 

令人吃惊的是,在中国,工作最艰苦的阶层是最让人羡慕、每一有志者都设法跻身于其中的官僚阶层。中国的各级官员必须埋头于各种公务,必须对每桩事的成功负责到底,而这类公务之繁杂,同样令人吃惊。以下摘自北京外国使馆的一位翻译对中国重要政治人物的采访报告:“我曾经询问过一位中国内阁大臣,他一直抱怨日常事务的繁忙使他过于疲惫和劳累。他说,他每天凌晨两点钟从家里出发,因为3点至6点他要在宫里值班。作为内阁大臣,6点至9点他要在朝中议事。他是兵部大臣,9点至11点要在兵部。他又是刑部的要员,每天12点至下午2点要在刑部办公室里。他还是外务部的资深部长,每天下午2点至5、6点要在外务部,这些就是他每天的工作安排。此外,他在工作的空隙还经常为其他各部门工作。他很少在傍晚7、8点之前到家。”我们的工会为实行每日8小时工作制而奋斗,当看到上面这每日的工作安排时,又会有什么滋味呢?据说,那位官员在那次谈话后6个月因劳累过度,心力交瘁,去世了。这并不奇怪,在中国,那些仍能为政府效劳的官员身上此类事的再度发生并不是不可能的。

前面我们已经说过,所谓勤劳的外延是指勤劳者的人数以及勤劳所持续的时间。正如我们所看到的,中国人的勤劳在外延方面是很广大的。中国人的一天开始于天刚蒙蒙亮之时,往往是半夜后不久,中国的皇帝每日上朝时,欧洲各国的宫廷还沉睡在睡梦之神的怀抱里。这对西方人来说简直是不可思议,而对中国人来说则是最自然的事。天子的所作所为不同程度地受到各地臣民的效仿。广州的铜匠、福州的锡匠、宁波的木匠、上海的磨坊工以及北方各省的纺织工和磨面工都是睡得晚、起得早。天还没亮,旅行者就会在集市上遇到卖菜的村民,他们早已从数里路以外的家里赶来,站在黑暗之中等待着天亮。西方人吃早饭的时候,中国人的早市已经结束。夏季清晨5点半后,沿着上海的主要街道漫步,没有比这里更能感受到东西方的强烈反差了。在黄浦江边建起高楼并在里面做生意的欧洲人还根本没有动静时,而亚洲人却已是经过了很长时间进入了高潮。几小时后,当西方人开始轻松自在地与中国人抢占市场时,当地人已经干完了半天的活。 

约翰·戴维斯先生曾相当正确地评论中国人的“热爱劳动”,他说,这标志着中国政府在使人民满足于自己的环境方面是成功的。这种热爱劳动的品质是中国人最显著的特性之一,必须受到长期的高度重视。 

关于中国人勤劳的内涵,仍需说几句。中国人是亚洲人,他们工作也像其他亚洲人一样。试图把我们的模式强加于这个生机勃勃的民族,那只能是徒劳的。在我们看来,他们显然缺乏我们所高度尊崇的诚心。盎格鲁-撒克逊人用不着基督教《圣经》的指点便能知道尽力做好自己应当做的事是非常的重要,但是,成熟的宗教和哲学虽然可以对中国人产生影响,却不能使他们改变步伐。他们受益于几千年以来所积累的经验,正像荷马之神,他们从来就是不慌不忙。 

人们不禁要问,当有朝一日白种人与黄种人进入空前激烈的竞争之时,谁将败下阵来? 

所罗门的经济学格言“勤劳致富”假如是正确的,那么中国人应当是地球上最富裕的民族,而且毫无疑问他们也将是的。\uwave{只要他们能够进行道德的再权衡,将某些他们明显缺失的品质,这些品质在他们所谓的“传统道德”中倒是有所提及,不过“总是”缺失的。那个时候,正直和真诚的品格,将会重新回归到中国人们的道德意识中去。那么(不会太久)中国人们就会收获因他们的无比勤劳所带来的全部奖赏。}

\begin{quotation}
作者在这里其实提了一个非常重要的观点,视图解答这个问题:那就是为什么中国人是地球上最勤劳的却大多不富有。作者认为正是中国人缺乏正直和真诚的品格造成了这点。我是部分认同,但现在大多数人都不会认为缺乏正直的品格何以造成的勤劳不富有的困局之间存在联系。我们需要深入的思考。

前面说了科举制度部分弥补了中国治世文化和乱世文化之间的鸿沟,但只是暂时弥补,到后面科举制度进一步腐化堕落,面子和地位的游戏也开始渗入科举制度之中,从而排挤那些本性上不善于此道之人, 于是中国文明后期的社会文化逐渐形成,这个时候面子和地位的社交游戏完全主导全局,那些本性正直诚实的人正在逐步被社会淘汰或者在社会底层挣扎,这个在清朝晚年已经可见一般。当曾国藩年青的时候意气风发的时候只是到处碰壁罢了,于晚年老滑之中似乎有得势成功之想,不过是同流合污做挣扎状罢了;当李鸿章磨砺自己的厚黑学,于晚年似乎小有所成的时候,亦不过是做裱糊匠罢了。真正的读书人要某落魄要某委屈自己的本性艰难学习为人处世之道,如同屈原一般企图听渔夫之言啜其醣而扬其波最后不过杯水车薪罢了。

文人志士于权势社会无所依托,尤牢记先人教训,奋发图强奈何无能为力而多郁郁而终之辈;底层平民于艰难世道下苟活乃勤劳不过伪装或麻痹或追逐近利。整个民族最后无正直诚实之人敢言黑即是黑,白即是白,而尽是弄权之人指鹿为马。智慧首先是要认清事实,首先是要有正直之人敢言黑即是黑白即是白,如果这基本的一步都无以谈起,那么你跟我谈论科技跟我谈论创新谈论自由的文学?在我看来不过痴人说梦罢了。

没有智慧独独剩下勤劳,忙了一辈子而不富有,你问我为什么?我说没脑子的不配享受富有幸福的生活 ,上帝也看清他们。
\end{quotation}


\chapter{讲究礼貌}
关于中国人和东方人的礼貌人们有两种很不相同的看法:一个是赞赏,另一个是批评。我们盎格鲁-撒克逊人,喜欢这样提醒自己,(无疑)是有很多美德的。而在这些美德中人们发现更多的是\textit{内刚},只有少部分的\textit{外柔}。因此,当我们来到东方,发现有那么多的亚洲人在调解人际关系方面具有比我们高得多的技巧,内心不由得充满羡慕。这是不会做某事的人对于能轻松做这桩事的人的一种羡慕。即使是对中国人具有过分偏见的批评家,他也不得不承认中国人已经把礼貌升华到了一个完美的高度,这对于西方人是不知道且未曾体验故想不到甚至几乎不可想像的。 

这使我们想起,中国的典籍上记载有礼仪准则300条,行为准则3000条。一个民族背负着如此繁多的礼节,要延续下去似乎是不可想像的。但是,我们很快就发现,中国人已经设法把恪守礼节熔铸成一种内在的本能,而非外在的需要,就像他们对待教育一样。这个民族的精英,曾为人们的日常交往制定出繁文缛节,而在西方,这一切只适用于宫廷和外交往来,当然,中国人的日常生活并不是完全被这些繁文缛节所束缚,而是这些规矩就像节日的盛装,该搬出来时就得搬出来;至于在什么场合需要这样做,中国人全凭一种准确的本能去辨认。在这样的场合,如果一个中国人不知道该如何去做,那么他就会像西方一个受过教育的人偶尔忘了9乘以9是多少一样,令人感到滑稽可笑。 

西方人难以欣赏中国人的礼貌,是因为我们心中有一个观念,那就是礼貌的定义是“一种真诚的善意的表达”,因为文明社会教导我们,对待个体人的福利如同(理论上)大家的福利一样。但是在中国礼貌则完全不是这样的。在中国礼貌更像是一种仪式的技术细节问题,就像所有的技术一样,重要,但不是作为心态或良心的某个指数,而是个体如何成为复杂的整体的一部分来研究的。礼貌用语的制定和使用,目的只在于维护既定的尊卑关系。这在西方人看来,即便不令人发疯,也会令人不知所措;而在中国人看来,这对于保障社会秩序是至关重要的,而且也是调解人际关系的润滑剂。有前就有后,有后也有前;该前的前,该后的后,各得其所,万事亨通。就像下棋一样,先走的必须说:“鄙王先走一子”,然后,对手说:“鄙王也走一子”。后来,对手事先告知说:“鄙王的士要吃尊王的卒,走到鄙王卑贱的象位”。这就是在下棋。一局棋的输赢与说几句客套话毫无关系,但是,正如下棋人不能事先说出下一步棋,否则就会使自己显得荒唐可笑一样,中国人对于对手的每一步棋若不能给予有理有节的回应,也会成为人们的笑柄,因为对中国人来说,客套即是下棋,不懂这些客套就等于无知。 

同时,中国人讲究礼貌的严格程度,是有城乡差别的。一个乡下人,虽然他知道必须有礼貌,但他并不知道像城里人那样的礼貌有哪些具体要求。 

但同时必须承认,即使中国有难得一见的不懂礼貌的人,他们也要比最有教养的外国人强得多:与他们相比,外国人只是怀中的婴儿。一般说来,除非外国人曾有过长期的生活体验,又担心自己有所失礼而被误认为没有教养,否则,他就不可能有中国人那样的礼貌之举。外国人并不懂那么多的“规矩”,即使学会了那些漂亮的礼貌用语,也表现出那样的麻木和无知。正是由于外国人在仿效中国人的最起码的礼貌方面表现出明显的无能和自愧不如,所以中国的知识界总是带着一种毫不掩饰的(自然流露的)轻蔑目光看待这些“野蛮人”。 

礼貌可以比做一个气垫。里面什么东西都没有,但它却能够很好地减缓颠簸。同时,还可公正地说,中国人向外国人所表示的礼貌(完全与向自己的同胞所表示的那样)更多的是为了显示自己懂得如何待人,其次才是考虑到客人是否舒服。你本不想生火烧水沏茶,他偏偏要为你生火沏茶;结果让你被烟熏得流泪,呛得喉咙像是在吞苦药;而主人仍然自信自己知道该如何待客,至于客人不乐意,那完全是客人自己的事。再比如,你在乡下租了一间较差的房子,房子的主人认为把房间打扫一下(象征性地)布置一下是他的职责;你已经来到了房间,他仍然还在打扫;飞扬的灰尘弄得你睁不开眼,你恳求他别做了,但他仍然继续做。也许,这就是《礼记》上所教诲的,应该为客人打扫房间,不管客人是否乐意。请客吃饭也有这样的礼节,这是令初来乍到者生畏(而有所涉足者常见)的礼节。在请客吃饭时,热情的主人特地为你盛上一大堆他以为你喜欢吃的东西,而不管事实上你是否喜欢吃,是否吃得下。若是你一点也不想吃,主人似乎会说,那就是你的不是了,而主人则肯定他自己并没有失礼;也没有人会说他失礼。如果外国人不懂这种游戏规则,那是自己的事,与主人无关。 

正是按照这个原则,一位中国新娘照例去拜会一位外国夫人时,她背朝夫人,向着完全相反的方向行礼,结果使女主人感到奇怪和生气。事后经过询问才知道,新娘朝北行礼是因为那是皇上所在的方向,她并没有在意女主人是在房子的南边。如果这位外国夫人不清楚自己应当站在房子的什么位置,那么这位新娘也就不必在意女主人会怎么想;至少她表明自己知道应当朝什么方向磕头! 

中国人的礼貌常常表现在送礼上。如前所述,这是给受礼人“面子”。所送的礼物有某种固定的老式样。一位常与中国人打交道的外国人,总会收到一些礼品盒,这些礼盒用红纸包裹得清清楚楚,内装油腻的糕点;即使他根本不可能吃,甚至被逼得走投无路表示不愿意接受,否则他只好拿去送人,送礼人还是不会把礼品拿回去。 

中国人的礼貌决不是不允许人们“对礼貌吹毛求疵”。受礼人经常会问,这些礼物花了多少钱。到别人家做客的人在与男女主人告别时常说:“给您添麻烦了,让您破费了!” 

一位外国人曾应邀参加一次婚礼。婚礼上糕点丰富。婚宴进行中,端上一盘糕点,仅有两三块,热气腾腾而受到夸耀(似乎都喜欢热的)。由于这位外国人是贵客,这盘糕点端给了他。但他却婉言谢绝了。不知是什么道理,这给正在进行的婚宴投下了一片阴影,那盘糕点后来没传给其他人,而被撤了下去。原来,按照习俗,每位参加婚礼的客人都要送一份礼钱当做婚宴的花费,照例是客人还在席上时就开始收钱,但在中国人看来,向客人收礼钱是不礼貌的,于是就以向客人送热糕点为托辞。每个人都知道送热糕点的用意,唯独这位外国人蒙在鼓里,他的拒绝使得其他人不便在当时拿出自己的礼钱。后来,他又应邀参加这一家举行的另一次婚宴。这一次,婚礼主持人,鉴于上一次昂贵的经验,对着客人比西方人还直截了当地说道:“要送礼金的请进来啊,在这里!”

在否定了中国人礼貌中令人厌烦的繁文缛节之后,我们仍要在社会交往方面向中国人学习许多东西。保持我们的诚实,抛弃我们的鲁莽,这是完全可能的;如果西方人的坚定的独立性掺入一定量的东方人的温文尔雅,那一切将会更好。 

然而,许多西方人根本不会用这种观点看待事物。笔者的一位熟人在巴黎住了许多年,以致于不知不觉地接受了那里的风俗习惯。当他后来回伦敦时,他已习惯于向见到的每一位朋友脱帽鞠躬。有一次,他向一位朋友鞠躬时,这位朋友嘲笑他说:“老朋友,请看,这里没有你的法国猴子可耍!”如果人们能够集东西方之精华于一体,能够安然地走在狭窄的、荆棘丛生的中庸之道上,那该是多么的惬意!


\chapter{漠视时间}
“时间就是金钱”,这是当今发达国家流行的一句格言。现代社会生活的安排极其复杂,一个商人能在特定的商务时间里处理好大量各种商务,这在上个世纪需要花费多得多的时间。蒸汽机和电力已经完成了一场革命,盎格鲁-撒克逊人曾以其身体素质为这场革命做了预先的准备。虽然我们的祖先曾无所事事,只知吃喝和决斗,但无论如何,我们的民族是具有冲劲的民族,这种冲劲驱使每个人无休止地去做一桩又一桩事情。 

中国人的问候语与盎格鲁-撒克逊人的问候语之间存在着一种很有意思的差异。前者遇到他的同事时说:“吃饭了没有?”后者则问:“做得怎么样?(how do you \textit{do})”做,这是人的正常行为,正如中国人看待吃一样。由此可见,感觉到时间就是金钱,一秒钟也不可放过,这已经成为我们的第二天性;而中国人,像大多数东方人一样,则是非常地浪费时间,中国人的一天仅有12个时辰,一个时辰与下一个时辰之间并没有明确的界线,只是象征性地把一天分为12个部分,他们所说的“中午”是指上午11点到下午1点之间。我们可以听到一位中国人在问:“现在是什么时候?”“现在是半夜什么时候?”这里的语言多少有点模棱两可,他应当进一步问:“现在是半夜几点?” 

在日常生活用语中,说到时间时,几乎都有类似的不确定,中国人所说的“日出”和“日落”,就其用来指称太阳所处的纬度(还有经度)而言,还算精确,但“半夜”,正如“中午”一样,并没有具体的时间所指。夜里的时辰通常用“更”来划分,同样模糊不清。只有最后一更,由于是在晨曦初露之时,才较为精确。即便是在城里,“更”所指的时段也多少有些不确定。我们所说的表,绝大多数人一无所知。有些人确实有表,但在他们当中,即使有人每隔几年将表清洗一下,以保证它正常运转,也几乎没有人会用表来安排自己的活动。普通人完全是根据太阳的高度来知道时间的,而把太阳高度说成是一杆子高、两杆子高,或几杆子高。若是遇到阴天,就根据猫眼睛瞳孔的放大和缩小来知道时间,对于日常生活,这已是够准确的了。 

中国人对时间的利用是与他们对时间测定的不精确相对应的。根据西尼·史密斯所说,世上的人分为两类,大洪水前的人和大洪水后的人。大洪水后的人发现,人的年龄再也不可能达到几百岁,更不可能近千岁,所以他们学会抓紧时间,以适应环境。相反,大洪水前的人不可能意识到长寿的梅修撒莱时代已经过去,他们的生活仍然依照家族的成规进行安排。 

中国人可以算做是“大洪水前的人”。中国的说书人,比如在茶馆里为吸引和留住顾客的那些说书人,会使人想起英国诗人丁尼生的“布鲁克”。听众可以随便来去,但他却是“没完没了”不停地说。演戏也是一样,有时,一场戏要接连演上好几天,当然,还是不能与泰国的戏相比,据看过泰国戏的人说,他们接连看了两个月。中国人的戏法,技艺高超,且有趣,但也有一个致命的弱点——他们总要向观众说一大堆空洞无聊的开场白,以致于外国观众还没看戏就已经后悔当初不该来。最可怕的是出席中国人的酒宴,其持续的时间之长几乎是没完没了。酒菜的数量之多、花样之繁,几乎难以置信。所有经历过这种场面的外国人都会感到恐怖和不知所措。而对中国人来说,这种招待所花费的时间还嫌太短。中国人有句最让人回味无穷的格言,这就是“世上没有不散的宴席”。但是,被诱人圈套出席这种酒宴的野蛮人却感到,这一原本可以为他们带来一线希望的格言,在这种场合总是难以实现的。

中国人从小就完全习惯于依照大洪水前的成规行事。上学的时候,他总是一天到晚读书,只有吃饭的时候才停一下。除此之外,无论是学生或是先生,都不知道还有其他的读书方式。科举考试要进行几天几夜,每一关都不是好过的。尽管大多数考生对这种不合理性的考试感到厌烦,但他们仍然相信这种考试对于检验人的知识才能还是有道理的。 

这种教育所产生的结果会使人联想到其形成的过程。中国人的语言基本上是属于大洪水之前的,掌握它需要梅修撒莱毕生的时间。与古罗马人一样,古代中国人意识到,若不自觉学习他们自己的语言,就永远不会说或写!中国人的历史是属于大洪水之前的。它可追溯到太初时代,尔后,则是混浊、舒缓、漫长的大河,其间,不仅有挺拔的大树,也有枯朽的草木。除了较缺乏时间观念的民族之外,没有人会去编写或阅读这样的历史;除了中国人的记性之外,没有人会有这么大的“肚子”能装下它们。 

中国人漠视时间还表现在他们的勤劳之中,正如我们在前面已经说到,中国人勤劳的内涵完全不像是盎格鲁-撒克逊人劳动时所表现的那样。 

曾与中国的承包人和工匠合作建房子的那些外国人,有多少希望再度合作呢?这些中国人来得迟,走得早,老是停下来喝茶。他们用布袋从很远的石灰坑里一袋一袋地运灰浆;若是用独轮车,一人可抵仨;但是谁也不这样干。如果遇到一点小雨,所有的工作还要停下来。这样,花费的时间不少,进度却很慢,往往很难看出这帮人每天到底干了多少活。我听说,有个外国人对他的木匠钉板条的缓慢进度很不满意,于是趁他们吃饭时自己动手干,结果完成了4个木匠半天的活。 

对中国的工匠来说,修理他们自己的工具也是桩很花时间的大事。然而,如果工具是外国人的,那就另当别论了。一件工具莫名其妙地坏了,但没人承认曾经动过。“没人动过”,这是一句很适合于他们的口头禅。在墙上插一些木条,用绳子捆绑一下,就算支起脚手架。整个工期,天天都有危险。不管干什么事,都没经验。沙子、石灰和当地的泥土原本以为都可以用,结果都不能用。外国人没办法了。他就像《格利佛游记》中所描写的被线牵制着,这些线凑在一起,对他来说实在是太多了。我们一直会想起一位广东的承包人。他是个鸦片鬼,他的允诺正像他的钱一样统统消失在鸦片烟中。最后,忍无可忍,只得把一些实在过分的问题摆到他的面前,“告诉过你玻璃的尺寸,你也量过窗子好几遍,可是你统统搞错了,都不能用。你做的门一块都合不上,一点胶也没用过。地板不够长,数量也不够,还都是节疤孔,而且没有彻底干透。”听着这番指责,那位脾气温和的广东人似乎有些可怜,然后又用一种文雅的语调抗议说:“不要这样说!不要这样说!这样说有失体面!” 

对中国人来说,盎格鲁-撒克逊人经常性的急躁不仅是不可理解的,而且完全是非理智的。很显然,中国人不喜欢我们的人格中所具有的这一品性,正如我们也不喜欢他们缺乏诚实一样。 

无论如何,要让一个中国人感到行动迅速敏捷的重要性,那是很困难的。我们曾听说,一大包外国邮件在相距12英里的两个城市之间被耽搁了好几天,原因是邮差的驴病了,需要休息!中国邮电系统的管理还只是停留在应该怎样与能够怎样的模仿阶段。 

最使外国人讨厌的是,中国人在社交访问过程中对浪费时间的不在乎。在西方国家,这样的访问是有某种时间限度的,他们不会超过时间。但在中国,则没有这样的限度。只要主人不提出要为客人安排食宿,客人就是精疲力尽了,也还是要说下去。中国人在访问外国人时,根本不可能意识到时间的宝贵。他们会一连坐上好几个小时,一个劲儿地说,不知在说些什么,也不说要走。一位高明的牧师有句格言:“想见我的人,也是我想见的人。”假如这位牧师在中国待过,无论时间有多长,他都会对他的这一格言进行实质性的修改。当他碰到上述的那种事之后,肯定会效仿另一位很忙的牧师,在他的书房醒目地挂上一条圣经中的格言:“主保佑你离开!”如果对一位正说到兴头上的中国人明确表示自己很忙,那往往会给他当头一棒。他会长时间地一言不发,默默地忍受着,其时间之长足以消磨掉10个欧洲人的耐心。终于他开始说话了,这正如谚语所言:“上山打虎易,让你开口难!”如果外国人都像已故的麦肯齐博士那样就好了。他觉得他的中国朋友不断前来做客,并且“只来不走”,浪费时间,影响到他的工作,于是习惯性地对他们说:“请坐,像自己家一样;我正忙,请原谅。”假如他能够模仿一位中国学生说得直截了当、简明扼要,那就更有意思了。那位中国学生学了一些短语后,想在老师身上尝试一下,于是下课时大声说道:“开门!出去!”结果,弄得老师差点晕过去。



\chapter{漠视精确}
中国人给外国人的第一印象是千人一面。他们的相貌似乎出自同一个模子;穿的总是蓝色;眼里无神,好像发直了一样;辫子像是同一个豆荚中的两粒豆仁,一模一样。但是,无论把中国人说成是怎么样,即使是最不善于观察的旅行者,只要略加体验就会发现,所谓中国人是千人一面的说法是不能成立的。两个地区,无论多么靠近,口音上都存在着有趣的、莫名其妙的差异。而且地区间相隔越远。差异就越大,以致于形成不同的“方言”。经常有人告诉我们,中国人的语言写起来都一样,说起来却大不相同。我们常常想到,中国人的风俗习惯也有同样的差异,按照中国人流行的说法,十里之外不同俗,这种事例,随处可见。然而,最常见的是计量标准上的差异,而在西方国家,绝对不变的计量标准才能保证生活的舒适。 

任何双重标准,对西方人来说都是令人烦恼的,而对中国人来说却是乐此不疲。两种货币单位、两种重量单位、两种度量单位,这些对他们来说似乎很自然、很平常,不必予以反对。向一位卖肉馅汤团的人打听每天做多少这样的汤团,他回答说,大概“一百斤面粉”,至于这些面粉能做成多少汤团,这个问题只能留待询问者自己去猜想。同样,向一位农民打听他的一头牛有多重,如果他给出的数太低,相差太大,他会解释说,这个数不包括骨头!问一位职员身高是多少,如果他给出的数与他的实际身高相差的太离奇,经过查问,他会承认他给出的数没把头部算在内!原来,他曾当过兵,在部队分配挑担时,人的锁骨的高度比较重要,因此,说自己的身高时一直就没有把头部算在内,这次他是疏忽了。一位乡下人的计量方式就不同了,他硬说他的家“离城90里”,但经过盘问,他承认没那么多,他说的是往返的路程。实际距离只有“45里”! 

在中国,计量不一致的最明显的事例是计量铜钱的方法。铜钱是这个国家唯一的货币,各地都采用十进制,这也是最简易的计量方式。但是,谁也不能保证一串钱无论在什么地方都是所预想的一百个铜钱,除非他特地算一下。他不必走遍18个省份的绝大多数地区,就会发现一串钱的铜钱数目各不相同,而且无法解释。按道理,“一串”就是1oo个铜钱,但事实上从100至99、98、96、83(如陕西省会)、直到直隶东部的33,各种数目都有,或许有些地方可能更低。银子买卖中的称重也是这样,甚至更为严重。各地的“两”都不一样,除非是巧合。这种情况把外来人搞得稀里糊涂,除了那些专门买卖银子的人外,谁都会遭受一定的损失,特别是会给那些老实人带来无尽的烦恼。这种货币混乱的动机是显而易见的,但我们眼下关心的只是存在的事实。 

所有各种计量都有同样的混乱。一个地方的斗不同于其他任何地方的斗。如果在粮食征税中总是采用这种斗,那么,就很容易在那些不像中国人那样会忍气吞声的民族中引发政治动乱。至今为止,“全世界都是一品脱是一品脱,一磅是一磅”;而在中国却是一斗不是一斗,一斤不是一斤。这种混乱居然还有道理可言。而且,到处还可看到(例如在盐业专卖中)纯属随意武断的标准,比如把十二两叫一斤。购买者买的是十六两的一斤,得到的却只有十二两;而且还是公开这么做的,同行的商人也都这样;据说这不是欺诈,只不过是盐业买卖中的“老习惯”,百姓们完全清楚。土地的丈量中也普遍存在类似的不确定性。在某个地区,一“亩”土地只相当于其他地区的一半,如果有人碰巧住在边界线上,那么他们就不得不准备两种丈量土地的工具,以分别用于不同的田亩制。 

要知道每斤粮棉的价格,仅凭现有的报告(正像到中国的旅行者经常做的那样)是很不可靠的,还必须首先弄清楚这里的“斤”是指哪一种斤。同样,要知道每亩的粮食产量,不能仅凭现有的统计数字,还必须弄清楚这里的“亩”是哪一种亩。在计量路程的距离方面,也普遍存在着类似的问题,每位到中国的旅行者都会有这种体验。在陆地旅行中,如果路程是以“里”计量,那么就有必要弄清楚这个“里”是不是指“大”的里!我们并不否认这样计量路程有某种根据,但我们要指出的是这种计量既不精确也不统一。据我们所知,人们普遍感觉到,一离开宽阔的国道,“里”就变“长”了。如果在国道上每天能走120里路,那么在乡村的小道上满打满算也只能是每天走100里,而在山区,就只能是每天走80里。此外,测算路程的长短往往不是根据实际的绝对距离,而是根据行走的难易程度,甚至中国人也不否定这一点。因此,若是说到山顶有“90里”,实际的里数还不到一半;而中国人还强词夺理硬要坚持,理由是走这段路程的困难程度相当于在平地上走“90里”。还有一件与测算长度有关的稀奇事,即从A到B的距离不一定等于从B到A! 在中国,欧几里得的假定“与同一量相等的量彼此相等”已不起作用了,需要插入一个否定词加以修改。我们可以举例说明:在中国最重要的一条公路,有一段路根据里程碑所示从北到南长183里,而从南到北却是190 里。真是太奇怪了,无论你走几趟,也无论多么仔细地看里程碑,事实就是这样。”

在写这段之前,我们已经在巴伯先生的《中国西部之行》一书中看到过类似的事例:“比如,我们说两地间的距离是根据两地一来一往的人们所估算的而定;这样,各人给出的数字当然不会相同,从A到B的人都说是1里,而从B到A的人却都说是3里。当地一位有知识的人解释说:运费是按里计算的;显然,上山时,苦力应得到更高的报酬;若是按照路的坡度来计付报酬那是很麻烦的,为了方便起见,就把难走或陡峭的路说得更长一些。原来如此。眼下,这些约定俗成的里程数就是旅行者一直想弄明白的。“但是”,我反驳说,“按照这种说法,雨天也肯定要加长里程数。晚上的里程数肯定要比白天的更长”。“很对,是要多付一些钱才行。”这种做法对当地人来说可能是方便的,但旅行者却会不断遇到麻烦。像这样估算路程的事还有:平地上,1法定哩被说成2里;不是很陡峭的一般山路,1哩说成5里;很陡峭的山路,1哩说成15里。一位云南的山里人,他老是少算了平地上的路程,但在他所住的山里就没这样。以后的旅行者对此不必大惊小怪。只要不是很陡峭的山路,他肯定都会把5里说成是1里。” 

在利特尔先生的《长江三峡之行》一书中,他说,有一段水路,顺流而下时说成是90里,面逆流而上时却说是120里。他估计是3.62里相当于1法定哩,或者说,250里相当于1纬度。★ 

与此类似的是,“整体等于部分之和”这一公理在中国也不能成立,尤其是在河道航行中,你通过打听知道到前方的某一地点有“40里”,然后,通过更细致的分析,你才知道这个“40”原来是两个“18”;“4个9是40,不对吗?”这种说法会让你哑口无言。照此说法,“3个18”就是“60”。我们曾听说过一件事,一位政府通信员在规定的时间内没有跑完规定的路程,他为自己辩解说,这个“6O里”是“大”里。由于他的申辩合理,他的上司下令测量这段路程,结果发现实际上是“83里”,从那以后,就一直按此计算。 

分布在一座城市周围的几个乡村,离城里从1里到6里不等,但每个村子都可以叫“三里屯”。人们经常可以看到,据估算只有1里的路程,如果道路两旁有房子,就会被说成是5里,而且每个村民都会认真地向我们保证,这条街确实是那么长。 

在这些情况下,各人可以根据各自的需要制定标准,大可不必为此大惊小怪。造秤的人徘徊于街上,根据每个客户的偏好在秤杆上镶上秤星。每个买卖人至少有两种秤,一种是用于买的,另一种是用于卖的。他们不买现成的秤,除非是杆旧秤,因为情况在变,秤的标准只能根据每个买卖人的需要而定。 

说人的年龄大小也是如此,其中尤其能反映出中国人的民族特性。凭着一个人出生年的动物属相,就能轻易地推测出这个人的年龄,这是再普通不过的事了。说一位老人有。“七、八十岁”,其实是去年才满七十岁。事实上,在中国,一过七十岁,就是“八十”的人了,如果想弄得准确,就必须减去这个“常数”。即使一位中国人说出他的准确年龄,所给出的也只是下一个春节后的年龄。用“十”为单位计算岁数的习惯根深蒂固,并且搞得很模糊。一些人是“一、二十岁”,“没几十岁”,或许“好几十岁”;在中国,严格准确他说出年龄是非常少见的事。这种模糊还延伸至“百”,“千”和“万”,“万”是中国人计算的实际限度。对于比这些笼统说法更加准确的表达,中国人并不感兴趣。 

一位熟人告诉笔者,有两个人花了“二百串钱”看一场戏,后来又补充了一句:“是一百七十三串,不过,这与二百串是一样的,不是吗?” 

一位绅士及其夫人在中国生活了好几年后要回国时,他们的中国朋友送来两幅卷轴,是要分别转赠给他们夫妇俩各自的老母亲--父亲都已去世,他们夫妇俩各自的老母亲恰好同岁。两个条幅上的题字分别是“福如东海”和“寿比南山”,而且每个条幅旁边还有一行小字。其中一个条幅上的小字恭贺受赠者享受了“七十年的富贵”,另一个条幅则赞美贵夫人享有“六十年的荣华”。夫妇俩在对这两幅卷轴大加赞赏之余,其中一位战战兢兢地问道,为什么明明知道两位母亲是同岁,却还要说一个是七十,另一个只有六十呢?得到的是一个很有特点的回答:如果每幅卷轴上都是写“七十年”,那会显得作者太缺乏想像力了! 

中国人讲究社会连带关系,这对我们所要求的精确是致命的,一位打官司要求咨询的人告诉笔者,他“住”在一个村里,但从他的口述中可以明显知道,他的住所是在城郊。经过查问,他承认眼下是不住在那个村里;经过进一步的调查才发现,他十九代之前就已搬出该村了。问他:“你难道不认为你自己现在是城市居民吗?”他简单地回答说:“不,我们现在的确住在城里,但我的老家是在那个村甲!” 

还有一个人曾经要求笔者去看看他村里的一座古庙,他骄做他说:“那座庙是我建的。”经过进一步调查才发现,那座庙建于明朝的某个时期,至今已有三百多年,当时,那个“我”只存在于可能语气之中。 

学习中文的人,最初遇到的一个困难是,找一个满意的词语表示自己的身份,以区别于他人。中国人的整个思维都与我们所习惯的不同,他们可能并不完全理解西方人为什么会有把一切都弄得准确无误的癖好。中国人并不确切知道他的村子里有多少人家,他也确实不想知道。他始终不能明白那些想知道这个数字的人到底为了什么。只有“几百”、“好几百”,或者“没多少”,而没有准确的数字,过去没有,将来也不可能有。 

中国人缺乏精确性不仅表现在数字的运用上,而且也反映在文字书写和印刷上。在中国,要弄到一本没有错别字的书并非易事。有时,所用的错别字比正确的字还要复杂,说明写错别字不是为了贪图简便,而是由于人们日常不重视精确性。文字书写的不准确更大量的是表现在常用字中,有些字经常用同音字代替,出现这种错误,或者是由于不认识这些字,或者是因为马马虎虎。 

漠视精视在书信的称呼上更是表现得一清二楚。中国人家信的称呼是用醒目的字体写的,“父亲大人”,“慈母大人”,“叔祖大人”,“贤弟大人”,等等 ,一般不写出“大人”的姓名。中国人非常讲究实际,但正如我们所看到的那样,他们对自己的名字却满不在乎。像这样的民族实在是独一无二的。我们常常发现,他们的名字一会儿写成这样,一会儿又写成那样;我们看到名字,但并不知是谁,还要问一下。最使人弄不清的是,同一人常常有好几个不同的名字,他的原名,他的“号”,甚至还有科举考试注册时专用的名字。正因为如此,外国人常常把一个中国人误认为其他人,村子的名字更不确定,有时会有两,三个完全不同的名字,并不是一个比一个更“恰当”。如果一个名字有了别名,它们可以互相交替使用,在官方文件上用原名,在平时交谈中可用别名;甚至也可以把别名当做形容词,与原名一起组成一个复合名。 

中国人缺乏类似于化学分子式那样绝对需要精确的教育,这是令人遗憾的。中国的第一代化学家也许会因为把“没几十个微粒”的某物与有“好几十个微粒”的另一物混在一起,而少了许多数字,造成预想不到的严重后果。中国人完全能够像其他民族一样学会对一切事物都非常精确-一甚至更加精确,因为他们有无限的耐心--但我们必须指出的是,他们目前还不重视精确,他们还不知道精确是什么。如果这一看法是正确的,那么就可以有两条推论:其一,在我们考查中国历史档案时,必须考虑到中国人漠视精确这一特性。我们采用中国人所提供的数字和数量很容易使我们自己受骗,因为他们从来就不想精确。其二,对于中国人所提供的冠以“统计数字”以抬高其权威性的各种材料,必须留有很大的余地。整体并不大于部分之和,然而,中国人的统计数字却相反。当我们审查完中国人的一份“统计数字”后,就立刻会像一位聪明的苏格兰人拿着一部很现实的“不确定大法”对美国最高法院说:这里有“对案件的最终的猜测!”


\chapter{易于误解}
当外国人学了不少汉语并足以表达自己的思想时,你首先发现的是中国人很有天赋。令你惊讶并感到痛苦的是,你所说的,别人听不懂。于是,你以更加的勤奋重新学习;几年后,你能够自信地与别人交谈各种复杂问题。但是,如果是与一个完全陌生的人交谈,尤其是与从未见过外国人的人交谈,你就会像最初说汉语时那样感到惊讶和痛苦。对方明显是听不懂,而且明显是不想听懂。他根本就没注意你在说什么,也不跟着谈话的思路,而只会打断你的话说:“你说的,我们听不懂。”他带着一种具有优越感的微笑,就像期待哑巴开口说话一样,好像是在说:“谁说能听懂你的话?你天生就不会讲中国话,这是你的不幸,但不是你的过错。可是,你应当承认你的无能。不要为难我们,因为你说的,我们听不懂。”在这种情况下,始终保持一种平静是不可能的,你自然要发火,说:“我此刻所说的,你懂吗?”“不”,他回答说,“我没听懂你所说的。” 

中国人听不懂外国人所讲的汉语,还有另一种情况,这就是即便他们听清楚你所说的语词,但由于没有注意到某些细节,所表达的意思还是没被搞懂,至少没有全部搞懂。比如,“外国人在中国”这个短语必须放在表达“在这种条件下”、“有条件地”,“根据这种条件”之类的一大堆短语之中。而中国人事实上并不用这类短语,觉得没有这个必要,这与外国人大不相同。中国人也不用时态,不在乎时态,而外国人就一定要注意时态。 

在中国,所要注意的问题中,最需要防止发生的问题是不要在钱上造成误会。当外国人要为所购买的商品付钱时(在中国人看来,这是外国人的主要职责),将来完成时态就像“军事急需品”。“你将来干完活之后,你就会得到钱。”但是,汉语中没有将来完成时态,或者说没有任何描述某事在什么时候发生的时态。中国人只是简单他说:“干活,挣钱。”在他的心目中,后句是主要的,并且不受“时间关系”的限制。因此,他给外国人干活时,希望马上能拿到工钱,这样才能有饭“吃”,似乎如果不是偶尔遇到这个工作,他就会连一点吃的都没有。我们必须反复告诫的是,在中国做生意尤其要避免在钱上造成误会。谁收钱,什么时候收,收多少,是银锭还是铜钱;若是银锭,成色怎样,重量多少;若是铜钱,“一串”有多少个--诸如此类的细节,在通常情况下是不可能说得太明白的。若是与营造商、经销商和船东签订合同,对方该做哪些事,要履行哪些条款,一开始就必须作大量准确的说明,否则就会一团糟。 

“自作自受”,这在中国是太常见了,并不会引起多大的注意。一位船夫或车夫受雇于外国人,本该按照雇主的要求,但有时却断然拒绝履行合同。在这种场合,中国的车夫固执得就像他的一匹骡子。那匹骡子躺在泥泞之中随心所欲地洗泥土澡;车夫用鞭子抽打着骡子,直至精疲力尽,仍无济于事;而骡子却视鞭子抽打如苍蝇挠痒痒。看到这一情景,不禁使我们想起德昆西”对中国人的评论,他讽刺中国人“像骡子一样固执”。他的说法显然有些过头;其实,中国人并不像骡子那样固执,因为骡子不会改变它的脾气,而任性的车夫则不然。受雇的车夫虽然在半路上不听从他的雇主,甚至对于雇主明确警告要扣掉他的全部“酒钱”也不予理睬,但过了半天时间到达终点时,他却对自己在半路上的所作所为予以辩解,并且赔不是。旅行者与他的车夫,船夫立下一个字据,通常是明智之举,这样就不会因可能的误会而带来麻烦。 

“有言在先”,这是中国人谨慎处世的至理名言。然而,事情往往是,即使费尽心思订立了字据,也还是会有出现误解的时候。在中国的外国人碰到这种情况时,无论你认为其中的原因是什么,都会发现钱是引起你烦恼的重要原因;至于对方是受过教育的学者还是一字不识的苦力,这几乎没有多大差别,所有中国人都有在误解中取得优势的天赋。他们就像腊月的北风钻进门缝,像河水流进船洞,迅速且不费劲。盎格鲁一撒克逊民族为了适应需要在某些方面也独立地发展出这一中国人的天赋。就像古波斯人具有拉长弓和讲真话两种重要的技能一样,盎格鲁-撒克逊人具有对敌人和对朋友都同样诚实和公正的天性,对此,中国人不久就会有所觉察。对中国人来说,这些品质似乎就像犹太人曾有过的某种独特的习惯。犹太人在对台塔斯时期的罗马人采取军事行动时,无论形势多么紧迫,每隔6天都必定要暂停一次。就像犹太人的那种习惯对罗马人是有利的一样,盎格鲁-撒克逊人的那种天性对中国人也是有利的。 

1860年之前西方与中国人所进行的一个世纪的外交,充分表现了中国人易于误解的习惯;在以后的年代里,这种习惯并没有消失。与中国的外交史在很大程度上是一部力图对被完全误解的事进行解释的历史。无论如何,中国人越来越清楚地看到,外国人是遵守诺言的,尽管也有例外。而且同样非常相信外国人办事公正(尽管也有某些个人和国家恰恰相反)。但是,正是基于这两点,中国人有能力对付即使是最顽固的外国人,“你是这样说的”,“不,我不是这样说的”,“但我认为你是这样说的,我们都是这样理解的,就算我们都是傻瓜,请付钱,是你自己说过要给的”。这就是中国人与外国人千百次争论的实质,而百分之九十七的结果是外国人付了钱,中国人心里很清楚,外国人为了表现自己的诚实和公正是会给钱的。在以下的3个事例中可以看到,中国人如何利用其他手段达到目的,而且是三次有两次成功。” 

有过体验的读者会发现每天都有大量各种误解的例子。你告诉一个苦力把院子里的杂草拔掉,而把开始抽芽的草皮留着,从而使你能够看到渴望已久的一片珍贵的草地;可是,这只粗心的牛拿着锄头把所有绿草都铲除了,使之成了荒地,还说这样才干净。他不“理解”你的意思。叫厨子到很远的菜市场去买一条鲤鱼和一只鸡;他没买鱼回来,却买了三只大鹅;他认为你就是这么吩咐的。他不“理解”你的意思。派送信人在收信之前把一包重要的信件送到法国领事馆;他回来说,法国领事馆不收该信件;原来他把信件送到了比利时领事馆,结果误了收信时间。他不“理解”你的意思。 

笔者的一位朋友的亲身经历可以很好地说明,可怜的外国人是多么容易误解和被误解。这位朋友去拜访一家中国的银行,银行老板与他关系良好;当说到最近银行附近发生的一场灾难性的大火时,这位外国人为大火没有蔓延到银行而向老板表示庆贺。对此,这位老板立刻觉得尴尬,并生气地说:“这是什么话?这么说不好!”直到后来,我的朋友才发现,他的冒犯之处在于他的话可以被误解为:大火如果再近一点把银行烧了,那才是最不幸的。因此,尽管是表示庆贺,但也属犯忌。一位刚到京城不久的外国人看到一群骆驼,其中有只小骆驼,就对长期受雇于外国人的车夫说:“你回家时,告诉我的小孩,让他出来看这只小骆驼,他从来没看过,这一定会令他非常高兴。”车夫没有立即回答,像是在酝酿某个思想一样,过了一会儿才略有所思地说:“就算你要买那只骆驼,也不能抬举它,否则就是害它!” 

笔者曾参加过一次有中国人参加的礼拜,布道者讲的是有关纳曼的事。他描述了叙利亚大将军来到埃利沙门前并代表随从人员要求进见主人的情景。为了尽可能说得形象生动,布道者如演戏般地模仿叙利亚的仆人喊道:“门卫,开门;叙利亚将军来了!”令布道者吃惊的是,一位坐在后排的人此刻突然不见了,像是被枪击倒了一样;事后才明白,这人完全是误会了。他是教堂的门卫,由于没有注意到前面所讲的内容,当突然听到有人叫他开门,就迅速地冲了出去,让纳曼进来。 

还有个例子是:某省一位传教士的听众所产生的错觉。这位传教士为了让他的听众有深刻的印象,就用幻灯机显示一只放大的普通寄生虫。这只虫子斜着映现在屏幕《圣经. 旧约》人物,跟从先知以利亚的圣者澎上,其庞大的身躯完全就像是埃及的鳄鱼。这时,只听见一位观众以一种敬畏的口气小声地宣布他的新发现:“看,外国大虱子!” 

从西方人的观点看,中国人误解西方人,而且这种误解经常是导致冲突的原因。然而,史密斯缺乏公正之处在于,从中国人的观点看,西方人也在误解中国人,这种误解经常也是他们强权与侵略的理由。两种文化交流,如何避免各执一端的片面,在今天依旧是一个重要问题。



\chapter{拐弯抹角}
我们盎格鲁-撤克逊人引以自豪的习惯是喜欢直来直去,心里怎么想,嘴上就怎么说。当然,考虑到社交礼节和外交的需要,我们在很大程度上不能完全按照这一习惯行事,然而,直来直去的天性实质上仍支配着我们,只是在不同场合表现不同。然而,经过与亚洲各民族不太长的接触之后,我们相信,他们的天性与我们根本不同——事实上是分别处于相反的两端。在这里,我们且不说亚洲各国语言中敬语的累赘,在这一方面,某些国家的语言明显要比汉语更加复杂麻烦,否则,我们的朋友会一直沉默寡言下去,保持一种莫名其妙的沉默。 
  中国人下决心把不好的消息传给他人时的举止非常有趣。在那种情况下,有时事情已不是什么秘密了,甚至可以公开直截了当他说了,但传消息的人还是完全有可能采取一种拐弯抹角、不着边际的方式说一件不能说,万万不能说的事。只见他心神不安地看看四周有没有人偷听,然后压低声音神秘地窃窃耳语;他伸出三个手指头,作为手势,不明不白地暗示那个没说出来的人就是他家的老三。他先含含糊糊他说了一番,然后指出事情的重要性;正当说到来劲的时候,他突然停住,不进一步说出事情发生的原因,然后意味深长地点点头,很可能是说:“现在,你可明白了,不是吗?”在这个全过程中,可怜且不开窍的外国人除了不明白还是不明白。传消息给你的人说到这种程度,如果你还是一无所知,那也并不奇怪,他会明确他说,总有一天你会发觉他是对的! 
  中国人与其他民族都有一个共同的特性,这就是都希望尽可能长时间地隐瞒坏消息,并以一种间接的方式表达出来。但是,中国人所要求的“好方式”其隐瞒程度之大,肯定会让我们惊讶并感到毫无必要。我们曾听说,有一位慈祥的老奶奶意外地遇到两位朋友,这两位朋友是特地赶来向老奶奶报告她那在外的孙子不幸去世的消息,当时他们正在小声地商量应当怎么转告。可是,当遇到老奶奶后,他们却只是反复说明他们正在闲聊,尽管后来不到半小时,消息就已经传开了。我们还听说,一位离家数月的儿子在回家的路上,他的朋友劝他快快回家,不要逗留去看戏,他便从中推断出他的母亲去世了!事实果真是如此。我们曾受托将某个中国人的一封信转交给离家很远的人:信的大意是:他不在家时,他的妻子不幸突然去世,邻居见他家没人管,就拿走了他家的每一样东西,而这些东西理应属于他的。但是,这封信的信封上却用大字写着不太准确的话:“平安家信!” 
  中国人善绕弯子还往往表现在该用数字的地方不用数字小比如,一部五卷本的书,每一卷不是用数字分别标上第几卷,而是标上“仁”、“义”,“礼”,“智”,“信”,因为这是“五德”的恒定顺序。硼多册的《康熙字典》,各册不是像我们所期望的那样用数字来区分,而是分别用跃干地支”来标识。在考场上,每个学生的隔间是分别按照《永乐大典》的字目次序进行标识的。 
  另一个事例是向已婚妇女了解她家成员和其他人的情况时她所表现的拐弯抹角。这种妇女不是用原名称呼,而是仅用丈夫的姓和娘家的姓合成的两字姓氏来称呼。平时被叫着“某某他妈”。比如,一位你熟悉的中国人对你说,“黑蛋他妈”病了,也许你从来没听说过他家有一个“黑蛋”,但他认为你肯定知道,但是,如果没孩子,那问题就更复杂了。也许这位妇女被称为“小黑蛋他婶”,或者其他拐来拐去的称呼。已婚多年的妇女很自然地称自己的丈夫是“在外的”,意思是,丈夫是忙家务事之外的事。结婚不久还没孩子的妇女在说到自己丈夫时常常由于没有合适的词而为难;有时称她的丈夫是,"先生”;有一次,她被逼得没办法,干脆用丈夫干活的地方来称呼他——“油坊是这么说的!” 
  一位著名的中国将军,在去战场的路上,经过一片沼泽地时,向那里的青蛙深深地鞠躬,他希望他的士兵们明白,要像这些青蛙那样的英勇才是值得赞美的。普通的两方人当然知道这位将军是在给他的部队以某种“巨大的动力”,但对于那些生活在中国的外国人来说,这也许算不得什么。中国的春节来临之际是一年一度借债的时候,一位熟人前来见笔者,他做着某种手势,似乎包含着什么深奥的意思,他用手指指了指天,又指了指地,然后指了指对方,最后指了指自己,一句话也没说,我们惭愧他说自己不明白其中是什么意思。但他仍然不予原谅。他以为通过他的手势能够很容易地推知,他希望借些钱,而且希望保密,只有“人知”、“地知”、“你知”、“我知”!“吃、喝、嫖、赌,”是四种最常见的恶习,现在又加上抽鸦片。有时,人们张开五指,说,“他五毒俱全”,就是指某人沾染了所有这些恶习。 
  中国人善绕弯子还表现在,由于他们的礼仪规定过于复杂,可以采取一种在我们看来纯属拐弯抹角的方法去冒犯他人。比如叠信的方式就可以表现一种故意的冒犯。故意不把一个人的名字摆在其他字之上的单独一行,就是对他的一种侮辱,其严重程度要超过英语中不用大写字母拼写一个人的名字,在社交场合,不说一句话哪怕是一句不中听的话,都可以算是一种侮辱,就像不到合适的地点迎接来客人的身份送客一样。规矩如此之多,献少任何一个简单的动作都可能或多或少地在表示一种隐含的侮辱。中国人当然一看就能明白,而可怜且无知的外国人却因此受到无数次的伤害,还以为是受至。特殊的礼遇!中国人因生气而互相辱骂时,充分运用各自的文学才华,很得意地用一种精致的暗讽表示一种恶毒的意思,其暗讽水平之高,使人不能当场听出其中的真正含意,需要仔细琢磨,就像糖衣药丸,里面才会令人恶心。再比如,“东西”——字面上是指东西方向——意思是一样物品,而称某人是“东西”就是骂人。同样,拐弯抹角他说某人不是“南北”,意思就是,他是“东西”! 
  即使是最没知识的中国人也会随机应变地凭空虚构各种似乎合理的借口,我们每个人肯定都会被这种高超的能力所吸引。除了外国人之外,没有人会认真对待这种借口,除非是为了保住自己的“面子”,其实,过于想把问题弄清楚的外国人根本没必要花力气,一会儿在空气中,一会儿在水中,一会儿又在土中,去追究他们,因为他们习惯于把事实作最简单的处理。当他们被追得走投无路时,即使是最无知的中国人也会有一道牢不可破的防线,他干脆装着什么都不知道,以保证他可以脱逃。他“不知道”.他“不明白”,这两句话,像上帝之爱,掩盖了一大堆罪恶。 
  从每天发行的北京《邸报》上,可以找到很多用以说明我们的论题的材料。在中国,这张报子最能清楚地反映中国政府的真实情况,尽管也有欠缺。在报上,古语所谓的“指鹿为马”己变得更加高级,运用也更加广泛;关于“事情并不像看上去那么简单”,报上有其他地方所不可比拟的真实写照。如果中国人真的是不愿意讲出事情的缘由,只能靠猜测去知晓他所说的意思,那么最好的事例可以从中国官员的生活中去找,在那里,拘泥于形式和矫揉造作已发展到极点。当中国的“官方报子”整版都登载着一些渴望退出官位的年老官员遭受各种痛苦的情况时,这里究竟有哪些含意?当他的迫切请求被拒绝,并被要求立即回到他的岗位上去时,这又意味着什么?这长篇的编年史作为事实材料披露出来,其真正的含意是什么?当一位被指控为有罪的高官被确定为无罪,而被认为只是犯了一些还够不上惩罚的过失时,这是意味着起诉人已没有足够的影响力,还是被指控的官员真得有做过那种事?谁能说得清? 
  我们完全相信,每一个细读过北京《邸报》同时读过报上每份文献的人,都能比较正确地了解其中的真正含意,这比读有关这个国家的所有作品都更能了解中国。但是,迄今为止,所有外界的野蛮人在理解中国时,都是采取一种根据其言了解其意的方法,以为这就是真正的中国人,其实我们会遗漏很多方面,难道没有理由对此表示担心吗? 



\chapter{顺而不从}
我们最初对中国人的了解是来自我们的仆人。他们是我们了解中国人特性的第一任老师,当然,他们并无意识到这一点,我们对他们也总是不满意,然而,他们给我们上的课却令我们难以忘怀。随着我们与中国人的接触日益广泛,我们发现,尽管仆人是中国人中很小的一一部分,但我们与他们相处而逐渐形成的结论明显地得到更加广泛的确证,因为从某种意义上说,每个中国人都是整个民族的一个缩影。本章所要讨论的中国人特性,其标题虽然互相矛盾而不能令人满意,但却是最为合适,只要略加描述,就会很容易使人明白。 
  外国人居住在中国,在所雇的仆人中,没有人能像厨师那样完全左右着全家的安宁。刚到任的厨师,当女主人告诉他应当怎么做,不应当怎么做时,他就像是服从的化身。对家里已有的规矩,他是真诚地赞同,给人以好感,但并不是说他已成为赢家。女主人特意举例告诫他说,前任厨师有一个不能容忍的习惯,面包坯还没有完全发好,就放进烤箱。还有其他很多事都不能与女主人所想的一致,于是两人吵翻。对此,新厨师的反应令人愉快,表示他无论有什么缺点,也一定不能固执。女主人还告诉他,在厨房里,狗、二流子和抽烟都是不能容忍的。他回答说,他讨厌狗,也不抽烟,是一个外地人,城里只有几个朋友,都不是二流子。此后,他开始履行职责。但没过几天,发现这个厨帅在烤面包方面是前任厨师的“把兄弟”,也是把没发好的面包坯放进烤箱;而且有数不清的人在厨房里进进出出,许多人还带着狗;厨房里香烟味浓重,成了永久的宝物。厨师自己承认,面包做得不是最好,但肯定不是由于揉得不够,在揉面方面,他是很讲究的;厨房里看到的那些陌生人是他的“哥们”,但他们都不曾有狗,而且他们都走了,不再来了——但是,次日又见到他们;没有一个仆人会抽烟,烟味肯定是隔壁传过来的,那一家的仆人都是烟鬼。这个厨师是个懂道理的人,但是,由于没什么需要改变,他也就不知道如何去改变。 
  同样还有一件事。要一个苦力割草,给他一把雪亮、锋利的外国镰刀,他微笑地接过镰刀表示赞同。但后来在干活的那天,他用的却是一把中国镰刀,由大约4英寸的旧铁片加上一个短柄做成的。他似乎在说:“旧的更好。”给洗衣工一台外国的洗衣机,洗起衣服来既节省时间,又省肥皂,省力气,最重要的是洗得干净;再加上一台绞衣机,既不费力又不损坏衣服的纤维。但是,洗衣机和绞衣机都被丢在一边,成了“有用的废品”,洗衣工仍然像往常一样搓洗和拧衣服,衣物都被洗坏了。要改变这种情况,只能靠不断地督促他们。 
  告诉园丁用手头的砖坯修理一下破损的围墙,但他认为在墙头上插上树枝会更好一些,并且就这样做了;如果你问其原因,他会说出这样做的优越性。雇一个送信人把一包重要的邮件送到很远的地方去;傍晚,把邮包交给他,本来,次日清晨就可以出发。可是第二天下午,还看到他在附近的胡同里;派人把他叫回来,问他是怎么回事,他说他要休息一天洗袜子!雇一个车夫也会有这种体验。告诉他走那条路,照理其他人也会这么走,他也答应了,但他却带你走另一条路,因为他曾听过路人说,那条路不好走。厨师、苦力、园丁、车夫——统统都不相信我们的判断,而只相信他们自己。 
  外国人开的诊所和医院里也经常可以看到这类现象。医生仔细地给病人做了检查,开了药;病人拿到药后,医生反复叮嘱什么时间吃药,吃多少,千万别搞错;病人生怕忘了这些详细的规定,来回一两次,才算搞清楚;可是一到家,他就一口把两天的药都吃了下去,因为疗效的好坏肯定与药量的多少成正比。给病人贴一片膏药,并明确地告诉他不要去动它,但还是不能阻止他随时揭掉膏药, 因为病人不希望变成一只“乌龟”,让一层硬壳长在皮肤上。 
  在一个诊所里,主治医生拥有各种医学头衔,且经验非常丰富,而助手却是一字不识,也不知药的名称和病的症状;但对于一般病人来说,助手的看法似乎与主治医生的看法同样重要。甚至看门人或苦力的一句话也足以使病人完全不顾医生的嘱咐,而采纳某种肯定是愚蠢且可能致命的做法,这些事听起来并不那么舒服,但却似乎是丰富的例证。 
  至此,我们所谈到的中国人顺而不从的例子都是外国人所遇到的,因为这些事最快引起我们的注意,并影响到我们的最实际的利益。但是,我们越是深入到反映中国人真实气质的人际关系,就越会看到“口是心非”的状况到处都是一样的。中国的仆人顺从且讨好中国的主人,与对待外国的主人是一样的,但他们不知道仆人是不能自行其是的,他们的主人也可能不会要求仆人唯命是从。外国雇主要求雇员切实地按照规定做,因为如果他们不这样做,雇主就一直会给他们难看。笔者的一位朋友有一大帮仆人,他们既极端忠诚又极端固执- 一这使他们成为既难得又讨厌的人- 一这位朋友每每谈起这些怪“家伙”,就表现出这类仆人的主人所常有的两难;他经常拿不定主意,不知是炒他们的鱿鱼呢,还是给他们加薪!中国的雇主完全清楚,他的指令会被撂在一边,但他事先会有所准备,就像留一些备用金以备坏账,或者像机械学所说的,留些空隙以减少摩擦。 
  中国的各级官员在他们的相互关系以及与最高层的关系中,也或多或少地有类似的无视命令的现象。导致违反上级命令的原因有以下几种:比如,个人的懒惰,为了朋友,最重要的是金钱的魔力。一位地方官,由于其居住地的水有咸味,就命令他的仆人用水车到几里以外的河里去运水。仆人并不是按部就班,而只是到附近有甜水的村子去取水;取来的水与官员要求的一样多;而且少走了三分之二的路程,皆大欢喜。如果这位官员确切知道他的仆人没有依照命令去做,那么只要有好水喝,他也许就不会过问这件事了。在中国,“会捉老鼠的猫,就是好猫”。一事成功,事事成功,中国人天生怕碍罪人,生怕出乱子,因此,出现了违反命令的不端行为,就算有500人知道内情,也不会有人去报告。有一个典型的中国仆人,要求他把水池里的水用容器装起来,以备后用,他却把水统统倒进了井里。这样,他表面上一副顺从的样子,而实际上却是完全相反,雷尼博士说到一件事:厦门某官员把国家的公文分成两部分,把后半部分放到前面,为的是让别人难以读懂。在与外国人打交道的事务中,这种手段是很常见的,中国的大臣并不想让外国人满意。 
  我们经常可以看到在执法过程中也有违法行为,而与法的要求相冲突。地方官判处一名罪犯戴两个月的木枷,只有晚上才能卸下来。但是,只要在“最关键的地方”花上几个钱,那么命令也就可以打折扣了,犯人只需在地方官进出衙门时带上木枷,装装样子;而其他时间,犯人尽可以把令人讨厌的木枷丢在一边。那么,地方官难道不知道贿赂会战胜他的判决,他应该悄悄地回去当场抓住违背命令的证据?但他没有这样做。地方官自己也是中国人,他知道,判决书一下,它就不被当做一回事了,因此,他会记得把服刑期延长一倍。这只是各部门官员之间错综复杂关系的一个实例,外国人会不断地看到类似的情况。上司命令下司去检查某一步骤的执行情况,下司毕恭毕敬地报告说,这件事已经做了;事实上,这期间根本什么都没做。在许多情况下,事情就到此结束了,但是,如果不断有来自某个方面的压力,而且命令非常急迫,下司就会把这种压力转嫁给更下一级的官员,并把上司的指责也转嫁到他们头上,直到这种压力的“风头”消去为止,然后,一切又照常如初。这就是所谓的“改过自新”。这种“改过自新”在很大程度上类似于禁止鸦片销售和种植,忽冷忽热,其结果也是可想而知的。 
  肯定会有人认为中国人是最“固执”的,我们用“顺”这个形容词去描述中国人“不从”的特性,似乎显得异常的不恰当。然而我们必须重申并确信,中国人远不是最固执的民族,事实上,他们远不如盎格鲁一撒克逊人来得固执。我们说他们“顺”,是因为在他们像骡子一样的“倔强”中含有一种依顺的特质,而这往往是盎格鲁-撒克逊人所缺乏的。 
  中国人能够不失风度地接受他人的指责,这正好说明中国人具有“顺”的天性。而盎格鲁-撒克逊人就没有这种本事,甚至连见也没见过。中国人能够耐心地、专心地、诚心听你指出他的缺点,并乐于接受,还说:“是我错,是我错。”也许,他会因为你善待他这样一个微不足道的人而感谢你,并保证,他会将你所指出的缺点立刻彻底地改正,并永不再犯。你完全知道,这些漂亮的承诺不过是“镜中花、水中月”;但是,就算是不着边际,也有可能使事情就此结束;而且,你如果对此加以注意,就会发现,你要求他们做的也不过就是这些。 
  把中国人比做竹子,这种比拟一矢中的,最为精确。竹子高雅,到处都用得着,它柔顺,中间是空的。东风吹来,它朝西弯,西风吹来,它朝东弯;没风的时候,它一点也不弯。竹子的幼苗是棵草。然而,草易于打结,而幼竹尽管柔顺,但很难打结。世界上没有什么比人的头发更柔顺的了。它可以拉到一定的长度,但是,拉力一旦没了,它就立刻缩回去。头发只是按照自己的重量倒向任何方向。许多人头上的头发长成怎么样,就是怎么样,一般是不能改变的。有一种头发俗称“牛舐过的”,也就是,一绺翘着不易梳理的头发,而其他头发,不管有多少,都必须顺着这一方向梳理。如果把我们居住的星球看成是一个头,各个民族看做是头发,那么,中华民族就是一绺古老的牛舐过的头发,它可以梳,可以剪,可以剃,但依然与以前一样,生长的大方向是不可改变的。

\chapter{思绪含混}
我们把“思绪含混”说成是中国人的一种特性,并不是指只有中国人才有这种情况,或者所有中国人都是这样。作为整个中华民族,他们完全有能力自主于世界民族之林,他们的智力当然并不低下,而且没有任何衰退的迹象。同时又必须记住,在中国,教育并不普及,那些没有受过完整教育或根本就没受过教育的人,他们在运用中国语言时,造成了思绪含混,有可能犯了律师所说的“事前从犯”的罪行。 
  不少人已经知道,汉语的名词是没有格的变化的,它们既没有“性”,也没有“格”。汉语的形容同没有比较级。汉语的动词也不受任何“语态”、“语气”、“时态”、“单复数”和“人称”的限制。名词、形容词和动词之间没有明显的区别,任何汉字只要能用的,都可以通用,不会有什么问题。我门并不是要说中国语言不能用来交流人的思想,也不是要说中国语言很难或不能把人类的各种思想都表达清楚(尽管这样说有道理),而只是认为,这种语言结构,正像夏天的酷热自然要引起午睡一样,会招致“思绪含混”。 
  与一个没受过教育的中国人交谈,要弄清他所说的是什么意思,那是很困难的。有时,他的话好像全都是宾词;这些词以一种复杂方式编排在一起,莫名其妙,不着边际。说话含心里以为,没有主格无关紧要,他自己是清楚在说什么,但决不会想到听众根本无法根据直觉理解他所说的这一大堆内容。显然,很有经验的专业猜测家可以给人多数中国人讲的话补上所缺少的主语或谓语,并指出其中有歧义,而不能表达真正的内容。有些往往是整句话中最重要的词,但却破产掉了,也找不到有任何头绪。在说话中,话题的主语发生改变时,说话人往往没有在态度、音调以及相光的细节上予以提示,因此,你会突然发现他已经不是像刚才那样在说他自己,而是在说道光年间他的祖父。他怎么会说到那里,又怎么再说回来,往往是一个难解的谜,但我们每天都可以看到这一成功的绝技。对中国人来说,没有预先的提示,而突然莫名其妙地从一个主题、一个人、一个世纪跳到另一个主题、另一个人、另一个世纪,是再平常不过的事了,就像一个人在看窗上的小虫的同时,不转移视线就能够看到同一视线上远处山坡上的牛群一样。 
  汉语动词没有时态,中国人讲话没有说明时间,地点变化的标记,这些都是事实;但同时,他们的思绪经常处于含混状态。在这种情况下,可怜的外国人若对一连串稍纵即逝的想法仍然抱有兴趣的话,最好的方法就是开始提出一系列问答式的询问,就像一位边远地区的猎人用斧子在无路的森林中“开出”路来,“你现在说的这个人是谁?”这个问题搞清楚后,还可以接着问,“你说的这是什么地方?”“什么时候?”“这个人做的是什么?”“他们为什么要这样?”“后来呢?”每问一个问题,你的中国朋友都会带着一种困惑或许是一种恳求的表情望着你,似乎在怀疑你可能没有五官。但是,沿着这样的线索不断地追问下去,就会找到阿利蒂纳*把人们从无望的迷宫中解救出来的路线。 
  对没有受过教育的中国人来说,无论什么想法都会令他惊讶,因为他肯定没有心理准备。他搞不懂,因为他也不想搞懂。他需要花一些时间增加思维能力,以便进入新的状态。他的头脑像一门生了锈的旧滑膛炮,架在腐朽的炮架上;在瞄准前,需要先调整方向,而最后肯定还是打不响。因此,当问他一个简单问题,比如“你几岁?”他*古希腊神话中米诺斯的女儿。她用丝线引路将提修斯从迷宫中解救出来。一一译者注会茫然地盯着问话人,并反问道:“是问我吗?”你说:“是的,是问你”;这时,他振作起精神,又问:“是问几岁吗?”“是,是问几岁”;他再一次调整他的注意点,“是问我几岁吗?”“是”,你说,“是问你几岁”,“58”,他回答说,这回他的炮才运转正常,对准了目标。 
  思绪含混的一个突出事例是,中国人习惯于用事实本身来解释事实。你问一位中国厨师,“你为什么不在面包里放些盐?”得到的解释是,“我们都不在面包里放盐。”“你们城里有这么多好吃的冷冻食品,为什么不留一点冬天吃呢?”“是的,我们不留冷冻食品到冬天吃。”一位拉丁诗人说过:“能够知道事物缘由的人是快乐的”;如果他生活在中国,他就会把他的格言修改为:“试图找到事物缘由的人是自寻烦恼。” 
  思绪含混的另一个事例是,他们无法将别人的一个想法原原本本地转告给另一个人。要A把某事转告于B, 再转告于c, 这在中国是最难做好的,或者是由于有关的人不了解该事的重要性,而根本就没有把该信息传下去,或者传到C时已是面目全非,不知所云了。以为这样一台复杂机器中的3个齿轮能互相配合,运转正常,那简直是异想大开。即使是那些有相当理解力的人,他们也觉得转达一个想法而不有所增减是很困难的,正如一根直棍插入清水中,折射出来的肯定是扭曲的样子。 
  善于观察的外国人处处会碰到这些奇特的现象。你就某种反常行为问道:“他为什么这样做?”“是的”,回答就是这么简明扼要。在这种含混不清的回答中,一般附有几个令人恼火的常用词。既有表示疑问的“多少”,又有表示肯定的“几个”。你问:“你在这里住几天了?”回答是:“是的,我在这里已经住了几天了。”在中国人的言语中,也许最含混不清的词是人称(或非人称)代词——“ta”,这个字既可以表示“他”、“她”,也可以表示“它”。有时,说话人为了表明他所说的人指的是谁,就用拇指含混地朝这个人的家的方向指一指,或者指向这个人目前所处的地点。但是,单音节的“ta”更经常被看做是一个关系代词、一个指示代词和一个指定形容词。在这些情况下,中国人的谈话就像英国法庭上证人作证,他以下列的语句表述一场斗殴:“他拿着一根棍子,他也拿着一根棍子,他打了他,他也打了他,如果他像他打他那样狠地打他,他就会打死他,而不是他打死他。” 
  你向一个散漫的仆人提出质问:“叫你,为什么不来?”他回答说:“没为什么。”坦率得不得了。这种思维含混的状态会导致各种往往令人为难的举动,使得讲究条理的外国人总要为此而生气。厨师做饭时,总要把佐料用得够,而做下顿饭时,就少用一些。问他是怎么搞得?他坦率地回答,“佐料用完了。”“那你为什么不及时再弄一些呢?”“我没再弄一些。”这就是他的最好解释。你向某人付一笔钱结账,于是就很花功夫地打开保险箱,非常细心地点钱给他;付完钱后,他坐下来聊了“老半天”,乱七八糟的什么都聊;然后,若无其事地说:“除了这笔账,我还有一笔账在你这里。”“那刚才我开保险箱时,你为什么不告诉我?不然我可以一次统统付清。”“噢,我是想那笔账与这笔账之间没有任何关系!”再比如,一位病人在诊所看病,已经随意地花掉了医生的大量时间,但不一会儿,他又回到候诊室;医生告诉他,他的病已经看过了,他则爽快而简要地说:“除了刚才看的病,我还有其他病!” 
  在我们看来,最愚蠢的是普通中国人习惯于生了病而不及时治疗,也许是因为当时太忙,或者是因为治病要花钱。他们往往认为,忍受一阵阵的打摆子要比花10个铜钱——约1美分——买一剂能治好病的奎宁更便宜。我们看到许多病例,都只是因为病人拖时间而使本来不需要花什么钱就可以治好的病发展到不可救药的地步。 
  一个人的家离外国人的医院不足半里路;他外出时染上了眼病,回家后痛苦地拖了两个多星期才去治疗;在这期间,他每天都希望病会自动好起来,但与此相反,他的一只眼因角膜溃疡而瞎了。 
  还有一位病人,他因脖子深度溃疡而每天都要接受治疗;当治疗到18天时,他说他的腿疼得让他睡不着觉。经检查才发现,他的腿上有一处像茶杯那样大、那样深的溃疡!他是想在他的脖子治好后才说他的腿! 
  中国人日常生活的这类现象会使我们想起查尔斯·里德的一本小说中的一段话:“人类不是没有头脑,而是他们的头脑有毛病——头脑含混!” 
  中国的教育根本无法使受教育者在能够理解和能够运用的意义上掌握一门学科。西方各国都有人在传说,某些布道者可以确切地证实,即使他们的经书上有天花病毒,他们也不会染上。而在中国人当中居然可以看到这类事。中国的狗一般不会自己去追捕狼,当看到一只狗在狼的后面,若不是朝相反的方向跑,那么至少是朝相互成直角的方向跑。与此相类似,中国人在谈论某一话题时,总是离话题越来越远。他往往触到该话题,时而像是要作彻底的探讨,但最后还是离它而去;说得非常疲倦了,还是没有把话说完。 
  中国是一个两极分化的国家。富贵者与贫穷者、受高度教育者与愚昧无知者,都生活在一起。成千上万命该贫困而又无知的人,他们的眼界狭窄,自然头脑含混。他们就像井底之蛙,看到的天空只是黑暗中的一块。有不少这样的人连10里以外的地方都没去过;他们并没有想过要过上比周围的人更好的生活。在他们身上,任何人都具有的天生的好奇心似乎已经泯灭。即使当他们知道,离他们家不到1里的地方住进了一个外国人,他们也从来不打听一下他从何处来,他是谁,他要干什么。他们只知道如何为生存而斗争,此外,就一无所知了。他们不知道人是否像有些人所说的那样,有3个灵魂,还是只有1个,还是1个都没有;凡是与粮食价格无关的事,他们无论如何也看不出其中会有什么重要意义。他们相信来世,相信来世中坏人会变成狗和虫;他们也纯朴地相信,身体最后会变成泥土,灵魂——如果有的话——会消失在空中。在西方,造就了“实际者”的力量,同样也造就了他们,他们的生命由两部分组成:肚子和钱袋。这种人是真正的实证主义者,因为你无法让他理解他没见过或听过,当然也没有任何概念的事物。生活对于他来说只是一连串事实,而且绝大多数是不称心的事实;至于涉及事实以外的任何东西,他立刻就成了一个无神论者,一个多神论者和不可知论者。给他一些意想不到的尊敬和食物,就足以满足他的依赖本性,但是,这一切在很大程度上取决于他周围人的习惯。在他看来,人的肉体只是独自地生长发育,而与心理的和精神的因素无关。要把这些人从麻木的状态中解救出来,唯一的办法就是输入一种新生活,向他们展示古代基督教创始人所讲述的终极真理:“人是有精神的”,因为这就是所谓“上帝的感召赋于他们知性”的全部含义。 

\chapter{不紧不慢}
从“nervous”这个词的不同用法中,可以看出现代文明的一个很有意义的方面。这个词的原意是“神经的,强有力的,刚强的,有活力的”。这个词的引申意思,也是我们今天经常碰到的,是“有神经衰弱或神经疾病的,神经过于紧张的,易激动的,有病的”。表述神经疾病处于不同阶段的各种专业术语,今天听起来像日常用语那样熟悉。现代的文明无疑使人们的神经过于紧张,神经疾病也比前一个世纪更为常见。 
  但我们现在要说的并不是那些患有神经疾病的人,而是一般的西方人。这些人并非有疾病而健康状况不佳,相反,他们经常以各种方式提醒自己,神经系统是全身最重要的部分,因此,我们说的是那些“神经过于紧张”的人,我们知道这也包括所有我们的读者。对于盎格鲁-撒克逊人来说,那些生活在蒸汽机和电力时代的人们,其神经的紧张程度当然不同于生活在帆船和马车时代的人们。我们的时代是日新月异的时代。它是一个急匆匆的时代。连吃饭的空闲都没有,神经一直处于高度紧张状态,其后果完全可以想像得到。 
  今天的商人有一种急切、不安的神态(至少在西方国家做生意的人是这样),他们好像时刻在盼望一封关系其命运的电报——他们事实上也经常是这样。我们的这种精神状态无意识地表现在各种行为之中。我们坐立不安,心情烦躁。一边谈话一边拨弄着铅笔,好像此刻应该写些什么,否则就太晚了一样。我们搓着双手,好像准备干一桩需要耗费全部精力的大事。我们拨弄着大拇指,像野生动物那样迅速转过头去,似乎是担心有某些被忽略的危险事物。我们总有一种感觉,觉得我们现在应该去做某件事,这样,我们必须先尽快完成手头上其他几件更为紧迫的要事,然后立刻投身于那件事中去。神经的过度紧张不仅导致了诸如“拉琴痉挛”,“按键痉挛”、“书写痉挛”一类的病,而且导致了普遍的紧张。无论就时间长度还是就休息的有效性而言,我们的睡眠都大不如前。树上的鸟叫声。射进我们昏暗房间里的一丝光线、微风吹动百叶窗的响声、说话声,诸如此类,都会令人讨厌地打断我们的睡眠,而一旦这样,就别想再睡着了。我们把每天的生活都安排得没有空闲,其结果是我们没有得到真正的休息。在今天,有这样一种说法:银行家只能抱着银行睡觉才能成功。可见,在股东们获利之时,正是银行家倒霉之日。 
  在我们对西方人日常生活中所熟悉的事实作了一番描述之后,如果一个西方人去了解中国人,那么他肯定会看到或感到有某种强烈的反差。对死去的中国人进行解剖研究并非寻常之事,当然也曾做过,但是,我们从来没听说过“黑头发人”的神经组织与高加索白种人的有什么根本不同。中国人的神经组织与西方人的相比,正像几何学家所说。是“相似的”;但是,他们的神经紧张程度却显然与我们所熟悉的大相径庭。 
  对一个中国人来说,在一个位置上无论待多久似乎都没有什么特别的差异。他可以像一台自动机整天地写个不停。如果他是一个手艺人,他可以从早到晚地站在一个地方干活,编织、打金箔或干其他什么事,而且是天天如此,没有任何变化,显然也根本没想过需要有任何变化。同样,中国的学生也是长时间地被限制在某个地方,既没有休息也没有变动;若这在西方,肯定会逼得小学生们发疯。我们的孩子几乎一生下来就好动,相反,中国人的婴儿抱在怀里却像泥菩萨那样静静地躺着。稍长大一点,西方人的孩子会与猴子一起做各种滑稽动作,而中国人的孩子却往往是长时间一动不动地坐着,站着或蹲着。 
  在中国人看来,活动活动筋骨对于身体来说是多余的,他们不理解为什么外国人都爱外出散步。至于冒着生命危险,像“猎犬追野兔”游戏那样你追我跑地打垒球,更是难以理解了。广州的一位教师看到一名外国女子在打网球,就问仆人:“她这样跑来跑去要付给她多少钱?”如果告诉说:“没钱”,他根本不会相信。在中国人看来,一桩事完全有能力雇苦力去做,为什么还要自己去做?他对此根本不理解;若是有人说这样做有什么好处,他更是听不懂了。 
  就睡觉而言,中国人与西方人也有不同。一般说来,他不论什么地方都可以睡。搞得我们根本无法人睡的干扰,对他却不起作用。用砖当枕头,用草梗、泥土或藤做的床,躺在上面就可呼呼大睡,其他什么都不管。他睡觉时,不需要房里暗一些,也不需要别人安静。“半夜啼哭的婴儿”喜欢哭就哭吧,根本不会吵醒他。有些地区,在夏天午后的两小时里,所有的人都本能似地(像越冬的熊)躺下睡觉,很有规律,也不管是在什么地方。在这个季节的午后两小时里,整个世界就像半夜后两点一样寂静。不论是干活的人,还是其他什么人,睡觉的地方并不重要。横卧在三轮车上,脑袋像一只蜘蛛向下垂着,张大着嘴,苍蝇在嘴里飞进飞出;若以这样的睡觉本事为标准,经过考试招募一支军队,那么,在中国要招数以百万计--不,数以千万计--这样的人,是轻而易举的。 
  此外,我们肯定能看到的是,中国人对呼吸空气似乎不讲究,没有什么地方可以算是空气流通的,除非是一阵台风掀掉屋顶,或是一场饥荒迫使房屋的主人拆掉房子变卖木料。我们常常听说中国人住得过分拥挤,但是,中国人觉得这很正常,似乎不会有任何的不方便,即使有一点不方便,那也是不足挂齿。如果他们像盎格鲁一撤克逊人那样神经易于激动,那么,他们就会像我们通常所想像的那样是很不幸的。 
  中国人不会神经过于紧张还表现在他们很能忍受身体的疼痛。对中国医院的手术情况有所了解的人都知道,中国的病人常常是面对疼痛而不退缩,有些疼痛还可能令我们外国的壮汉所望而却步。这一话题可以很容易扩展为一篇论文。但我们必须把它搁在一边,而去听一听乔治·艾略特在一封信中所说的:“最高的感召与选择是不用麻醉药,眼睁睁地去忍受疼痛。”她说这句话肯定是由于她所不感兴趣的神学套话激怒了她。如果她是对的,那么毫无疑问大多数中国人使他们的感召与选择成为可信。 
  布朗宁夫人曾说过:“不抱同感去看,只会造成曲解。”无疑,这只是对像这位著名女诗人一类具有敏感大脑的人而言,西方人不喜欢被别人看,尤其是他正在做一件难做的事时更是这样。但是,中国人也许愿意在别人的观看下做好他们的工作。在外国人不常去的那些地方,我们的到来,会引出一大群中国人,他们用好奇的目光盯着我们,使我们一下子就产生了厌烦。其实,他们只是不带任何情感地看,并不是要伤害我们,但我们还是经常抱怨,若不把他们驱散,我们就会“发疯”。而对中国人来说,西方人这种本能的感觉完全不可理解。他并不在乎有多少人在看他,什么时候看,看多久;若是有人对别人的观看表示强烈的反感,那么他会自然地怀疑那个人肯定有毛病。 
  西方人不仅睡觉时需要安静,生病时更要安静。如果他平时从未有过这样的要求,那么他现在病了,可以要求不受噪声的干扰;朋友、护士、医生都会相互配合确保这一对治好病最为重要的条件。如果病人得的病已是无可救药,那么病人更是处于一种最安宁的环境之中。中国人的习惯与外国人的最大区别就在于如何对待病人。某人得病的消息一传开,来自四面八方的干扰都强加于病人身上;而且病情越重,干扰就越多。此时,谁也没想到需要安静;而且说来奇怪,也没人要求安静。那么多前来探视病人的客人需要热热闹闹地迎送、招待,有些人担心病人不久就会死去而痛哭不止,尤其是和尚、尼姑以及其他驱鬼的巫师大闹一场。对大多数西方人来说,看到这种场面,还不如死了更好。那位著名的法国夫人对前来探视者说:“她正在死去,请原谅不要打扰。”西方人对此没有不抱同感的。而在中国,决不会有这种原谅,即使有,也不会被接受。 
  在这个令人心烦意乱的世界里,无论什么地方的人都会感到担忧和焦虑。中国人不仅像其他民族一样受到这些邪恶的影响,而且要更深重得多。在许多地区,他们的社会生活条件使得有相当比例的人总是挣扎在死亡线上。只要雨水稍微减少,就会有成千上万的人挨饿;只要雨水稍微增加,洪水就会冲毁他们的家园。中国百姓很难幸免于官司的纠缠;一旦吃了官司,即使是完全清白,也难逃倾家荡产的厄运。这些灾难不仅说说而已,而且可以感到正在不断地俏然降临。对我们来说,最恐怖的莫过于等待一场不可防止并会带来可怕后果的灾难。中国人在面对这种灾难时,也许是因为它的不可避免而“眼睁睁地去忍受”,这正是这个民族最显著的特性之一。那些亲眼目睹灾荒年月成百上万百姓默默地死于饥饿的人,能够理解其中的含义。要全面了解中国人,就必须去看,但无论看到什么程度,西方人都难以真正理解,就像中国人很难真正理解盎格鲁-撒克逊人继承并发展了的个人自由和社会自由的理念。 
  无论我们从哪个方面去看中国人,我们都会发现,中国人是而且肯定一直是一个谜。我们将不断地去理解他们,直至我们终于相信,他们与我们相比是“缺乏紧张”有。这一含义复杂的说法会对这个民族未来与我们民族的关系产生怎样的影响——这一影响很可能随着岁月的推移而增强——我们不想冒昧地作出猜测,但我们至少相信适者生存这一普遍规律。在20世纪的生存斗争中,最适应者是“神经过于紧张的”欧洲人,还是不知疲倦、不急不躁的中国人呢? 

\chapter{轻视外族}
第一次到广州旅行的欧洲人很难想到,中国的这一商业中心与欧洲已有360年的频繁交往。在这期间,西方国家与中国人打交道,几乎没有什么能够令我们有理由感到骄傲的举动。外国人无论抱着何种目的来到中国,中国人通常对待他们的态度就像古希腊人对待其他非希腊人一样,而把他们看做是“野蛮人”来对待。即使在中国的官方文件中,也一直习惯于用“野蛮人”而不用“外国人”来指称外国人。只是到了1860年,由于某条约的特别条款规定,才开始不允许使用“野蛮人”这个词来指称外国人。 
   与中国人对待西方外来民族的这种态度有关,中国人的邻国长期以来就一直是一些最差的国家,因而他们被奉承惯了。尽管这些奉承无非是花言巧语、不怀好意。当他们发现,他们所接触到的外国人可以在哄骗和威胁之下,按照中国人的意愿行事,他们确信自己具有无法形容的优势地位,并且一直是按照这种看法处事,这种状况直到北京被占领才被迫发生改变。从那以后,尽管只过去了三十多年,但中国已经发生了很大的变化,可以说,中国人现在已经终于意识到了外国文明和外国人的全部价值。然而,不必对中国人作更广泛、更密切的了解,任何一个无偏见的观察者都会相信,目前中国人对外国人的态度,心里想的、官方的和非官方的并不都是尊敬,即使中国人实际上并不轻视我们,但他们好像是带着恩赐的态度对待我们,且往往是无心的。这就是我们目前所要面对的现象。 
  中国人对外国人最感到奇怪的是他们的服装,我们并不认为自己的服装有什么值得骄傲的。的确,东方人的所有各种服装在我们看来都是那么的臃肿,摆来摆去,限制了“人的自由”,但这是因为我们要求动作灵便,而完全不同于任何东方人。当我们考察东方人的服装式样是否适合于东方人时,我们不得不承认,这种服装完全适合于东方人。但是,东方人,特别是中国人,在看我们的服装时,找不到任何值得赞赏的地方,更多的是批评和嘲笑。东方人的服装要求宽松,穿在身上可以掩盖住身体的线条。有体面的中国人是不敢穿着短上衣到公共场所去的,而在中国的外国租界,经常可以看到许多外国人穿着紧身短上衣。外国人的短上衣,双排纽扣的礼服大衣,尤其是既难看又没样子的燕尾服,这些对中国人来说,都是莫名其妙;特别是有些上衣外套穿起来无法全部遮住胸脯,还露出一些内衣,更是不可理解,他们还看到外国人衣服的尾部钉着两粒纽扣,觉得那个地方没有什么可扣的,也不能起到装饰作用。 
  如果说外国人的男装在普通的中国人看来是荒唐可笑的,那么女装就更是这样,不管怎样说,它都有背于中国人的道德观念,更谈不上体面了。西方文明是伴随着男女之间的自由交往而产生的,只要我们看一下中国人对男女之间自由交往的限制,就会自然地感觉到,只依据传统标准的中国人完全有可能误解和曲解他们所看到的一切。 
  外国人听不懂中国话是中国人产生优越感的主要原因之一。比如,一个外同人,即使他能够流利地说现代欧洲所有各国的语言,但只要他听不懂一个不知字的中国苦力所说的话,那么这个苦力就会瞧不起他。的确,苦力若是这样,只能进一步表明他自己的无知,但他那毫无道理的优越感却是真的。,如果这位外国人硬要在这种环境中待下去,并努力地去掌握中国人的语言,他会不断地受到蔑视,就连自己的仆人也会在一旁说:“哦,他会听不懂!”其实听不懂的主要原因在于中国人自己说得不清楚。但中国人并不会承认这个事实,即使承认了,也不会削弱他的天生的优越感。这种情况,所有学习中文的外国人都经常而且永远会碰到,因为,无论他知道了多少,都总还有他未曾知道的新大陆。在一般情况下,外国人在中国有了一定的经历后,就不会因为他偶尔才知道某事而感到不好意思,更不会因为他对某事全然不知而感到不光彩。中国人在对外国人所表现出来的有关中国语言文学方面的知识进行估量时,往往就像约翰逊博士对女人唠唠叨叨的劝诫所作的生动描述一样;约翰逊博士说女人的劝诫就像狗用后腿走路——是无法做好的,但后来又惊奇地发现,居然做好了。 
  外国人对中国人的风俗一无所知是中国人产生优越感的另一个原因。他们几乎不相信居然会有人不知道他们早已知道的事。 
  外国人常常受到中国人间接的冷落而不知,这就导致中国人愈加故意地轻视外国人。对“当地人”把我们看成什么嗤之以鼻不予计较,反而会受到相应的惩罚。 
  许多中国人会有意无意地采取一种逗趣的方式对待外国人,时常还伴些贬意,就像利特默先生对待大卫·科波菲尔*,似乎心里还不停地嘀咕:“这么小,先生,这么小!”当然,在中国的外国人随着经验的不断积累,迟早会成为精明的观察者,而那时,情况就会有所不同。然而,一个人无论经验多么丰富,总还有他没听说过的或者第一次听到的事,总有许多细节是他所不知道的。 
  任何普通的中国人都会很容易做到的事,外国人却不会做,这就导致中国人看不起我们。我们吃不下他们所吃 
  *美国作家狄更斯小说《大卫·科波菲尔》中的人物。——译者注的东西,我们经不起太阳的暴晒,我们无法在嘈杂的人群中入睡,也不能没有新鲜空气。我们不会用他们的橹划船,也不会喊“吁!吁!”使唤牲口。众所周知,1860年,英国的炮兵部队在去北京的路上被无人驾驶的马车队搞得没有办法,因为英国军队中没有一个人能够叫中国的牲口走动一步! 
  无法适应中国人的观念和礼仪,以及其他更重要的规矩,使中国人毫不掩饰地轻视一个在他们看来没有“礼貌”或不懂“礼貌”的民族。其实,不是外国人不会鞠躬,而是他总觉得以中国人的方式鞠一个中国式的躬很难,不仅难在身体上,心理上也很难接受。外国人不把礼仪当一回事,常常表现出漫不经心的样子;即使他是一个很有耐心的人,但面对一场有礼貌的大战,其结局事先已经确定并为双方所知道的大战,就算只有20分钟,他也会不耐烦。外国人不愿意花“老半天”时间去闲聊。对他来说,时间就是金钱;但对中国人来说,根本就不是那么一回事,因为在中国,每个人都有很多时间,但却不是每个人都有钱。中国人不知道,他所浪费的时间是他自己的时间,而不是别人的时间。 
  外国人由于省却了大量令人厌烦的繁文缛节,而把时间用于其他方向,因此,与过分讲究礼节的中国人相比,外国人显得非常单薄,即使外国人自己也这么看。与中国官员富丽堂皇的长袍和温文尔雅的举止形成对照的是外国来访者那不成样子的跪拜;面对这种反差,即使有礼貌的中国人也难免会笑出声来。在这种场合,必须记住,对付中国人轻视外国人的最有效方法是,对东方人所看重的官架子不屑一顾。如果中国人见到的是“大美利坚皇帝”, 并且曾见过他穿着市民的服装,叨着雪茄,在大街上散步,那么中国人的心里又会怎么想呢?假如一个与中国的道台相当级别的外国领事为调解一桩国家间的纠纷,前往某省会与省长会晤,那么肯定有成千上万的人会聚集在城墙上,想亲眼目睹这位外国大官的浩荡队伍,结果看到的只是两辆马车,几匹马,一名翻译,一位中国的跟班,还有厨师等几人。东方人看到这种场面,自然会从好奇变为冷淡,再变成轻视,这并不奇怪。 
  我们在某些方面自认为肯定比中国人优越,但却不能如我们所想像和所期望的那样给他们留下印象,他们承认,我们在机械设计发明方面占优势,但许多发明却被轻易地看做是莫名其妙且没有实际用途的戏法,是某种超自然力的结果,是孔夫子闭而不谈的魔法。有一些承包商到中国后发现,中国人对蒸汽机和电力应用的奇迹是那么的不放在眼里,因而感到失望。此外,中国人反对一切都采用外国模式(尽管有时也不得不采用)。他们不关心卫生设备和通风设备,也不关心生理学。他们喜欢接受一些西方进步的成果,但不采纳西方人的方法,若要他们采纳西方人的方法,他们宁可把进步的成果也抛弃掉。只有那些肯定能直接使中国成为“强国”的东西,才能被接受,其余的必须暂缓一步;任何改良,如果没有时代精神,不会比中国优越,都可能被抛在一边。某些中国学者和政治家显然意识到中国的劣势,但又声称,西方国家所采用的知识只是古代中国人在高度发展数学和自然科学中所积累的,而近代中国人却不幸让这些东西被西方人盗走。 
  中国人显然对外国人在应用领域方面的能力不很感兴趣。撒克逊人欣赏“能人”,正像卡莱尔喜欢别人都称他“皇帝”。对中国人来说,外国人的技艺既令人感兴趣,又令人吃惊,而且以后若用得着,他们不会忘记和拒绝使用;但这绝不意味着在这些方面他们要效仿外国人。成千上万的中国人也许从来就没有这样想过。他们的理想学者是书呆子。这种人什么都学,什么都不会忘记,拥有多个学位,学习刻苦,废寝忘食,虽有几寸长的手指,却不会做事(除了教书);正是这样,才保持了灵魂与肉体的统一,成了不食人间烟火的超人。 
  西方国家并没有使中国人意识到自己落后于西方各国。中国驻大不列颠前任大使郭阁下的所言很能说明这一点。当听到里格博士说英国的道德状况比中国的要好时,阁下并没有立刻对此作出评价,而是富有感情色彩地说“我感到非常震惊”。这种比较从表面上看,尤其是从外交的观点看,是不成功的。它涉及到对这两个国家内部生活的深入了解和对现状的各种原因进行分析的能力。进行任何诸如此类的比较根本不是我们当前的目的。必须承认的是,中国的文人学士是外国人的主要对手。外国人虽拥有各种机械技术,但仍被中国的文人学士认为没有足够的能力欣赏中国伦理道德之伟大。这种蔑视在那些“头在宋代,脚在现代”的典型中国学者身上很能体现出来。就是这类人在近年撰写并散发了大量极力排外的文章,这些文章铺天盖地,充斥了整个中国。 
  曾有人认为,中国可能会被西方的各种发明所占据。刀叉、长筒袜、钢琴会从英国运到中国,在这种作用下,中国将会被“欧洲化”。如果说有一天中国会被这种方式所占据,那么这只能是很久以前的某一天,而决不可能有过这样的一天。中国不是一个可以任人宰割的国家,中国人也不是一个可以任人宰割的民族。要使中国人对西方人保持稳固而持久的尊敬,唯一的途径是通过可信的客观事实表明基督教文明无论在总体上还是在细节上都取得了中国已有的文明所不能相提并论的结果。如果没有这些可信的事实,中国人仍会在与外国人的接触中表现出恩赐和轻视的态度,这不是没有理由的。

\chapter{缺乏公心}
中国最古老的经典之一《诗经》上有一句也许是农夫所说的祷告辞,大意是:降雨先至公田,尔后再及私田。无论周朝兴盛时期以及后来的各朝代是否真有其事,现在是肯定不会有农夫或其他什么人祈求老天“先”降雨到“公田”了。中国实质上是采取家长制的统治方式,要求百姓服从于顶头上司。一个种植园的黑奴听到一句格言:“人人为自己,上帝为人人”,但没有正确理解其含义,而把它改成:“人人为自己,上帝也为自己!”普通中国人对权力本质的看法与这个黑奴对古老格言的解释有着相似之处。作为百姓,他认为,“我有责任照顾好自己”,至于政府,他认为,“政府既成熟老练又强大无比,完全能照顾好自己。用不着我们操心”。作为政府,尽管是家长,但更多的是在照顾家长自己,而不是照顾他的家庭。一般说来,若不是百姓有难,政府是不会为百姓着想的;而由于事先没做什么,事情发生后,就必须做得更多。百姓明白,政府努力减轻诸如洪水泛滥造成的灾害,目的是为了保证税收不受损失。而百姓自己努力防治这类灾害,则是出于自我保护的本能;因为若是让政府来办理这类事,反而要增加百姓大量的苛捐杂税。 
  中国的道路状况很能说明政府对公共事务的不重视以及百姓缺乏公心。在这个国家,各地都曾有过宽阔的标准公路;这些公路用石子铺成,两旁种着绿树,连接着许多最重要的城市,但这些道路现在都已损坏,这种情况不仅在北京附近的地区可以看到,像湖南、四川这些很远的地区也同样可以看到,筑路需要花大笔的钱,而维护保养则相对要容易些;但是政府和百姓都忽略了维护保养,以致于造成这类公路的损坏,严重妨碍了交通,甚至整条路都报废。假如这些交通要道的毁坏发生在明末清初的动荡年代,那么,扣除政治动荡的那几年,至今少说也有250年,足以修复这些交通干线。但是,这种修复工作从未有过,甚至也没人提出过,其后果就是我们今天所熟悉的这种状况。 
  政府的态度与百姓的态度是相互影响的,百姓关心的只是个人不要遭受损失,而不顾公共财产到底会怎样。事实上,中国人就从没想过,一条路或其他什么东西是属于“公共的”。“河山”(即国家)被认为是当朝皇帝世袭的财产,他在位多久就占有多久。道路也是他的,若要修复什么的,让他去干,而从另外的意义上说,穿过农田的那部分道路又是不属于皇帝的。农田是农民凭力气开垦出来属于自己的,农民要怎么用就怎么用,不必征得土地所有者的同意,因此,穿过农田的那部分道路归农民所有,但是,通过农田的那部分道路同其他田地一样,要支付税赋,因此,这部分道路的所有者所获得的好处并不比其他人更多。在这种情况下,农夫要尽可能地利用道路,他在道路两旁扩展沟渠和田埂,使道路越来越狭窄,交通也更为困难。若是夏季暴雨冲毁农田,道路和农田混在一起,农民会在路上重新开挖出自己的农田。这样,再加上自然的狂风暴雨,原来的道路最后成了一条水沟。中国人根本没有我们所说的“道路权”的概念。 
  在潮白河上乘船旅行,途经天津与北京的交界处时,偶尔会看到河面上有小旗;经打听才知道,这些小旗是用来指示下面有水雷,过往船只必须绕开而行!据说,中国的一支部队在进行军事演习时居然直接在大路上打炮,结果是,交通中断,牲口受惊,一片大乱。 
  车夫在马路中间装卸货物,过往的人只能等到他干完活才能通过。农夫砍树,他会把树横倒在路上,过往的人只能等到他把树砍断、搬开为止。 
  乡村自由自在的生活方式正好与城里的占道行为形成对照。北京宽阔的街道两旁排满了各种货摊。这些地方本不是摆摊之处,如果皇帝经过那里,这些货摊都会很快搬走,皇帝一走,又都回到原处,在大多数中国的城市里。狭窄的街道上排着各式手工作坊。杀猪的、理发的、肩挑卖吃的,做木工的,修桶的以及其他无数工匠,都插进小街的两旁,与城市生活溶为一体,并成为令人窒息的拥挤之处。更有甚者,女人们把被褥拿出来,晒在街上,因为他们的小院子远不如街上来得宽敞,中国人几乎没有不能摆到街上的东西。 
  这些不只是造成交通阻塞。木匠在其摊位前留下一堆木块,染色工把长长的布匹挂在高处,卖面条的沿街晒起面条,因为摊前的空地不是属于“公共”的,而是属于摊主的。但是,对所占道路具有所有权,就需要有相应的维修的责任,现阶段的中国人根本没想过这一点。一个人即使想维修道路(这是不可能的),他也没时间和所需的一切;而许多人合在一起干,更是不可能的,因为每个人都生怕自己比别人干的更多而得到的好处却更少。如果地方官员要求沿路的村庄保证所属范围里的道路畅通,那一切就好办多了,但是,任何中国的官员都肯定想不到这一点。 
  中国人对属于“公共”的东西不仅不当一回事,或不加爱护,或占用,甚至还偷盗。铺路用的石子被人拿去用了,城墙上的方砖日渐减少。在中国的某个港口城市,外国人墓地的围墙被弄得一块砖也不剩下,据说是因为那个地方并不特别属于什么人。不久前,北京紫禁城里发现某些建筑物屋顶的铜饰物被盗,在皇宫里引起了一场非常大的轰动。中国人普遍知道,在18个省份中,皇帝是最容易被骗的。 
  中国人是否有爱国心?这是一个经常提出来的问题,同时也不是用一句话就能回答清楚的问题。中国人,尤其是知识分子,无疑是具有强烈的民族感情的,因此他对外国人抱有敌对情绪,并认为西方人的发明源于中国。近年来,湖南省出现大量排外的文章,恶意诽谤外国人,妄图引起混乱,把洋鬼子赶出中国。在中国人看来,出版这些文章的举动是值得赞赏的,正如我们看待反无政府主义一样。这场运动主要是由于误会,另一方面,也由于是对西方国家的憎恨。也许有许多中国人都认为,这场运动充分体现了爱国主义精神。但是,这些写文章的中国人是出于为国效劳的需要,还是为了获取报酬,这个问题需要有更多的证据才能说得清楚。一个中国人是爱国者可以不必非常关心当前清朝的命运,但我们完全有理由认为,无论怎样改朝换代,民族的整体感情是与今天一样的——这就是极度的漠不关心。对此,孔夫子曾在《论语》中含蓄地说:“不在其位,不谋其政。”在我们看来,这句意味深长的话一半是结果,另一半则在很大程度上是造成中国人对与自己无关的事漠不关心的原因。 
  对此,哈克先生有一个很好的事例:“1851年,道光皇帝驾崩。当时,我们正离京外出旅行。有一天,我们在一家客栈喝茶,试图在在座的中国人中发起一场小小的政治讨论。我们谈到了皇帝在近日驾崩,这本是一桩肯定会引起每个人兴趣的重要事件。由于皇位的继承还没有公布于世,我们对此表示担忧,于是就问:‘你们谁能知道。皇帝的3个儿子中哪一个会继承皇位?如果是大儿子,他会延袭现行的政府体制吗?如果是小儿子,他还大小,据说朝廷中有两派,他会倾向哪一派?’我们简要地提出各种猜测,目的在于激发这些善良百姓提出自己的看法。但是,他们根本不听我们的。我们一次又一次地向他们提出有关问题;这些问题在我们看来是相当重要的,但他们只是摇头,只顾着抽烟、喝茶。他们的冷漠确实引起了我们的不满。这时,这些够格的中国人中有一位从凳子上站了起来,走到我们身边,完全是以一种长辈的架势,拍拍我们的肩膀,讥笑地说:‘朋友,听我说,你何必要为这些吃力不讨好的事操心劳神呢?大臣们关心国家大事,他们吃的就是这碗饭。他们挣他们的钱,我们别为与我们毫不相干的事烦恼,我们傻乎乎地去关心政治,图个啥!’其他人也随声附和,‘是这么个理。’于是,他们示意我们,茶凉了,烟抽完了。” 
  曾记得,1860年英国军队进攻北京,用的就是从中国山东买来的骡子。天津和通州为了各自的利益签订了投降条约,提出只要英法联军不侵犯这两座城市,他们同意提供所需的一切。为外国军队干苦力活的绝大多数是从香港雇来的中国人。这类苦力被中国军队俘虏后,被剪掉辫子又送还给英军,——不难看出,如果说中国人真有爱国心,真有公心,那么这些词的意思也不同于盎格鲁一撒克逊人使用该词时所包含的意思。 
  当人们被迫起来反抗统治者的压迫和苛捐杂税时,总会有一些人站出来成为带头人。此时,政府会作出妥协。但事后,“受骗”的群众无论被如何处置,带头人总难免为了正义而一死。为了正义甘冒危险、愿献生命,这才算得上是公心的最高体现。 
  中国历史上的关健时刻,尤其是改朝换代时,总有一些志士仁人挺身而出,担当起重任,勇敢地献身于他们所崇敬的事业。这些人不仅是真正的爱国者,他们本身也无可辩驳地证明,中国人能够在具有公心的领导人的带领下,激发出极大的英勇气概。 

\chapter{因循守旧}
中国人比任何其他民族都更确实地相信,已经过去的时代才是他们的黄金时代。古代圣人带着无比的崇敬谈论着更古的“古人”。孔夫子说他不是一个创始人,而是一个继承者。他的天职是把所知道的一切,包括长期被忽略的和被误解的,收集起来。正是他在完成这项事业中所表现出来的执著和才能,使他成为他这个民族非凡的圣人。正是他的崇古、述古,构成了儒家成圣学说的基本主张。按照儒家的道德理论,有好的君主,才有好的百姓;君主是盘子,百姓是盘中的水;只有盘子是圆的,水才是圆的;若盘子是方的。水也就是方的。根据这种理论,只有明君统治时代,才有美德的盛行,“尧舜”时代,夜不闭户,因为没有盗贼; 路上丢失了东西,最早看见失物的人会守候在那里,并与其他来人轮流守候,直到失主完好无损地领回失物。这些事就连大字不识的苦力也会对我们说起,常常可以听到这样的说法:就美德与正义而言,现在不如过去;就违背良心而言,过去不如现在。 
  对现状不满的倾向并非只有中国或中国人才有,世界各国都同样有;只是在中国这种倾向似乎更为严重。人们相信,古代一切最美好的东西都保留在经典作品中,而今天只是继承而已,因此,这些作品被当做纯粹的偶像。传统的中国人视中国的经典就像正统的基督徒视希伯来语的《圣经》一样,以为其中囊括了过去所有最高、最美的智慧以及从古到今都普遍适用的一切。虔诚的基督徒根本不相信《圣经》还需要增加些什么,而中国儒家更不相信中国的经典还需要有所增补。基督徒与儒家都认为,一切都尽善尽美,还想更好,那是不可能的。 
  正像许多虔诚的基督徒用《圣经》的“经文”为《圣经》作者从未想过的事辩解一样,儒家学者也经常能从“古圣人”那里找到现代政府行为的依据,以及古代数学乃至现代科学之源。 
  古代经典铸造了中华民族,也造就了中国的政体;无论这种政体的质地如何,至少它是经久耐用的。自我保存是个人同时也是民族的第一法则。一种统治方式经过长时间的运用最后仍然适合,这种统治方式就可能被奉为经典。如果某位研究中国历史的学者能够对中国的政体如何形成为今天的样子有清楚的了解并成功地予以解释,这就是一桩惊人的发现。从他的发现中,我们肯定会清楚地看到,为什么中国几乎没有经历过其他民族所经常发生的那种政体改革,曾有一个故事,说的是一位工匠砌一堵石墙;墙有6尺厚,4尺高。问其原因,他回答说,这种墙若是被风吹倒,反而会更高!中国的政体根本不可能被推翻,因为他是一个立方体,它翻倒时,只是换了个面;无论是外表还是内在本质,都与原来的一个样。这种过程的反复出现,使中国人懂得了其结果肯定是像猫用脚走路那样不会改变;于是,人们便开始相信当初设计建造者的天才。任何要求改良的建议都成了十足的异端邪说。结果是,古人毫无疑问地优于后人,后人自愧不如地劣于古人。 
  有了这些清楚的认识后,也就不难理解中国人为什么一开始就盲目固执地遵循过去的生活方式。对中国人来说,习惯与道德是同一回事,因为它们同出一源,本质相同;这种看法与古罗马人是一致的。对中国人来说,侵犯他们的风俗习惯就是侵犯了最神圣的领域。他们无需从最终的意义上理解这些风俗习惯,或者严格地说,完全理解它们;而只要像母熊保护她们的幼仔一样,出于本能地坚决予以维护。这不只是中国人才有的本能,它是人类所共有的本性。值得重视的是,成千上万的人所乐意为之献身的信仰却是一种他们所不了解、而且并不用以规范自己生活的信仰。 
  中国的风俗习惯,正如中国的语言一样,我们并不知道它们是以何种方式形成的。风俗习惯,如同人的言语一旦形成,便难以改变。但是,中国的风俗习惯与语言形成的条件是各不相同的。因此,我们会看到有各种令人眼花镣乱的风俗习惯可用以说明常言所谓的十里不同俗;同样也会听到有令人莫名其妙的方言。风俗与方言一旦形成并固定下来,就会像成形的熟石膏,即使打碎了,也还是不会改变。按理说是这样,但实际上肯定会与事实有相当大的出入,因为没有哪一种风俗习惯是永恒不变的,在某种新的条件下,改变总会发生。 
  下面的事例最能说明问题。清政府曾在中国国民中推行一种全新的削发方式,绝大多数的人极力反对这种改变,宁死不从。但满族人却一直就是这样做的,并以此作为忠诚于皇帝的标志,表现出他们能很好地适应这种削发方式。推行新的削发方式的结果,正如我们所看到的,今天的中国人最引以自豪的莫过于他们的辫子,至于对清政府这一做法的仇恨只是残存于广东、福建本地人曾用来遮盖民族耻辱的头巾中。 
  佛教进入中国,只是在一场最具决定性的战争中才得以实现;而一旦完全扎下根来,它就像土生土长的道教,难以替代。 
  中国的风俗习惯从最初形成到今天的样子,很容易使人得出一个基本的假定,这就是,现存的就是合理的。长期形成的习惯是一种专制。无数人遵从习惯,但没有一个人知道这样做的缘由。他的职责是遵从,并且他遵从了。在中国,对宗教的信仰程度因地区不同而迎然相异,但有一点是可以肯定的,这就是成千上万做过“三大宗教”所有仪式的人,他们根本不懂什么叫信仰,就像他们不懂埃及象形文字一样。若是问起某一宗教成规的原因时,通常只有两种回答:一是认为,与上帝沟通的各种做法都是古人传下来的,肯定自有其牢靠的根据;二是认为,“每个人”都这么做,我也应该这么做。在中国,是机器带动齿轮,而不是齿轮转动机器。既然是放之四海而皆准,那么,只要遵从准没错。 
  蒙古人有一个习惯,任何人,只要有鼻烟,就会分给他的朋友,每个人都带有一个小烟盒,遇见朋友,他就拿出来;即使盒内的鼻烟已经吸完,他也要把烟盒递给朋友,他的朋友会装着从中拿起一撮,然后把盒子送还主人。如果客人把盒子看成是空的,那就有失“体统”,而按照适当的习惯去做,则会保全主人的“面子”。这一切都是按照既定的惯例;在许多特定的场合都只能是这样。珊瑚虫早已死了,但留下了珊瑚礁,为了避免翻船,行船必须一丝不苟地按照航线。 
  始终如一地按照以往的方式行事并非为中国人所特有。印度的苦力习惯地用头顶着东西进行搬运,并且用同样的方法为修建铁路运土。承包商叫他们用独轮车,苦力们反而把独轮车也顶在头上。巴西的苦力搬运东西的方式与印度苦力一样。一位住在巴西的外国绅士要佣人去寄一封信,他惊奇地看着佣人把信放在头顶上,再压上一块石头。思想过程的僵化导致行为模式的僵化,中国人做事也有这个问题。我们可以举出许多我们所熟悉的这种事例。最初教厨师做布丁时,打开一? ?τ是把它倒掉;后来这个厨师每次做布丁都把第一个鸡蛋打开后倒掉。拿一件有补丁的旧衣服,要裁缝照样子做一件新衣服,结果新衣服上也缝了一块补丁。说这样的故事并无意于夸大中国人的某种特性,但却是非常真实的事实。 
  每个对中国的风俗习惯有所了解的人都能举出中国人因循守旧的例子。对我们来说,中国人的因循守旧的确难以理解,除非我们明白了他们这种行为的根本原因。住在北纬大约25。的中国乡村,人们按照常规,冬天不穿皮衣,夏天需戴草帽,若不是这样,那才怪呢!有的地区,只有到了非常冷的冬天才烧炕取暖,如果旅行者正巧赶上突如其来的“寒潮”,通常根本无法说服店主烧炕,因为烧炕的季节还没到! 
  我们都知道,中国的工匠不愿意采用新工艺,但最守旧的莫过于外国人窑厂中的烧砖师傅。有一次,需要用到比当地所流行的砖更大一点的方砖,于是,外国老板下令烧制这种砖。实际上,只需要准备一个尺寸稍大的木模子就行了。但需要砖时,砖却没烧制出来。把接受任务的烧砖师傅叫来一问,他说他拒绝参予任何诸如此类的创新,他的唯一理由是:天下没有这种模子! 
  无论对中国这个大国的未来是有兴趣,还是没兴趣,人们都不可能不看到,中国人的因循守旧会影响到外国人与中国以及中国人的关系。19世纪的最后25年是中国历史上的关键时期。有大量很新的酒提供给中国人,但是,中国人只有各种很旧的酒囊用于装酒。中国人由于天生因循守旧,几乎没有接受多少新酒,而且,就是所接受的那一点点,还是找新瓶来装的。 
  中国人目前对西方各国的态度是一种拖延的态度。一方面,他们不太想要新的;另一方面,又根本不想放弃旧的。正像我们看到古老的土屋,本该早就归还给泥土了,但却用粗糙的泥柱子支撑着,拖延着不可避免的倒塌;已经过时的旧风俗习惯和旧宗教信仰仍然被支撑着,仍一如既往地履行着旧的职责。“旧的不去,新的不来”,这个是没有道理。由此及彼的变化过程可能会长期受到阻挡,但也可能会突然实现。 
  当初,把电报引进中国时,沿海某省的英国总督禀告皇上说,当地人对这桩事抱有很大的敌意,以致于连电线也架设不起来。但是,后来与法国人开战时,不仅架设电线的支架完全不同,而且有关当局还立即建起了电报站,电报受到了欢迎。 
  不久前,许多人还相信风水,坚决反对在中国修建铁路。最早的铁路很短,只是建在开平煤矿的出口处,由于要经过一大片中国人的墓地,坟墓需要搬迁,这与英国和法国的情况是一样的。只要看一看被一分为二的墓地就足以计人们相信,风水与火车狭路相逢时,风水根本不是火车的对手。后来这条铁路的延伸显然是由于财政问题才担搁了,与风水完全无关。 
  中国人在处理重要的事务中,既有天生的因循守旧的一面,又会有侵犯惯例的另一面。在中国,孝道是最重要的;一位大臣的父母去世,他必须离职回家守孝,但是,宰相则不然。皇帝会不顾他一而再、再而三的“含泪”申诉,要他在本该守孝的日子里继续尽心尽责于国家大事。在中国,最不可改变的是君臣父子关系,君为臣纲,父为子纲。然而,在最近一次的皇位变更中,由于皇位由旁系亲属所继承,而小皇帝的父亲仍健在,这样,小皇帝的父亲要么就要自杀,要么就退休。光绪继承皇位,其父亲醇亲王就得辞官。醇亲王得病,其儿子光绪皇帝只能以探望下臣的名义探望其父。既是儿子的父亲,又是儿子手下的重臣,因此还需要有某种权宜之计。 
  如前所述,因循守旧的本能使得中国人过分地看重成规。但是正确地理解并谨慎利用中国人的这一本能,可以使之成为外国人的重要的保护伞,从而使他们能够顺利地与一个如此敏感、如此固执又如此守旧的民族打交道。外国人只需模仿中国人的做法,把一切都看做是理所当然的,装着好像真有那么一回事似的,在受到责难时为其辩护,并且想方设法坚持下去。因此,居住在内地和其他城市的外国人只要像居住在北京的外国人一样明智地采取一种墨守成规的方式,就不会有什么事。险恶的暗礁对于航船似乎是一种无法通过的障碍,但一旦穿过,便可进入一片神秘而又平静,不怕风吹浪打的环礁湖。


\chapter{随遇而安}
我们这里要说的是,中国人对于舒适与方便的不讲究。但这只是依照西方人而不足东方人的标准来说的。因此,本章实际上主要是谈论东西方人在所谓舒适与方便问题上的根本差异。 
  首先看看中国人的服装。在前面谈到中国人轻视外国人时,我们已经偶尔论及西方人的服装式样几乎不能力中国人所接受。在这里,我们要说的是,中国人的外观打扮也会令西方人所难以接受。中国人在外观打扮上,把头的前半部位的头发剃光,让本应得到保护的部位暴露于外;当我们看到—个伟大的民族居然会有这样一种反常的打扮习惯,肯定会感到意外。如前所述,中国人是在刀尖之下被迫采取了这种削发方式,并以此作为忠诚于皇帝的标志,既然如此,我们就不必对此作进一步的关注,而只要看到这样一种事实:中国人自己并没有感到这样做有什么不舒适,或许她们压根儿就没想过要恢复明朝的削发方式。 
  同样,我们也只能这样地去看待中国人几乎一年四季,尤其是夏季,不戴帽子的习惯。在炎热的夏季里,所有的行人都举着扇子遮挡太阳。其中也有一些人是用阳伞,但肯定只是很少的一部分。中国的男人常常因为是戴帽子而引起别人的讨厌。据我们观察,中国的女人只有装饰用的头巾,当然,在挑剔的外国人看来,这头巾根本起不到装饰作用。中国人认为,夏天带一把扇子在身边,就够舒服了。在夏天,经常可以看到苦力们光着膀子,拼命地拉着沉重的盐船逆水而上,一边使劲地扇着扇子。即使是乞丐,经常也是打着把破扇子。 
  中国文明有许多不可理解的现象。据说这个民族是最早的游牧民族,照理说,在利用这一天赐良机方面,他们肯定有相当高的水平。但是,他们却不懂毛纺技术。唯一的特例是,我们看到这个国家的西部有一些毛纺业。但这种技术并没有得到普及,我们仍然看到成群的羊漫游在山野之中。 
  在棉花传人之前的古代,这个国家人们的衣服是用诸如灯心草一类的植物纤维制成的,这一点是可以相信的。但同样可以肯定的是,现在整个国家完全是依靠棉花织布制衣。在那些冬季非常寒冷的地区,人们要穿好多件衣服,把身体裹得严严实实的。一个小孩穿了这么多的衣服,若是摔倒了,常常是爬也爬不起来,就像是掉进了一个桶里。但我们从来没听说有人抱怨穿这么多衣服难受。既然要穿就不要怕难受。然而可以肯定的是,没有一个盎格鲁-撒克逊人愿意忍受这种束缚,他们会想方设法摆脱它。 
  与冬天穿那么多衣服形成对照的是,他们没有任何内衣。在我们看来,不穿那种可以经常换洗的棉毛内衣会坚持不下去。中国人没考虑过这种需要。他们用那么多衣服裹住身体,像套着许多层的袋子,但却有许多空隙,让寒风透过刺人肉体;而且,他们并不在意这种状况,尽管也承认这种衣服不理想。一位66岁的老人说他已经被冻麻木了,我们给了他一件外国内衣,并告诉他每天都要穿着,免得受寒。一两天后,我们发现他居然把它给扔了,因为他都快被“烤死了”。 
  中国人穿的鞋是用布做的,经常会渗水,稍一沾水,里面就潮湿。天气一冷,他们就会觉得脚底整天冷冰冰的,中国人的确还有一种上了油的靴子,是用来防潮的,但是,它虽好却很贵,使用的人也只限于极少数。雨伞也是如此。它们被看做是奢侈品,而不是生活必需品。被雨淋湿的人,他们并不认为应当把湿透的衣服换下来;他们觉得让身体来焐干衣服并没有什么不好。中国人说外国人的手套很好,但自己就没想搞一副;那种不灵便的连指手套,他们不是不知道,但即使在北方,也难得一见。 
  按照外国人的看法,中国服装最令人烦恼的是没有口袋。外国人一般都希望衣服上有许多口袋可用。他要在外衣胸前的口袋里装记事本,下面的口袋装手绢,衬衣的口袋放铅笔、牙签、怀表之类,还有其他地方要放小刀、钥匙和皮夹子。如果一个外国人身上还带有小梳子、折尺、开塞钻、靴扣、镊子、指南针、小折叠剪刀、弹子球、小镜子和自来水笔,这对外国人来说并不算什么稀奇事,这些东西是他经常所要用的,不能少。而中国人几乎没有用到这类东西;就算要用,也没有口袋放这些东西。如果他有一块手帕,他就把手帕塞在怀里;他若是带着孩子,他的孩子也是这样。如果他要带一份重要文件,他会认真地扎紧绑腿,把文件塞在里面,然后上路;有时,他干脆把文件往裤腰一别。在这些情况下,若是带子不知不觉地松了,文件就会丢失——这是常有的事。如果身上还要带其他什么东西,他们一般都把东西放在卷起的长袖里,或帽子的某个地方。中国人很难在身上找个放小东西的地方,可以把钱卷成小筒架在耳朵上。像钱包、烟袋和烟杆一类的小东西,为了保险起见,就系上带子,挂在腰带上。如果带子松开了,这些东西就有可能丢失。钥匙、梳子和一些古钱系在外衣的纽扣上,脱衣服时还得小心,以防这些东西丢失。 
  对我们来说,如果普通中国人的日常外衣是令人感到不舒服的,那么他们在晚上的穿衣方面就更是不必多言了,因为他们是脱光衣服睡觉的。无论是男人或是女人,他们都没有睡衣。据记载,孔夫子曾要求穿比身体长一半的睡衣,但据推测,他所要求穿的并不是普通的睡衣,而是他节食期间所穿的长袍。但有一点是可以肯定的,这就是,现代中国人并没有人仿效孔夫子穿这种晚礼服,也没人有意去节食。就新生儿而言,他们由于不知道婴儿对温度变化异常敏感,而不注意给婴儿盖好被子,甚至随意掀开被子,向他人展示自己的孩子,仅凭这种荒唐的做法就足以解释为什么许多中国婴儿满月之前会死于因突然受寒而发作的惊厥。当孩子稍大一些时,有些地区的中国人并不是使用尿布,而是用一种沙土袋。对于西方国家心疼孩子的母亲来说。只要有这种想法就足以令她们恐怖万分。这些可怜的孩子吊着个怪物,就像青蛙“背”着个铅弹,无法动弹。在这种做法流行的地区,若是一个人没有实际经验,就会被说成是还没脱掉“土裤子”。 
  中国人很能将就不仅表现在服装上,而且也表现在住房上,为了说明这一点,我们不必去说那些没自己房子的穷人,中国人并不在意在房子的四周种树遮荫,而宁可搭个凉棚。若是没这个能耐,他们本可以在院子里种一些遮荫的树木,这又不难。但他们不是种这种树木,而是种一些行榴之类的观赏性的灌木。酷暑来临,院子里热得受不了,他们干脆坐到街上去;再不行,就回到自己的房间里去。大多数的房子只有南门而没有北门,因此无法形成空气对流。若是能开个北门,也许还能凉快些。当问起这等方便之事为何没人做,得到的回答往往是:“我们从来就没有北门!” 
  北纬37。以北的地区,中国人一般是睡炕。炕是用砖坯垒起来的,中间烧火加热。若是没有烧火,冰冷的炕会令外国人无法忍受。若是火太旺,他会在后半夜由于身上被烤得厉害而醒过来。无论如何,炕的热度总不可能整夜都一直很适宜。一家人就是睡在这样的炕上。此外,这种炕由于其材料的原因,还会生虫,即使每年都更换砖坯,也无法保证能赶走这些不受欢迎的客人,因为整个房间的墙上都被它们占据了。 
  大多数中国人都知道,许多害虫会传播疾病,但是,并没有人去防治这些害虫。不少人家的墙角上挂着蜘蛛网而没有人打扫,据说,即使是城里人也很少有例外。苍蝇和蚊子的确是很令人讨厌的东西,偶尔烧一些有芳香气味的草驱赶它们,但这种害虫并不会惹中国人生气。 
  睡觉的枕头应该是怎么样才算舒适,这也反映出舒适标准的不同。在西方国家,枕头是一个装着羽绒的袋子,正好支承着头部。在中国,枕头是用来支承颈部的;有的是一张小竹凳,有的是一截木头,更多的是一块砖头。以中国人的方式,枕中国人的枕头,简直就像是在受折磨;同样也可以肯定的是,没有一个中国人能在我们所使用的枕头上睡上10分钟。 
  我们在前面已经说过,中国人不懂毛纺技术。但更难以理解的是,他们似乎对家禽的羽绒也不感兴趣。尽管中国人非常节俭,但他们并不知道可以很容易地用羽绒制成被子,而让羽绒随风飘走,所以羽绒的价格很便宜,甚至不要钱。他们只知道羽绒可以卖给外国人,除此之外,就是把羽毛扎起来做成鸡毛掸;在中国西部,家禽的羽毛有时被厚厚地散布在地里,以免刚出芽的麦子和豆子被动物吃掉。 
  对西方人来说,理想的床是有弹性且坚固的。最好的床大概要数钢丝床,近年来,这种床已很流行。但是,当中国一家最好的医院添置这种高档用品时,居然有病人不躺这种弹簧床,而宁可躺在地板上,因为他们感到躺在地板上就像是在家里一样,这使得那位置办这些用品的好心医生十分生气。 
  中国人的房子到了晚上几乎总是很昏暗。当地的菜油灯发出难闻的怪味,而且灯光仅够用来照见东西。人们的确也知道用煤油点灯的好处,但绝大多数地区还是一直用豆油、棉籽油和花生油点灯,这纯粹是由于保守的惯性力所造成的,他们只是满足于仅仅能看见东西,而根本不在乎能进一步看清东西这种更高层次的舒适。 
  在西方人看来,中国人的家具既不灵巧又难看。中国人坐的长凳不像我们祖辈所坐的那种有靠背的宽大的长椅,而是一种没靠背又很窄的凳子;如果凳子的某条腿不牢固,或者凳子的一端没人坐,那么坐下去的时候,凳子肯定就会翘起来,这并不奇怪。在亚洲各国,中国人是唯一使用椅子的民族,但按照我们的看法,中国人的椅子是很难看的。其中有一种椅子的式样好像是英国伊丽莎白女王或安妮女王时代所流行的,很高,靠背又直,非常呆板。还有一种比较多见,样子看上去很大,可以坐一个大胖子,但支撑力不相称,估计很快就会垮掉。 
  西方人对中国人住所最不满意的肯定是潮湿和冷。由于房子的地基随随便便,房子经常会潮湿。对大多数外国人来说,房间里泥土的地面或用没烧制好的砖铺成的地面非常不舒适,而且对健康也相当不利。还有,房门松松垮垮实在是令人讨厌。这种门有两扇,根本关不密,四周透风,就算只有一扇门,而且用结实的纸把门缝糊好,也还是不能很好地抵挡住刺骨的寒风,因为中国人不懂随手关门,就是教也教不会。一位商人在他的办公室门上贴了一条告示:“请随手关门”,这在中国纯属一句废话,根本没人会随手关门。无论是房子或是院子,门框都做得很低,一般的人进门都要低头,否则就会撞在门框上。 
  中国人用纸来糊窗子,这种窗子抵挡不住风、雨、太阳、炎热,或灰尘。百叶窗不常见;就算有,也少有人用。 
  大多数中国人的家里只有一只锅,一口容量很大的大铁锅。但每次只能煮一样东西,煮饭的时候就不能烧开水。而且,还要有一个人蹲在灶口不断地往灶膛里添柴草。几乎每次煮饭都是这样。水汽和烟弥漫整个房间,若是外国人肯定会被呛得睁不开眼,或者喘不过气来,而中国人明明知道这会害眼病,但似乎还是无所谓。 
  对西方人来说,中国人的住所最不舒适的是冬天没暖气。绝大多数地区,即使是冬天最冷的地区,取暖只是靠煮饭的锅灶和炕。中国人对于炕的舒适予以高度赞赏,女人们有时称之为“自己的母亲”。但是,对西方人来说,这种炕实在是不舒适,因为西方人要求热源发出的热是适宜于人体的恒温。因此,在寒冷的夜晚,中国人的炕根本不如“壁炉”或者火炉来得舒适。在一些产煤地区,煤的确己被作为燃料,但就全国范围而言,这些地区非常有限,而且烧煤时,煤烟总是在屋子里出不去。木炭要非常节省地用,家境好的也不例外;就像烧煤一样,若是烧得不小心,危险还很大。天气冷了,屋子里冷得令人难受,待在家里的人往往把所有能穿的衣服都穿上;外出时,就没有衣服可增加了。我们问他们:“你冷吗?”“当然冷”,他们总是这样回答。他们给西方人的印象是,这一辈子就从来没有使自己暖和过。在冬天,他们血管里的血液就像河水一样,表层冻结了,只有底层在缓慢地流动。曾有一位中国的道台在国外说过,美国的监狱比他的衙门更舒适,我们如果知道中国人的住所是那种样子,就不会对这位道台的话感到惊奇了。 
  我们曾指出中国人对拥挤和噪声并不在乎。天气一冷,中国人肯定是挤在一起取暖。甚至是三伏天,也经常可以看到轮船上仅有几间的船舱里挤满了人,或是坐着,或是躺着,没有一个西方人能忍受这样的拥挤,而中国人似乎并不在意,西方人喜欢住独门独户的房子,既通风又不受打扰。中国人对是否通风和不受干扰并不在乎;即使他们有这样的条件,似乎也不觉得有什么好。中国的城市周围无计划地建有许多小村落,换言之,大家都挤在一起,好像是由于地价大昂贵;而恰恰是由于挤在一起才抬高了地价,上像在城里一样。结果是狭小的院子、拥挤的房子统统都挤在一起,人满为患,谈不上还有什么活动空间。 
  一位住在中国小客栈的中国旅客在吃完晚饭躺下睡觉后,欣赏着大队人马的光临而带来的喧闹。而他的外国旅伴醒着躺在那里直到半夜,头脑清醒地聆听着一大群骡子在那里嚼草、踢腿和长叫。这些响声时而还交替地伴有木头的撞击声和狗叫声。在一个小客栈的牲口棚里,看见有50头驴,那是常有的事,整个晚上会有想像不到的热闹。正如哈克先生所说,中国人并不是不知道在牲口的尾巴上吊一块砖就可以使牲口不乱叫,但就是没人这样做。道理很简单,中国人对50头驴怎么叫并不关心。而外国人却不愿意留下这种没搞清楚的问题。中国人不在乎动物的吵闹,这不仅限于某个社会阶层,而是中国人的天性。一位中国大官的太太一下子在家里养了大约l00只猫,这个事实很能说明问题。 
  中国的所有城市都有无人看管的狗到处侵扰,中国人对此熟视无睹与佛教的不杀生灵有关。然而,中国人的这种态度比起东方其他国家来,还算好一些。曾任美国驻华公使的罗斯·布朗先生出版过一本有关东方游记的书,书中配有他自己所作的插图;其中一幅画的是各种各样的狗正在举行一个会议,有精瘦的狗,还有癫皮狗;题为“君士坦丁堡大观”。书中同样有一些很能反映中国许多城市概貌的插图。中国人对这么一大群狗在那里无法无天地乱叫,似乎不感到有什么严重的不舒服,也不担心会被疯狗咬伤而造成危害,尽管此类事经常会碰到。就算被疯狗咬伤,治疗的方法也往往只是在伤口上敷一些狗毛;这种做法与我们的一句谚语有惊人的相似之处,这就是“被什么狗咬伤,就用什么狗的毛医治”。打狗似乎还没有引起人们的重视。 
  以上说的这么多都是为了说明中国人对舒适不舒适并不在意。同样也可以轻易地举出很多事例以说明中国人对方便不方便并不关注。下面只要略举几个事例,中国人骄傲地自称为有文化的民族,事实上,也是世界上有文化的民族。笔、纸、墨、砚被称为“文房四宝”。但是,这四件必不可少的东西,没有一件是可以随身携带的。当要用的时候,根本不能保证它们就能在手边;就算这四件齐备了,没有水来研墨,还是无济于事,如果不知道如何正确地把笔毛弄软,毛笔还是不能用,还很可能把笔弄坏,反而浪费时间。中国人没有像铅笔之类的书写工具替代毛笔;即使有,也不知道怎么用,因为他们没有削铅笔的小刀,也没有衣袋装铅笔。在前面说到中国人节俭时,我们已经力图证明他们具有用很不合适的工具做很好的事这种高超的技艺。但又必须看到,西方人经常采用的那种节约劳力的方式,却不为中国人所知。在西方的高级宾馆中,宾客们要用什么有什么——冷热水、灯光、供热和其他服务。而中国18个省份中最好的旅店也只是像下等的客栈;口头上说得很好,但客人却没能得到应有的服务;每当这时,客人只好到房门外大声叫喊,希望店主能听到他们的呼声,但往往事与愿违。 
  中国人的许多日常用品并非想要时就能弄得到,卖货的小贩时而来时而不来,也没有定规。有时候,连天天要用的东西都无法搞到,好像是被丢在了苏丹。在城里,夜间行路要打灯笼;而在一些城里,只能在那些带有灯笼沿街叫卖的小贩那里才能买到灯笼,正像我们向卖牛奶或卖鲜酵母的小贩买东西一样。在中国,城市人口所占的比重不可能会大,因为买东西不方便。例如,有的地方习惯于二月份卖建房用的木料;一根木料在一个集市上卖不出去,就又拖到另一个集市上去卖,拖来拖去,直到卖掉;若是卖不掉,就拖回去。如果一个人没经验,硬要在五月份去买木料,他根本就买不到,他会立刻明白东方智人所言:“世上的机会只有一次。” 
  在谈到中国人节俭时,我们已经说过,中国人买来的工具大多数是还需要加工的;消费者买来一些部件,然后根据需要自己组装,这在我们看来是很不方便的。 
  笔者曾叫一个仆人去买一把劈柴的斧头。市场上没有,他买回的是14个(进口的)大马掌,然后请铁匠打一个斧子的头,再请木匠装一个柄,整把斧头所花的钱比外国的一把好斧头还贵得多! 
  在中国,最不方便的事是缺乏卫生设备,这一点给西方人留下难忘的印象。比如在北京,如果要改善一下排水系统,那么由此会产生更多需要治理的问题。一个人不论在中国住多久,他都会有一个既令他兴趣而经常提起又无法解答的问题:在中国,到底哪个城市最脏?一位从中国北方来的外国人对旅居厦门的人说,中国南方的城市在感觉上要比北方的城市更好一些。为了证实这种感觉,他们在厦门到处都走了一走,结果发现,真的是出奇的干净——这是就中国的城市而言。这位厦门的旅居者出于对旅居地的嫉妒,说了一句:在游览时,刚下过大雨,把街道都冲洗干净了!后来,这位旅游者到了福州,说是发现了中国最糟的城市;到了宁波,情况完全一样;到了天津,情况还要更糟,最后,如果他公正而诚恳地撤回他在北京时的看法,那也并不奇怪。 
  要说在中国生活不便利,西方人肯定有点印象:缺少邮局,道路状况差,货币流通不畅。私营的邮局当然有,它们也经营传递信件和包裹的业务,但发挥的作用很有限;与这么大的国家相比,它们的业务所覆盖的地区小得可怜。关于中国的道路状况,在前面说到中国人缺乏公心时,已经谈过。山东有一条几十公里长的穿山路,窄得两辆车子不能对开。路的两端分别有人把守,只允许车子上午朝一个方向行驶,下午朝另一个方向行驶!正是由于中国人的穿着一一尤其是中国人的鞋子一一是我们所描述过的那样,也正是由于中国人的道路是我们所知道的那样,因此,不管什么时候,一下雨,中国人就得待在家里。在西方国家,我们把下雨天不懂如何出门的人叫做傻瓜,而在中国,下雨天不懂待在家里的人才被说成是傻瓜。 
  中国人的言语中最常用的一句是“等雨停了再说”。除了政府部门之外,大多数中国人都认为,他们的活动要随着气候的变化而发生相应改变。就算是急迫的公务,在这句习惯语面前也变得不那么要紧了。我们听说,中国有一个很有实力的要塞,配备有精良的武器和经外国人训练过的士兵,但每逢下雨,岗哨上的士兵都自作主张地撤回室内,根本看不到一个人在露天站岗。他们是“等雨停了再说”!1870年的天津惨案,本来会有更多的人被杀害,但一场及时雨阻挡了前往租界行凶的暴徒。在敌占区受到追逐的外国旅游者,或许一场阵雨就能使他们受到最好的保护。一位在中国的外国人曾经看见,从两英寸长的水枪里喷射的水柱,5分钟内就能驱散气势汹汹的暴徒。橡皮子弹也远不如那么奏效,因为许多人也许会停下来去捡废弹头,而对于冷冰冰的水,自汉朝以来的每个中国人都像猫一样抱以反感。无论是从怎样的意义上讲,泼冷水都被看做是致命的。 
  关于中国人的钱,这个题目写一小段根本不够,至少可写一篇综合性的文章,或者一本书。其中各种怪事都有,足以使一代西方人发疯,除非能找到对付这种怪事的良策。在“漠视精确”那一章 里,我们已经说过一些令人莫名其妙的事。说是一百钱,但不是100个铜钱;说是一千钱,也不是1000个铜钱,谁也不知道到底有多少,只能凭经验知道个大概。在许多地方,一个钱可抵两个,20个可抵40个;所以当一个人听到有人要支付给他5oo个铜钱,他知道实际拿到的只能是250个,甚至更少;当然,各地还会有所变化。在钱中,混入小钱或是假钱也是常有的事,为此事,商贩之间常常发生争吵。地方官吏也为钱的短少不定期地颁布文告,打击掺假行为。这又使得衙门里的下属有机可乘,加重对当地所有钱庄的税收,给货币的流通造成不同程度的困难。人们需要纯真的钱,因而其价值立即上升。一旦市面上纯真的钱耗尽——失去其货币的功能——不纯的钱就会到市面上流通,且其面值并不会下降。这样,不纯的钱取代了较纯的钱,从而形成了不断起作用且不可克服的规律。钱的状况越来越糟,以致于在河南某些地方,人们上街需要带两种完全不同的钱,一种是通常用的好坏掺杂的钱,另一种则完全是假钱。有些东西只是付给假钱。而至于其他商品,如果是经讨价还价而成交的商品,就要付两倍的价钱。 
  中国人的钱实在是“脏钱”,无脏不成钱。500钱或l000钱(名义上的)的钱串子很容易断,散落的铜钱重新数、重新穿起来麻烦得很。通行的铜钱重量不一,但都是又笨又重。相当于1墨西哥元的铜钱,其重量不轻于8磅。人们挂在腰带上的小钱袋总共只可以装下几百个铜钱,如果所需的钱超过这个数,那么带起来就麻烦了。用银锭买卖东西,损耗总是很大,并且用银锭的人不管是买还是卖难免受骗。如果是使用钱庄的银票,麻烦还是不少,因为一个地区的银票到了另一个地区或者完全不能通用,或者会被打很大的折扣;而拿到银票的人,当他去开出银票的钱庄兑现时,很可能会与钱庄的贪心鬼就所兑付的钱的好坏发生一场争斗。奇怪的是,在这些不利的条件下,中国人生意照做;正像我们每天都能看到的那样,他们已是习惯于这些烦恼之事,几乎不觉得有什么负担,对之叫苦连天的只是外国人。 
  外国旅行者在经过中国乡村时,经常会看到一只驴伸直地躺在地上,一条结实的缰绳绑着它的脖子,拴在一根柱子上。由于缰绳太短,牲口的头被吊起了45 ,好像脖子要脱臼似的。令我们奇怪的是,它为什么不会挣扎而弄破脖子,反而心甘情愿听任摆布。没有一头外国驴会这样。读者读到这里一定会觉得,中国人虽然生活在水深火热之中,却似乎仍然感到相当舒适;当然这只是按照中国人的舒适和方便的标准,而与我们的标准完全两样,这就是我们在一开始就提出的问题。中国人已经学会如何适应他们所处的环境。当遇到困难时,他深知困难是不可避免的,而以极大的耐心默默地承受着。 
  一些熟悉中国人及其生活方式的人,虽然他们也熟悉我们所关注的那些方面,但却经常断言,中国人不文明。这种判断是很肤浅且完全错误的,其所谓的文明与舒适是缺乏哲学依据的。中国现在的状况比起3个世纪前的状况无疑已发生了很大的变化,这种变化与我们自己所经历的变化是一样的,只有看到这一点,我们才能有一种公正的比较,我们不能把米尔顿、莎士比亚和伊丽莎白时期的英国看成是不文明的国家,但可以肯定的是,那个时期的英国是我们现在大多数人所不能容忍的。 
  现在没有必要去说大不列颠群岛在过去的3个世纪中发生惊人变化的各种复杂原因。更加精彩的是,最近五十年以来,人们关于舒适和方便的标准已经发生了根本性的革命。我们如果被迫回到我们的曾祖父和祖父的时代,也许就会提出我们活着是否有价值这样的问题。时代变化,我们也随之改变。中国则相反,时代没有变,人也没有变,舒适与方便的标准与几个世纪前一个样。但只要出现新情况,这些标准也将必然要改变。这些标准肯定会与我们已经习以为常的标准相同,这既不仅仅是希望,也不仅仅是要求。 

\chapter{顽强生存}
中国人极富生存能力,这构成了中国人其他特性的一个重要背景,而其本身也值得思考。可以从以下4个方面加以考虑:中国人的繁衍能力,对不同环境的适应能力、延长寿命的能力和康复再生的能力。 
  外国旅行者对中国人的第一印象是人口过多。中国似乎到处都挤满了人,事实也正是如此。日本的人口也很多,但是明摆着,日本的人口密度不如中国的人口密度大。就人口的相对密度与绝对密度而言,中国最类似于印度。不过,印度的民族和语言多种多样,而中国人,除了那些影响不大的民族之外,几乎是统一的。在这个辽阔的国家,无论我们走到哪里,都可以看到人口过剩。就算是有些人口稀少的地区,我们一般也能轻易地找到能够被接受的原因。令人胆战心惊的太平天国暴动,随后较小的回族暴动以及1877一1878年遍及5省的空前饥荒,使中国的人口减少了大概有好几百万。我们看到,在中国,战争所造成的破坏并不像在西方那样能得到很快的修复,这是由于中国人极不愿意离开自己的故乡,投奔他乡。尽管如此,我们还是不难发现,无论多大的破坏力都不及修复力来得强大。我们相信,只需几十年的安定和农业丰收,中国的绝大多数地区就会从本世纪那一连串的灾难中恢复过来。这种恢复的前提条件已经具备,这是有目共睹的;无论是否愿意,都不得不接受这个事实。在中国各地,无论是城市还是农村,最引人注目的是一群群的孩子,他们像查尔斯·兰姆在给好吹嘘的母亲泼冷水时所说的,“挤满了所有死胡同”。这么多的孩子靠什么为他们提供吃穿,这是中国社会永远解不开的谜;但必须记住的是,许多孩子并没有基本的“吃穿”;换言之,极度的贫穷显然不可能导致中国人口的减少。 
  要制止中国人口的迅速增长,唯一有效且持久的方法是采用鸦片、战争、饥饿、瘟疫等一类导致其民族灭亡的手段。中国人的繁衍能力之强,现有人口数量远远超过其他任何国家,这已是不争的事实。即使作最低的估计,中国现有的人口数也已达到约两亿五千万。这个数是可以肯定的;问题并不仅仅在于人口的数量,更在于增长的速度:,我们缺乏可靠的统计数字,只能靠宠统的不精确的方法得出结论,然而幸运的是,这种结论几乎不可能错。中国人结婚很早;传宗接代是中国人普遍接受的占统治地位的观念,其次才是爱惜钱财。 
  与中国人口的迅速增长相比,法国的人口状况正好相反,其人口增长率是欧洲最低的。最近,其居民的绝对数呈下降趋势。这个事实引起了这个国家对未来的严重担忧。而另一方面,中国人并没有任何比盎格鲁-撒克逊人衰退的迹象。上帝给予人类的指令中最早有文字记载的是,教导人类“在尘世间休养生息,落土为安”。正如一位学者所说:这个指令“已经为人类所遵从,并且只有这一上帝的指令为人类所遵从”。这在中国要比在任何其他国家都更加正确。 
  正如我们已经说过,中国幅员辽阔,几乎拥有各种土壤、气候和物产。无论是亚热带地区、近北极地区,还是这两个地区之间,中国人看上去都十分兴旺。若有所差别,主要是由于各地区本身的特征和该地区承受人口的能力所造成的,而不是由于各地人民适应环境的能力有任何内在的差异,来自广东、福建两省较小地区的中国人,他们移居印度、缅甸、暹罗、东印度,太平洋群岛,澳大利亚、墨西哥、美国、西印度、中美洲、或南美洲,我们从来听说过他们不能很好、很快地适应各种环境的事。相反,我们听到的是,他们适应得又快又好,并且比当地人更刻苦、更节俭,加之他们的团结和凝聚力非同寻常,以致于其他民族为了维护自身利益,要求“中国人滚出去!”正是在这种情况下,中国人不再大规模地整体移居国外,这对于其他民族的心态安宁来说,当然是最大的幸运。如果今天亚洲大陆的东部像中世纪的中亚那样,都是一些不可征服且极力把自己的能量朝向其他地区的人,那么我们很难想像我们每个人以及适者生存的教条将来会变成怎么样。 
  由于完全缺乏统计数字,我们只能最笼统地说一说中国人的长寿。中国各地都有非常多的老人,这种结论也许所有观察者都会同意。年长者总是很受尊敬;长寿是一种极大的荣耀,被列为五福之首。出生的日期,直至时辰,被慎重而准确地记录下来,需要时就报出来,尽管通常的计算方法并不严格,不准确,这在前面已经说过。坟墓的石碑上刻有死者的享年,但是,除了石碑的产地及附近地区外,只有很少坟墓有墓碑,因此,还需从其他方面推断死者的享年,仅靠墓碑实际上是不够的。 
  很少听说中国人有活到百岁以上的,但接近百岁的到处都能找到,如果认真去找,还会有不少。事实上,占中国人口大部分的穷人,他们营养极度缺乏;若是考虑到这种情况,那么不禁要问,这么多的人是怎样活到如此长的岁数的。众所周知,本世纪以来,所有西方国家的平均寿命都在不断提高;这是由于人们越来越注意生命法则,改善防病治病的手段。而另一方面,中国的生活条件与哥伦布发现美洲时相比似乎没有多大的变化,这一点应当引起注意。如果社会与医药科学能像过去的50年里关注英国那样关注中国,那么中国长寿者的数目肯定还会有非常大的增加。 
  住在中国的外国人都知道,几乎所有中国人都不懂得卫生规则,即使懂了,也公然不予理睬。那么,对自然法则的无知和违抗所招致的各种疾病为什么没能灭绝中国人呢,这一直是外国观察者想知道的问题。在中国,每年都有许多人死于完全可以预防的疾病;而事实上,这样的人数并不是多得不得了,这表明中国人在抗病和康复方面有奇特的能力。中国人为了一点小事就拼命,与其顽强的生存一样,都是中国人重要的特性。 
  我们已是多次遗憾地说到,由于缺乏重要的统计数字,我们不得不依靠外国观察者的记录;这些一年比一年更多且更有价值的记录来源于数量不断增加的外国诊所和医院。 
  为了说明中国人的康复能力,分析和整理每年度的医学报告是一桩非常有用的工作,其结果肯定既新颖又有说服力。然而,我们只能陈述几件事实,并略加说明。其中有两件为笔者所熟悉,第三件取自天津一家大医院所出版的报告。这些事例的说服力就在于它们互相联系且非常特别,但又能与我们大多数读者的观察事实相一致。 
  几年前,笔者与一个中国人家庭同住一幢房子。一天下午,听到窗下传来叫声,那窗台是用砖坯砌成,下面有个洞,洞中有个大蜂窝。一个才14个月的小孩正在那里玩耍,看见这个洞,以为是一个好玩的房子,就自作主张地爬了进去。这个孩子剃着光头,脑袋呈红色。蜜蜂或许是被这突如其来的侵扰所激怒,或许是把光头误认为大牡丹,停在光脑袋上就蜇,孩子被抱出来时,已被蜇了30多下。孩子哭了一阵,被放在炕上睡着了。手头上没有任何药品,患处也没有敷任何东西。整个晚上,孩子一点吵闹也没有;到了第二天,肿包全都消失了。 
  1878年,北京有一个外国人家庭雇佣的马车夫患了流行性斑疹伤寒,当时有许多人死于该病。马车夫患病后第13天,病情危机;他突然变得很暴,力气大得能抵几个人。3个照看他的人被弄得精疲力尽。那天晚上,病人被捆在床上,以防他逃跑,当看守者熟睡时,他设法解开绳子,完全光着身子逃了出去。大约凌晨3点,看守者发现人不见了,四处寻找,连水井也找过了,生怕他投井。后来,在一堵约10英尺高的院墙处发现了他的踪迹。他爬上一棵树,然后跃到墙上,再跳到墙外的地上,并马上沿着皇宫城墙的护城河跑去,两小时后,找到了他;只见他把头伸进城墙下涵洞的铁栅栏内。因为他的头热得不行,赶快到这里凉快凉快,显然,他已经这样待了很长时间。在带他回去的路上,他的热病居然完全消失了;尽管腿上还有点风湿痛,但肯定慢慢会好起来。 
  一个大约30岁的天津人,经常到中国军队的演习场去捡废炮弹壳,并以此为生。有一次,他偶尔弄到一枚炮弹,试图把它拆开来,结果引起爆炸,炸掉了他的左腿。他被送进医院,实施手术,膝盖以下被截去。这个人并没有因此改变这种危险的营生方式,又尽快回去捡弹壳。大约6个月后,类似的情况又发生了。他的整个左手掌被炸掉了,伤口破烂,右臂上部被炸得严重烧焦,鼻梁和上嘴唇被炸裂,右边脸颊、右眼的上眼皮,额骨的旁边和右手腕被炮弹片划伤,右小腿也被炸开很深的裂口,露出骨头,受了重伤的这个人昏到在地上,任凭日晒,孤立无助达4个小时。一位大官正好看见此事,便命令一些苦力把他抬到医院,自己也护送着走了两里路。抬的人显然是不愿意抬;只等那位大官一走,就把可怜的伤员扔进了沟里,不管死活。那位伤员尽管因流血过多而精疲力尽,但还是爬了出来,单足跳了500码,来到一家米店,找到一些吃的,用一个大篮子装起来,吊在脖子上,用一只没受伤的手吃饭。店主为了赶他走,只好用筐子把他抬到医院门口,让他在外面等死。尽管由于失血过多,脉搏微弱,几乎不行了,但他神志清楚,还能交谈。他吸食鸦片成痛,到了无法戒除的地步;但对于如此的重伤,除了第五、第六天有腹泻和轻微的打摆子外,完全没有其他不好的症状,四个星期后,他拄着拐杖被允许出院。 
  如果一个民族具有像中国人那样的身体素质,能够在战争、饥饿,瘟疫和鸦片的影响下生存下来,如果他们能进一步注意生理和卫生法则,有适宜的食物,保证营养,那么完全有理由相信,他们自己就足以占据世界的主要地区,并且还会更多。


\chapter{能忍且韧}
“忍”这个词,包括3层完全不同的含义。首先,它表示长期不抱怨、不生气、没有不满情绪的一种品质或行为;其次,表示默默地忍受或承受任何苦难、泰然处之——镇静自若地忍耐——的一种能力或行为;再次,它也可以作为坚韧的同义词。显而易见,这里所涉及的各种品质;与中国人的生活都具有非常重要的联系。对待中国人的各种特性,我们不能将它们分隔开来独立地进行考察,否则就会弄不清楚,而考察忍与韧这种特性尤其是这样。中国人的这种特性与他们“不紧不慢”、“漠视时间”的特性有密切的关系,与最能直接体现中国人忍与韧的“勤劳刻苦”更是有着不可分割的联系。以上有关章节中所说的内容本来己足以表明忍与韧是中国人特性中的主要美德之一,但由于只是附带提到,难免不连贯,而应当以更为全面的叙述加以增补。 
  在中国这一人口稠密的国家中,生活的水平之低,是名副其实的“生存斗争”。为了生存,就必须有生活资料,因此,每个人都得竭尽全力为自己获得这些生活资料。中国人完全可以说是“把贫穷变成了一门学问”。极度的贫穷和为生存而进行艰苦的斗争,这本身并不会使任何人勤劳刻苦;但是,如果一个人或一个民族具有勤劳刻苦的天性,那么,贫穷和为生存而斗争就会使这种天性得到最有效的发展,同样也会使节俭这一中国人重要的特性得到发展,而且,还会发展出忍与韧的品格。猎人和渔夫懂得,他们的生计靠的是他们行动的隐蔽和小心,以及等待时机的耐心;不论他们是属于哪一种民族,“文明的”、“半文明的”,或是“野蛮的”,他们总是隐蔽,小心而有耐心的。中国人长期以来一直在最为恶劣的条件下谋生,因此,他们能把最文明民族积极的勤劳刻苦与南美洲印第安人消极的忍耐结合在一起。 
  中国人心甘情愿为很少的报酬长时间地干活,因为报酬再少总比没有要好得多。长期的经验告诉他们,勤劳刻苦并不一定就能有更多的机会,西方人则以为,机会是勤劳刻苦的自然结果。所谓“自然的”结果,是指相应的条件具备后,结果随之而来。显然,每平方英里500人的人口密度,这样的条件并不适合于证实所谓“勤劳与节俭是幸运儿的双手”这样的格言。中国人只是满足于干活有钱拿,而这种满足正体现出他们忍的美德。 
  谈到己故的格兰特将军,他在环球旅行回来时,有人问他,他所见到的事情中,最出乎意料的是什么,他立即回答说,他所看到的最奇怪的事是一个中国小商贩凭着自己强劲的竞争力战败了一个犹太人。这件事的确意义重大。犹太人的品格至今已为人熟知,他们卓有成就,令人惊叹,但犹太人毕竟只是人类中的一小部分。而中国人则在世界总人口中占有相当大的比例。那个被中国人战败的犹太人与其他犹太人肯定没有本质的不同,而那个成功的中国人与其他数百万的中国人肯定也无本质的差别。因此,若是其他中国人有机会与犹太人竞争,除了竞争者的身份不同外,竞争的结果也许没什么两样。 
  中国人的韧性是世界一流的。如前所述,一位中国学生年复一年地埋头参加考试,直到90岁才如愿以偿,否则他死不瞑目。这样做并非为了报酬,而且也不可能有报酬,只是为了证明自己有超凡的韧性。这是中国人所具有的一种内在天性,就像是鹿的飞跑能力和鹰的敏锐视力。就算是商店门口最下贱的乞丐,也可以从他身上看到类似的品性。他不受欢迎,但他一而再、再而三地出现;他的耐性不衰,他的韧性不变,非要讨到一个铜钱不可。 
  有一个故事,说的是一位阿拉伯人的头巾被陌生人偷走了,失者非但没有因丢失了这件重要物品而去抓小偷,反而立刻去部落的墓地,坐在入口处;有人因这种奇怪的行为问他,为什么不去抓小偷,他镇定且具东方特色地回答:“他肯定最后会来这里的!”这个将消极等待予以夸张的故事使我们经常想到,这种情况不仅也存在于中国人的个人行为里,而且还存在于政府行为里。康熙皇帝的统治从1662年直至1723年,其时间之长,成就之辉煌,使他成为亚洲最受赞美的君王。然而,正是在这最伟大的中国皇帝的统治时期,被称为“国姓爷”的一位中国爱国将领竟敢在广东、福建两省沿海地区进行大肆破坏,居然连政府的战船也根本对付不了他。在这种情况下,康熙想出一个权宜之计,命令沿海所有军民朝内地后退30里,约合9英里,在这个地方,旧王朝的维护者就无法再进犯了。这个稀奇古怪的圣旨下达后,居然大获成功。“国姓爷”后撤了,终止了继续搔扰的计划,转而进发台湾,把荷兰人赶了出去,他也因而被封为“靖海侯”*,而被招安了。每个外国人读到这段难得的叙述时,都会赞同《古代王国》作者所作的评述:一个政府既然有足够的能力迫使这么多的沿海军民撤离城市与农村,不惜代价地退到内地,那么就应该有足够的能力装备一支舰队,去打败那些对留下的家园进行肆意破坏的敌人。 
  中国政府具有韧性的另一个例子也非常值得注意,它在旅居中国的外国人心目中至今仍记忆犹新。1873年,驻巴克尔和哈密的中国将军左宗棠受命平息回民起义。这次起义起初只是星星之火,后来像野火一样遍及整个中国西部,并波及中亚。所要面对的困难大得几乎无法克服。当时在华的外国报刊纷纷载文,嘲笑左宗棠的承诺与清政府通过贷款筹集资金支付大量军费所表现的昏庸无能。然而,左宗棠的军队前往平息暴动不足一年,就己进军到天山两侧,给起义军以沉重打击。他们每到一地,若遇粮草不足,就转而开垦土地,自己种粮,以作后备。正是这样一边打仗一边种地,左宗棠的“农垦军”彻底完成了任务,其功绩被看做是“现代国家中最卓著的”之一。*原著有误,历史上被封为“靖海侯”的是施琅。--译者注 
  在我们看来,中国人的忍主要表现为毫无怨言地等待和默默地忍受。据说,检验一个人的品性,真正的方法是研究他处于风雨交加,饥寒交迫之中所表现的行为。如果检验结果令人满意,就“温暖他,擦干他,让他吃饱,使他成为天使”。在当代文学作品中,经常有一种说法:遇到一个没饭吃的英国人就像遇到一头失去幼仔的母熊一样危险,这种情况无论是对所有盎格鲁-撤克逊人还是对不列颠岛的居民,都是适用的。可见,我们这些引以自豪的文明人仍然受到肚子的奴役。 
  笔者曾经看到大约150人,其中大多数人是走了几里路,去参加一次宴会,结果却碰到一桩倒霉事。宴会原订于10点钟开始,许多人都把宴会当做是早餐,但是宴会并未能按时开始。后来,又来了一些人,于是先来的人只好站在一边为后来的人充当侍者。后来者细嚼慢咽地吃着,那种小心谨慎的样子是中国人的一个特性,比起我们来要高雅得多。先来的人没吃东西,长时间耐心地等待着,然而又出乎意料地来了一些人,看来又得等。那么,这150位遭受冷遇的人会怎么样呢?如果他们不列颠岛的居民,或者是“基督教国家”的其他公民,那么,我们完全清楚他们会怎么做他们肯定会一直带着难看的脸色,直到下午3点坐下吃饭,并且还大骂一通,说自己运气不好。他们肯定会采取严厉的方法,“写一封带有5个‘先生们’的信给伦敦的《泰晤士报》”。但是,这150位中国人根本没有这样干、他们不仅没脾气,而且一直非常诚恳、礼貌地服从于主人,似乎他们的等待是无足轻重,早吃晚吃确实都是一样的。读者可曾知道,有哪一种西方文明能经得起如此意外而又严厉的考验呢? 
  中国人的神经紧张程度与我们的大相径庭,已经表明“神经麻木的说都兰语的人”像北美印第安人那样都是甘愿忍受痛苦而不抱怨的人。中国人忍受痛苦不仅靠毅力,而且靠耐心,而后者往往要困难得多。一位双目失明的中国人问外国医生,他的视力能否恢复,并且还干脆地说,如果不能恢复,他就不再为治眼操心了。当医生告诉他无能为力时,这个人回答说:“这下可心安了。”这并不是我们所谓的无可奈何,更不是绝望,而只是一种能使我们“忍受痛苦”的品格。我们把焦虑看做是现代生活的祸根,侵蚀刀刃的铁锈。而中国人却具有不着急的天性,这对于完全有理由着急的整个民族来说,的确是桩好事。地大物博的国家遭受着周期性的干旱、水灾,以及由此引起的饥荒,诸如打官司这样的社会麻烦事以及因某种不确定因素而造成的更令人担忧的灾难,困扰着成千上万的人,但结果却完全可能出乎观察者的意料之外。我们曾多次问一位被夺走了土地,房屋和妻子的中国人,以后会怎样,他总是回答:“再没有太平的日子嗲!”“那什么时候才有个头呢?”“谁知道?”“也许早,也许迟,但肯定麻烦不少。”生活在这样的条件下,除了无限止地忍耐,还能有什么更好的法子呢? 
  给外国人印象最深的也许是,那种不幸的灾难接踵而来时,中国人所表现的忍。外国人所最熟悉的那些中国省份,很少能幸免于水灾、旱灾和饥荒所造成的灾难,1877至1878年的那次令人恐怖的大饥荒,有几百万数不清的人遭难,这些往事今目击者久久无法忘怀。当时,由于黄河水泛滥,突然改道,给广大的地区造成了无法估量、难以想像的灾难。几个省份最好的地区都被破坏,肥沃的土地被冲毁,变成了一片黄沙地。几千座村庄消失了,死里逃生的灾民无家可归,四处流浪,陷于绝望。大批的人并非因自己的过错而突然家破人亡、陷于绝望,这对任何政府来说,都不是好对付的。自我保存是自然界的首要法则,那些无缘无故被迫陷入饥饿的人联合起来迫使有粮食的人拿出粮食分给饥饿的人们,难道还有比这更合情合理的吗? 
  在一些大城市,贫穷的受难者最为集中,确实有按某种方式发放救济,但是,救济相当有限,时限又短,并且不向灾民,即使是重灾民,提供任何药品,这也是事实。对于遭灾严重的那些人的以后生活,政府就没考虑那么多了。至于土地的开垦、房屋的重建、以及新环境下的重新生活,政府一概不管。百姓要求减免赋税,经常是得不到应允,除非是一而再、再而三地向地方官表明根本就没东西可以用来抵交赋税。对于一个来自西方国家的外国人来说,“面包、面包,否则就流血”的革命口号是很熟悉的,但很难理解,为什么无家可归,饥饿绝望的难民宁可在被洪水和饥荒摧毁的地区流浪,而不愿意团结一致向当地官员要求救助。这些地方官确实无力满足他们的要求,但总能迫使他做一些事,而这也算是开个头,以便能迫使他做更多的事。如果他不能“安抚”百姓,他就应该下台,就让其他官员取代他的位子。但是,我们反复恳切地询问大饥荒时的中国人,为什么不采取那样的举动呢?回答肯定就是一个词:“不敢”。如果还要花费口舌,硬要说,一个人与其饿死,还不如起来造反被杀死,就算是非正义的也没有什么不好,那么回答仍然是“不敢,不敢”。 
  中国人之所以不采取这种做法大概有两个原因。他们是最讲实际的民族,凭着本能感觉到这样做是徒劳的,因此他们不可能进一步联合起来。但我们必须相信,主要的:原因还在于中国人可以有本事无限地忍下去。正是因为忍,在中国可以看到一种最压抑的情景:成千上万的人明明可以轻易地夺得多得装不下的粮食,但却偏偏要让自己,默默地饿死。中国人对此怪事已是习以为常,以致于无动于衷,就像身经百战的老兵无视战争的恐怖一样。那些遭受苦难的人们已经注定要一辈子面对苦难,挣扎在死亡线上。灾难降临时,他们只知道承受,像是不可避免、不可战胜的一样。如果他们有能力用独轮车推着家人到能够讨到饭的地方去,他们是会这样做的。如果一家人走散了无法在一起,他们会尽可能地各自寻找生路,直到灾害过后再团聚。如果得不到救济,灾民们就会成群结队地冒着严寒,沿路乞讨,行走千里,穿过好几个省,希望找到粮食收成较好、较需要苦力、较能生存的地方,如果洪水退去,外出乞讨的农民回到自己的家乡,当泥土还非常烂,耕畜无法耕犁的时候,在泥土中挖开长长的裂口,然后在这小小的缝隙中熟练地播下一些麦种;于是又重新上路,乞讨为生,直到收割时才回来。如果天公作美,他又会重新以种地为生,不再乞讨,但是他心里非常清楚,倾家荡产和挨饿仍然可能发生。 
  灵魂不灭的一个有力证据是,人的灵魂中最高级的才能在现世中根本没有合适的机会得以施展。如果这个论据确凿,那么,中国人这种无法匹敌的忍肯定有其更崇高的目的,而不仅仅是使他们去忍受生活的苦难和被活活饿死。如果适者生存是历史给予的忠告,那么,一个具有忍这种天赋的民族,加上强大的生命力,肯定会有一个光明的未来。

\chapter{知足常乐}
我们已经看到,中国人在忍受任何困难方面具有令人惊叹不己、不可思议的能力。这种能力或许可称为一种心理悖论。明明是处于毫无希望的境地,他们却没有表现出一种失望;更确切地说,他们似乎是在作无望的抗争,并且往往不是为了希望。19世纪末叶,大多数民族都有一种烦躁焦虑的特性,而据我们所观察,中国人并没有这样的特性。他们似乎并不打算去迎接“一个美好时代的到来”,甚至从没想过会有这样的时代。 
  但是,“忍”和“韧”这两个词根本不能涵盖中国人全部的美德。我们还必须注意到他们往往能在不幸的环境中保持心灵的安宁和长期的精神愉快,我们可称之力“常乐”。我们的主要目的是要唤起人们对这种美德的注意;当然也可能会提出某些值得思考的问题,以帮助对这种美德的理解。 
  说中国人“知足”,并不意味着中国人只是满足于已经获得的,而不希望获得更多更好。正如我们在论及中国人的因循守旧时已经看到,他们的知足最能表现在他们对待自己的政体上。他们从来也没想过要改变那种政体。这就是中国人的脾气,对此,我们毫不怀疑。就一般的意义而言,因循守旧并非为中国人所独有,所有民族都有这种现象,不过中国人较为典型而已,很显然,中国人的因循守旧观念充满着整个社会,世代相传,完全压制了人们对于命运安排的不满。他们当然会遇到灾难,但却被认为是不可避免的。始终固守这种观点的人不可能会努力去推翻已建立的秩序,道理很简单,因为他们身受的压力过于沉重了。中国的知识界实际上是思想和行动的领导者,这是其他任何国家所不可比拟的。但是,中国的知识界却极力说服中国人,现存的体制是公认最好的。他们以史为鉴,旁征博引,以令人信服的实例告诫中国人,对他们的体制作任何改良都是行不通的。他们这种顽固不化的因循守旧正是这样慢慢地生长起来的。 
  中国人是一个相信命运的民族,对此还没有人完全认识到。在中国的经典中,有大量有关“天命”的论述。百姓之间也经常说到所谓“天意”。这种说法往往与我们所说的上帝无所不能极其相似。但在基本思想上还是有着根本的区别:对我们而言,“上帝”意味着一个任其观念中完全拥有并创造世界一切、能给我们带来关怀和预言的造物主;对中国人而言,“天”实际上指的是一个笼统的非人格的东西,而且还是完全说不清的东西,实际上就是命运。“好命”与“坏”,对中国人来说,其意思类似于儿童故事书中的“好神”与“坏神”。依据这些神秘的东西就能做成一切,或不能做成一切。 
  中国人的占卜、巫术、算命,其复杂的说法和做法是依据一种直线式的力的作用和相互作用。有数不清的中国人正是依据这种普遍实用的说法度日生活的。当然,在中国的不同地区,这种迷信对人们日常生活的影响不尽相同,但它们却是真实地存在于人们的心目之中。中国人,无论是男人或是女人,都总爱说:“这是我的命。”尤其是那些运气不佳的人更是这样。相信命运的必然结果是走向绝望,或者失望;带着这种念头的人,尤其是中国人,只能像病人一样等待着最后的解脱,等待着时来运转。也许中国人并不像土耳其人那样始终相信命运,也许中国人的“命”不同于“真主之意”,但是,一个民族如果像中国人那样相信命运的存在,相信命运是事物的本质,是不可违抗的,那么就肯定不会为改变自己的命运而坚决抗争。 
  历史是以实例教人的哲学,这是希腊人长期以来的看法。正如我们所看到的,中国人自己的历史一直是他们的老师,正是吸取了历史教训,才使他们形成了因循守旧的尽最大可能地屈从而毫无怨言,而不愿花力气撞墙以表现出或多或少的顽强不屈,“不能医治的伤疯必须忍受”,这一古老的格言是他行事的依据。简而言之,中国人知道富该怎样,穷该怎样;最重要的是,在这两种情况下,他知道如何知足。 
  中国人的常乐,我们必须视之为一种民族特性,与他们的知足有着密切的联系。他们所获得的幸福总是超过所期待的,与我们不同,他们总是尽量地自得其乐。普通的中国人没有过分的讲究。他们总是模范的客人。不管在什么地方吃饭,吃什么,他们都觉得够好的了。即使是为数众多的缺穿少吃的人,他们也始终保持不慌不忙,其样子显然会令我们感到惊异。 
  中国人的常乐一般与他们的好交际密切相关,这与盎格鲁-撒克逊人忧郁孤傲的性格形成鲜明的对照。中国人的主要乐趣之一似乎是与人聊天,无论是老朋友还是陌生人,差别并不大。毫无疑问,中国人所遭受的许多痛苦可以通过聊天而感到大为缓解。 
  值得一提的是,许多中国人乐于种树养花以点缀非常简陋的环境,并成为最大的爱好。有一句难以表达清楚的说法:“东西不多,应有尽有”。 
  对于我们的中国仆人,虽然有许多批评或许是公正的,但他们经常任劳任怨,为许多人长期做额外的工作,不仅没有怨言,而且经常觉得没有什么可怨言的,这的确又是难能可贵的。 
  中国的仆人,若是习惯于叫命苦,就会常常受到同伴的笑话,有时还会成为笑柄或口头禅。中国人不知疲倦地辛勤劳作,这在前面已经说过,但值得注意的是,那些纺线的人不仅能一直纺到半夜,为了节省一点油钱而在黑暗中劳作,而且不叫一声苦。他们起早贪黑,并视之为理所当然。像苦力、纤夫和独轮车夫之类,他们的劳动最为辛苦,但不仅没听到他们对世上分配不公有过牢骚,而且他们还常常放弃休息的机会拼命干活,并为一顿便饭而满足。有见识的旅行者经常提请人们注意中国劳工的这种很有意义的特点。霍西先生在《中国西部三年记实》中谈及扬子江上游时说道:“纤夫也值得一提。除了乐师和潜水员之外,他们几乎都是小伙子;他们总是愿意在岸上奔忙,吃饭的时间从不超过一刻钟,并且从来不发脾气。”阿奇博尔德·利特尔先生在《长江三峡之行》中也有类似的描述:“我们的5名纤夫紧紧攥住纤绳,踏着凹凸不平的岩石,一寸一寸地拉着船逆水而上。我无法用语言来赞美这些穷苦力的顽强和韧劲,他们两个月才挣两元钱,每天三餐只是粗米饭,配一点洋白菜,正是靠这些食物,他们每天从早到晚,使尽力气。” 
  笔者认识一位受雇于外国人的车夫,他经常是推着沉重的车子一连行走数月。在行程中,他必须很早起身,走到很晚,运送着沉重的物品翻山越岭;一年四季,不论气候如何,赤脚涉水;每到一处,还要为雇主准备食宿。干了这么多的活,得到的报酬却不多,而他却没有任何怨言;干了几年的活,他的雇主说他从来也没见过这个仆人发脾气!除了某些细节有所不同外所有的读者都可能切实地作出同样的陈述。 
  也许中国人生病的时候最能表现出他们乐观的天性。一般来说,他们对一切都表示乐观,也希望人人对自己的境况保持乐观。甚至对于极度的病痛,他们也常常表现出充满希望的乐观。我们知道,许多中国病人,遭受严重疾病,往往又极度贫穷,总是得不到适当的营养,身边又无亲人,甚至还可能遭到亲戚的冷遇或抛弃,几乎看不到一线希望,但是,他们仍然一直保持快乐与镇定。而在同样的情况下,盎格鲁一撒克逊人肯定会表现出烦躁不安的情绪。 
  我们相信,具有这种快乐性格的中国人决不在少数,每个在中国待久的外国人都会遇见他们。我们需要重申的是,如果历史所告诉的“适者生存”是真的,那么中国人就会有一个美好的未来。 


\chapter{孝悌为先}
讨论中国人的性格,不能不谈谈孝顺。这可不是个容易对付的课题。“孝顺”与我们不得不采用的许多其他概念一样,难以用英语词语将它准确地翻译过来。其意义也和我们所理解的大相径庭。汉语中还有不少包含这一意义的概念,其中与“孝顺”联系最紧密,也最常用的一个是“礼”。为了对此加以说明,并为讨论中国人的“孝顺”性格提供一个背景知识,最好先引用卡莱尔先生的一段话(引自《中央王国》):“礼是中国人所有思想观念的集中体现;在我看来,中国可以贡献给世界的最合适、最完美的专著就是《礼记》。中国人的感情靠礼来满足;他们的职责靠礼来实现;他们的善恶靠礼来评判;人与人之间自然的关系靠礼来维系——总而言之,这是一个由礼来控制的民族,每个人都作为道德的、政治的和宗教的人而存在,受家庭,社会和宗教等等多重关系的制约。”对这段话,威廉姆斯博士的评价最令人信服,他说:“将‘礼’译为‘ceremOny’很不准确,ceremony’的意义太过贫乏,而‘礼’不仅指人的外在品行,还包括支配所有真正的礼仪和礼貌的正确原则。” 
  翻阅一下“四书”和其他古代典籍,尤其是《孝经》,最容易让人确信,中国人十分重视孝顺。目前,我们只关注中国人现实生活中的孝顺观,看看他们是如何理解孝顺的,孝顺是如何成为中国人独一无二的特性的。要切记,中国人的孝顺是多侧面的,并不是在所有的条件下或所有的观察家都能发现其实质。 
  1877年,在上海召开的传教士会议上,雅蒂斯博士宣读了一篇论“祖先崇拜”的论文。在这篇精心写作的论文中,他具体阐述了自己三十年来在中国的观察与经验。在论文的开头,作者提出,祖先崇拜只是孝顺的一种表现形式,接着又说,“孝’这一概念,容易产生误导,我们应当警惕,以免误人歧途。在我们了解的所有民族中,中国人是最不孝顺的,不服从父母,他们一旦知道了自己的需要,就开始固执己见。”曾在中国生活了三十三年的、著名的中国典籍翻译家莱格博士,则断然否定雅蒂斯博士的观点。他宣称,他在中国的生活经验与此截然相反。这种相互矛盾的现象表明,人与人之间总存在着不同的观点,就像两支温度计一样。要想获得正确、全面的观念,就必须将这些互相冲突的观点联系起来,综合考虑。 
  长期的经验证明,中国的孩子,没有接受过如何正确听从父母的教育,我们把立即服从父母当成一条规则,他们对此却一无所知。可是,这些不受约束或半受约束的孩子长大之后,情形就不再像我们所预料的那样了。中国人认为,“树大自然直”,这个比喻就是说,孩子长大之后,自然知道自己应该怎样做,它也可能讲的是其他意思,但它确实为孝顺行为提供了理论依据。不过,这种现象似乎是由人们的孝顺观念、受教育的方式和各地孝顺的典型共同促成的。《孝经》中说:“五刑之属三千,而罪莫大于不孝。”还有一种最普通的说法:“孝为万德之首,其诚存于心,而不在行。以行而论,世无孝子。”中国人还特别指出,任何道德缺陷都可追溯到孝心。违背礼节是因为缺少孝心,不忠心耿耿是因为缺少孝心,不克尽厥职是因为缺少孝心,对朋友不忠诚是因为缺少孝心,临阵胆怯是因为缺少孝心。这样,孝顺的内涵就远远超出了行为的范畴,不仅包含了行为的动机,还包含了所有的其他道德内容。 
  一般人认为,孝顺实际上是出于感激。《孝经》敕令章对此作了强调。据孔子说,父母死,要守孝三年,因为“子生三年,然后娩于父母之怀”。守孝三年似乎成了对父母这三年养育之恩的回报。就是小羊羔吃奶时,还知道要跪着呢(羔羊,兽也,跪哺乳)!孝顺的人还要善待自己的身体,因为它是父母赐予的。不善待它,就等于忘恩负义。孝顺的人,当父母在世时,要竭力服侍;去世后,要经常祭拜。孝顺的人,要继承父道,子曰:“三年无改于父之道,可谓孝矣。”父母明显有了错误,作子女的也可以努力促使他们纠正。威廉姆斯博士引用《礼记》中的一段话,可以为证:“父母有过,下气怡然。柔声以谏,谏若不入,起敬起孝,说则复谏。不说,与其得罪于乡党州闾,宁孰谏。父母怒,不说,而挞之流血,不敢疾怒,起敬起孝。”令人担忧的是,在大多数西方国家,可以完全不听父母的告诫,然而,就连在中国都很少听到这样的事。在《论语》第二章 ,我们发现,孔子对孝作出了几种不同的解释。在不同的情况下,他的解释也不同。第一次是在鲁国一位名叫孟懿的官员请教时,他只简单地说了句:“无违。”意思很容易理解,就是“不违背”,那位官员自然也是这样理解的,可是,孔子和他的同胞一样,也具有“拐弯抹角的天赋”。他并不亲自对孟懿作出解释,而是直到后来,他的弟子樊迟驾车送他时,才又重提这件事。樊迟听了,自然问他:“夫子,您是什么意思呢?”孔子就抓住这个机会,作出了如下解释:“生,事之以礼,死,葬之以礼,祭之以礼。”毫无疑问。孔子希望樊迟能将这话转述给孟懿,这样,孟懿就会理解“无违”的真正含义了。还有一次是回答“孝”意味着什么。孔子强调对父母要事之以礼,否则,只照顾他们的身体,就无异于把他们当成马、狗来看待了。引用上面那些,是想表明,中国人的孝顺观主要是应该依从父母的愿望,满足他们的需求。在中国,这是个古老的观念,孔子曾明确地说:“今之孝者,是谓能养。”这也说明他感到当时与古代已大不相同了,而他对古代则心往神驰,希望复古。夫子的这些言谈已过去好多世纪了。可他的教诲已深深地渗入到中国人的骨髓中。如果今天他仍活着的话,我们深信,他会更坚定地说:“今之孝者,是谓能养!”我们现在已经了解了中国人是如何看待孝顺与其他社会职责间的关系,可我们还不清楚中国人在现实中如何理解孝顺。随便挑十个未受过教育的中国人来问,怎样才算是“孝顺”?可能会有九个人回答:“不让父母生气。”父母生气是因为子女没有好好地服侍。说得简单些,还是应该“无违”,这是孔子的话,尽管他这样说时,包含着“特殊的意义”。 
  如果读者想知道有关的实例,就请看一看《二十四孝图》,它讲述的故事在中国可谓家喻户晓。其中讲到东汉的一位少年,六岁时随父亲去拜访一位朋友。他发现,那人家里的桔子特别好吃,于是,就像一般的中国人一样,偷偷地塞了两个桔子在袖筒里。但在他告辞鞠躬时,桔子掉了下来,气氛变得十分尴尬。可是,这位少年非常镇静,他马上跪在主人面前,说了两句令其名声留传千古的话:“家母喜欢吃桔子,我是拿给她的。”他的父亲是当时的一位高官,在西方人看来,这孩子不可能没其他机会为他的母亲弄到桔子,但在中国人眼里,他却成了典型的孝子,因为小小年纪就能够为母亲着想,不过,也或许是因为他反应敏捷,很快就能想出借口吧。晋代也有一位少年,因为父母没有蚊帐,就想出了一个绝妙的权宜之计,每天早早地上床,整夜静静地躺着,一动不动,甚至连扇子也不摇一下,为的是让家里的蚊子都来叮自己,好使父母能睡个安稳觉。与他同时代还有一个少年,在家里很不受继母的喜欢,可他的继母有个爱好,就是喜欢吃鲤鱼,但在冬天又弄不到。于是,这少年就不加思索地脱去衣服,躺在冰上。冰下的一对鲤鱼看到这情形,大受感动,就钻了个冰窟隆,跳了上来,以供他那暴戾的继母享用。 
  中国人认为,“偏袒妻儿”是一种不孝之举。《孝经》敕令章中曾把它与赌博并列。《二十四孝图》中有一个典型的例子。一位汉朝人,家中很穷,没有足够的粮食来养活老母和一个年仅三岁的儿子。他就对妻子说:“我们太穷,甚至连母亲都养不起。但孩子会争母亲的口粮。为什么不把孩子埋了呢?孩子埋了,咱们以后可以再生;母亲死了,就不能再有了。”妻子不敢反对,就挖了个两尺多深的坑,可在坑底,他们挖出了一坛金子。坛子上刻着一些字,说这些金子是上天赐给这位孝子的。假如没挖到金子,孩子可能就真被活埋了。按照一般人的孝顺观念,这人的行为可以理解,做法也正确、“偏袒妻儿”的感情不应阻止他活埋儿子以使其祖母活下去。 
  中国人还相信,父母的痼疾、只要吃了子女的肉,就有可能治愈。这些肉应该做好后、让父母无意中吃下。即使不敢肯定会治愈,中国人认为总有可能。北京《邸报》上经常出现这类事例。笔者认识一个年轻人,为了给父母治病,就曾经从自己的腿上割下了一块肉。对那块伤疤,他一直十分自豪,就像个老兵一样。毫无疑问,这类事情并不普遍,不过也许并不罕见。 
  中国人的孝顺中,最重要的方面是孟子说的:“不孝有三,无后为大。”需要有后,是因为需要人继承香火,祭把祖先。这已成了生活中最重要的内容。同样因为这一点,中国的男孩子必须尽早成婚。三十六岁做祖父,在中国司空见惯。笔者的一位熟人,在弥留之际,曾责备自己有两不孝:一是不能亲自为老母亲料理后事;二是没安排好儿子的婚事。他的儿子当时只有十岁左右。这种想法,无疑会为大部分中国人所接受。 
  中国人休妻一般有七种理由,第一种就是不生男孩。对男孩的渴求,导致了纳妾制度。也随之产生了各种不幸。他们生男孩时就兴高采烈、趾高气扬;生了女孩,就神情沮丧、意气消沉。大部分的溺婴事件也与此有关,这种事南方比北方多。有时,人们甚至根本就不知道。想获得这方面的信息极为困难,因为人们对此讳莫如深。中国的私生子也不少,但无论男孩女孩,人们都不希望把他们留在世上。即使不能直接证明各地溺杀女婴的事件比实际上要少,但从道理上肯定活埋三岁小孩以便养活其祖母的行为,无论如何都不能逃脱杀人的罪责,即使是不受欢迎的女孩。 
  中国人守孝的观念,上文已作了阐述,原来要求应满三年,可实际上已缩短为二十七个月。在《论语》第一十七章 ,夫子的一个门徒就坚决反对守孝三年,坚持说一年就足够了。对此,夫子最后说,在三年守孝期间,君子不能行乐,但如果你把它缩短为一年,只要行乐时能心安,就行乐好了。可是,夫子明确评价他“不仁”。 
  守孝比一切社会职责都重要,作儿子的,为政府服役时除外,一生要为此付出很多时间。也有一些特别的孝子,会在父亲或母亲的坟前搭个棚,整天住在那儿。最平常的做法是夜晚住在那儿,白天照常生活。也有一些人情守礼仪,完全沉浸在悲痛中,什么事也不做。笔者也认识这样一个人,他对父母极尽孝道,在父母坟前守了很长时间,仍然心绪不宁,给全家带来了一个不必要的负担。但中国人对此极为赞赏,完全不考虑后果。履行仪式是绝对的,其他任何事情都无关紧要。 
  好多人为了给父亲或母亲置办体面的葬礼,卖掉了最后一块田,甚至扒屋卖棒。这种行为是一种社会性的错误,但又很不容易让中国人明白。它符合中国人的天性,也符合礼,所以,他们觉得必须去做。 
  中国人极重视札仪和孝行,胡克神父依据自己的亲自经历,为我们提供了一个精彩的例子,那时,他来到中国,尚不足一年,住在南方某地。他雇用了一位家在北京的教师,教师家中有一位老母亲,母子已四年未通音信。有一次,神父要派一个信差到北京去,考虑到这是一次难得的机会,就让教师也写封信回家,听说信差马上要走,这位教师就从隔壁教室叫了一个学生,对他说:“过来,拿着纸,替我给我母亲写封信,别耽误时间,信差马上要走了。”胡克先生十分惊讶,就问那孩子是不是认识老师的母亲,结果是他根本就不知道还有这样一个人,“你没告诉过他,他知道写些什么呢?”老师不以为然地说:“他不知道该写些什么?他学作文已有一年多了,掌握了不少文雅的辞令,你认为,他不清楚儿子该怎样给母亲写信吗?”孩子很快把信写好了,而且还封了口,老师只是签了名。这封信可以送给帝国的任何一位母亲,她们收到信时,也都会同样满心欢喜。 
  由于孝行对孩子的影响不同,就导致了两种情况。当然,两种极端的例子在哪儿都能找到。杀死父母的现象并不多见,这种人一般都是疯子,但对他的处罚与常人没什么不同。普遍百姓,终日在穷困潦倒中痛苦地呻吟,父母对子女过分苛酷,有时在所难免,所以才会有这种事情发生。另一方面,主动代父接受死刑的事也时有发生,它有力地证明了孝心的真诚与力量。尽管做父亲的可能罪该万死。 
  西方基督教国家的家庭关系纽带十分松弛,对刚从这种纽带中解脱出来的西方人来说,中国的孝行的确有些吸引力。尊敬长者的品质就对盎格鲁一撒克逊民族特别有益。在西方,儿子长大后,想去哪儿就去哪儿,愿做什么就做什么。在中国人眼里,这有点像长大了的牛犊或驴驹,因为只有动物才不受礼的约束。站在中国人的立场思考一些问题,就会发现,我们还有许多社会行为需要改进,我们大多数人就像生活在玻璃房子中一样,确实应该小心谨慎,不能乱扔石子。不过,不重点强调一下孝顺的几个致命缺陷,一切讨论都将徒劳无功。 
  中国人的孝顺观念有五大缺陷,两个已经讨论过了,还有三个未讨论。第一是它对作儿女的,列举了一大堆义务,可是对父母的义务,却只字不提。在中国,提这类建议是多余的。而在世界其他各地,它一直都是必不可少的。神启的智慧曾引导使徒保罗,使他以精炼的语言道出了理想家庭的四大支柱:“你们作丈夫的,要爱你们的妻子,不可苦待她们。”“你们作妻子的,当顺从你们的丈夫,这在主里面是相宜的。”“你们作女儿的,要凡事听从父母,因为这是主所喜悦的。”“你们作父亲的,不要惹儿女的气,恐怕他们失了志气。”孔子道德思想中的那些世俗的智慧怎么能与这些意义深远的准则相比呢?所有的教义都不为女儿说话,全都为了儿子。在这方面,多少世纪以来,如果中国人不是色盲的话,怎么会没发现这是对人性的严重摧残呢?生为男身,就被家里奉为至宝,生为女身,则成了家中可怕的累赘,就算不一定被溺死,也一定会终生饱受歧视。 
  中国人认为,妻子是卑贱的。孔子没有说过丈夫应该对妻子如何,或妻子应该怎样对待丈夫。儒教只是要求男人应该依从父母,同时也强迫妻子这样做。妻子与父母产生矛盾时,因为妻子次要、卑贱,她就应该让步、屈服。中国家长制的社会结构存在着严重的弊病。它压抑人的某些天性,但又将另一些天性训化至极端;它使整个社会成了老年人的社会,青年一代则倍受压抑,处在从属地位。钢铁般的压力禁锢了人的思想,阻碍了社会的发展利有益的变革。 
  孝道中传种接代的宗旨是一系列弊病的根源。它要求,无论有没有养活孩子的条件,都必须生养。它导致了早婚与人口泛滥,使人们倍受贫困的煎熬。它也是一夫多妻制和纳妾制的根源,它永远是一个祸根。祖先崇拜真正是中华民族宗教信仰的集中体现。如果正确地理解的话,它是一个民族被迫套上的最沉重的苦轭。正如那茨博士在上面的那篇论文中指出的:令人恼火的是,数亿中国人都受无数死人的支配,“活着的一代受过去无数代人的控制。”对于令人窒息的保守主义来说,祖先崇拜是最好的形式与保证。如果保守主义不受到道德上的打击,在本世纪的最后十五年,中国如何能够使自己完全适应新的形势呢?如果中国人继续把过去的死人当做真正的神灵,他们如何能够向前迈出切实的一步呢? 
  我们认为,中国人的孝顺完全是由恐惧和自私造成的,这二者是左右人灵魂的最有力的因素。鬼魂因为具有制造灾难的力量,故而受到崇拜。孔子有一句富有智慧的格言:“敬鬼神而远之,可谓知矣。”忽略了供品,鬼魂就会发怒,接着就要报复。崇拜它们就是一种较保险的方式,这似乎就是各种崇拜死人观念的核心。活人之间,推理也同样简单。儿子孝顺老子,也要求自己的儿子尽孝,这就是养孩子的目的。“种树遮荫,养儿防老。”无论是老子,还是儿子,都很清楚这一点。“没有尿床的孩子,就没人坟前烧纸。”每一代都要偿还上一代的养育债,也要求下一代最大限度地偿还自己。因此,孝顺的品行就年复一年、代复一代地传了下来。 
  对于中国人过分夸张的孝顺,有一种忧郁的观点认为,中国人既没有把崇拜对象具体化为上帝,也没有能够认识到上帝的存在。祖先崇拜是孝顺最完美的,最终的表现形式,它纯粹是由泛神论、不可知论和无神论构成的。它把死人变成神,神也不过是死人而已。他们只对父母表示爱、感恩和畏惧,他们不知道天上的父,就是知道了,也毫无兴趣。中国人要么接受基督教,要么放弃祖先崇拜,二者不能共存。在这二者生死斗争中,适者生存。



\chapter{仁爱之心}
“仁”向来被中国人列为“五常”之首。“仁”字在汉字中由“人”和“二”两部分组成,可能是想表明,仁产生于两个人的相互交往。对文字本身的意义,我们没必要深究,因为它并不能代表生活——聪明的观察家应该了解的是现实生活。不过,尽管有一些本该了解真相的人常常作出浅薄的论断,认为中国人不具备仁慈的品质,这绝不是事实。孟子曰:“恻隐之心,人皆有之。”儒教教人温良,佛教劝人慈悲,这不能不对中国人产生显著的影响。更何况,中国人有强烈注重实际的天性。他们一旦要“行善”时,肯定会找到大量行善的机会,并做出各种“善事”来。 
  中国人引以为荣的慈善行为有设立保育堂,建立麻疯病院、老人收容院和免费学校等。因为中国缺乏实用的户口统计,现在可能还不知道这类机构究竟有多少。戴维.希尔牧师曾调查过中国中部的一些地方,发现杭州城有三十家慈善机构,每年的开支大约为八千英镑。但是,冒昧地说,这些慈善机构仍然相对不足,因为中国人口众多,尤其是大闹市区人口密集,他们需要大量慈善机构。 
  中国发大水或闹饥荒时,各地普遍设立施粥棚,也为穷人捐赠衣物。这些事也不全由政府来做,民众自己也互相帮助,共度难关。这类耗资巨大的事例并不罕见。灾荒年头,逃荒的灾民潮水般地涌进城市,相当必要时,他们被允许在车棚里、空房子里住下来。因为假如这些成群结队的灾民遭到拒绝,他们就会采取行动,实行报复。这时,让步是最明智的做法。 
  另外,各省在外地设立的同乡会也属于慈善机构。它主要照顾离乡在外,穷困潦倒的人,或者客死他乡、遗体无法运回家乡的人。这是一种保险性质的日常性事务机构,中国人大概也这样认为。 
  在一些劝人行“善”的书中。有的人对自己做过的恶事直言不讳,引以为耻,也宣扬自己做过的善事,并引以为荣,善恶的结果会在判官的生死簿上显示出来,并决定着他们的来世。这种简单的报应观念清楚地反映了中国人注重实际的天性,就像我们已经讨论过的,他们总是执着地为来世考虑。在他们眼里,来世不过是现世的伸展与延续。大部分中国人乐于行善的目的是期望获得回报。有时,公开善行背后利己的动机,会带来不可思议的后果。1889年4月,杭州的官吏为帮助因黄河泛滥而受灾的难民,试图通过对城中茶馆卖出的每杯茶水抽税来筹集资金。但古都的民众对这一做法的态度就像1773年波士顿市民对茶税的态度一样。官吏贴出告示:“行此无上善举,必得善报。”他们想以此来赢得民众的支持。可是,民众与茶馆联合起来,进行抵制,终于使这一计划彻底破产。满城居民如此团结一致,共同抵制强制人获得的“善报”,对我们来说,确实罕见。 
  为穷人提供棺材;把暴露野外的人骨头收集起来并重新埋葬;烧掉捡到的字纸,以免它们遭亵渎;买活鱼、活鸟,把它们放回大自然;还有些地方,为需要者赠送神秘的膏药,免费种痘,低价出售或赠送劝世良言,这些都成了中国人行善的主要内容。因此,也正如我们所看到的一样,真正对人怀有善良意愿的行为就退居其次了。而这些陈旧的做法又几乎如出一辙,千篇一律,做的人也极少动感情,动头脑。站在岸边,看渔人撒网、捡鱼,撤网、捡鱼,当然比帮助站在家门口的乞丐容易多了。 
  况且,对注重实际的中国人来说,有一点是十分重要的,那就是鱼一人水,鸟一出笼,它们就自谋生路去吧,他们应做的已经做完了。鸟儿或鱼儿们不能指望放回它们的人会为它们提供更好的生活条件。对人来说,他们只是在积德,在做自己的事,至于鱼或鸟以后的命运,他们可管不了。 
  在中国,“善门难开”,关上更难。没有谁能预料到愿望良好的行为在将来会有什么样的结果,也没人知道因此而招来承担更多责任的危险。明智的做法就是对自己的行为时刻谨慎。一个住在中国内地的传教士,曾应当地一些绅士的请求,帮助一个双目失明的乞丐,为他治眼,其实,不过是小小的白内障而已。后来,乞丐的眼痊愈了,他重获光明。然而,那些绅士知道之后,却说传教士砸了乞丐的饭碗,因为,他现在不能再讨饭了。因此,传教士应该养活他,雇他看门。有时,一个很少与其他人交往的慈善的老太太,款待其他的老太婆——她们看起来似乎应该得到周济,但是她却会成为这些人残酷榨取的牺牲品。我们曾听说过这类事情,虽然只有一例,但估计并不罕见。我们不能不承认,中国人很少有发自内心的仁慈,哪怕是那么一丁点儿也没有。 
  瘟疫、饥荒爆发或黄河大决口时,地方政府或中央政府迟早总会派人到灾区,试图帮助灾民。不过,他们从不采取长久性的、大规模的防范措施,仅仅是采取一些权宜之计,似乎这种事只会发生一次。对灾民的帮助也经常在关键时刻偏偏中止了。比如说,人们经过长期痛苦的煎熬,好不容易挺到了早春,这是个最容易发病的季节,可是政府只给一点儿救济就把他们打发了,要他们赶快回家,老实干活。理由不用说,谁都知道:政府的钱用完了,田里还正需要人干活。麦收前,只要他们有吃的,就足够了。政府也很清楚,如果不给一点救济,天气转暖,瘟疫就可能爆发,人们大批地死去要比小灾难更令人注意,更易引起麻烦。 
  “腊八舍粥”也一样,是典型中国式的慈善活动,它也只注重活动的表面形式。腊月初八这天,平时没机会行善的人,早已准备好要慷慨施舍。按照风俗,他们一整天会向所有来讨粥的人施舍,不过,这些粥都是最便宜、最难让人下咽的。这就是所谓的“行善”,人们以此来积德。如果某一年碰巧丰收,可能就没有人来讨粥了,因为即使穷光蛋在家里也可以吃到同样或更好的饭食。即便如此,仍不足以使施粥者停止舍粥,或换上更好的食物。一天过去了,没有一个人来讨粥,它们最终被倒进了猪槽。而行善的富人们也带着悠悠的满足感回屋睡觉去了。今年的义务他己尽了,良心也得到了满足,他是个仁慈的人。但假如遇到了坏年头,米价暴涨,他们就没心思行善积德了,因为他们“行不起善”。 
  前面,我们说过对乞丐的施舍,在中国,成群的乞丐随处可见。他们所得到的施舍有点保险的性质。众所周知,城里的乞丐常常组成强大的帮派,他们远比与自己争斗的任何帮派都更强大,因为他们一无所失,也无所畏惧,这可是无与伦比的优势。如果一个小店主拒绝了一个乞丐的乞讨——他会像日内瓦仲裁那样镌而不舍,就会有成群的乞丐前来骚扰。就连一个精神麻木的中国人也会感到这是个沉重的负担,乞丐们要等到自己不断升级加码的要求完全满足之后,才让店主继续做生意。店主和乞丐对拒绝的结果都很清楚,因而使得这类善行就像涓涓细流,绵延不绝。 
  对经常可以看到的,川流不息的难民,人们也同样对待。通过这些,你将认识,这不仅仅是使难民受惠,更重要的是行善者以为自己因此可以获得福报,中国人施惠的每一个对象,都可被看成“小情人”,行善者的一切行为目的只是使自己在现在或未来生活得更好些。 
  对于中国人这种扭曲的慈善行为,应该再加上重要的一点,即无论何种事情,好事也罢,恶事也罢,都不能逃脱日益萎缩衰退的中国政体的压榨,而且这种压榨和政府的其他计划一样组织严密。想知道一个中国人把赈济款据为己有的全部细节,简直比登天还难。不过,在一些紧急关头,如大饥荒中,可以充分肯定,即使民众的深重苦难也不能阻止元耻的官吏侵吞手中的赈济款。此时,人们的注意力都集中在民众的苦难及赈济款上,如果外界既不知道款子的筹集情况,也不知道其使用情况,结果就可想而知了。 
  当中国人开始更多地了解西方文明的时候,他们所了解的只是西方人强迫他们接受的西方文明中最坏的成分。在他们看来,基督教世界遍布非基督教世界无法比拟的慈善机构,这肯定是件了不起的事。这也可能会促使他们去探求隐藏在这一意味深长的事实背后的东西。我们还应该提醒中国人去注意一个令人深思的细节:表示“仁”的汉字与其他和感情有关的汉字不同,它没有以心字作偏旁,这说明,它代表的美德通常是缺少诚意的,其结果,我们已经知道了。慈善活动应是一种本能,无论有无明确的必要,都要找机会表现出来。中国人完全缺乏这种精神,这的确不是人类的进步。如果中国人想创造出真正的慈善,就必须经历西方人过去的经历,把仁慈变成人生的重要成分。 


\chapter{缺乏同情}
我们已经考察了中国人的慈善活动。仁慈是一种善良的天性,同情也建立在它的基础上,我们姑且认为中国人的确做了些慈善事业,下面所要阐明的是中国人明显缺乏同情。 
  我们要时刻牢记,中国人口众多,各地会定期发大水或闹饥荒。很多国家的事实都表明,社会条件是控制人口增长的重要因素,但在中国,似乎不怎么灵验。传种接代是中国人的首要愿望。最穷的人家也要在儿子很小时就给他们娶媳妇,随后这些孩子又生出一大堆孩子,就好像他们生活有保障一样。还由于一些其他原因,结果使得中国人的生活简直就是干活,吃饭,吃饭,干活,几乎就像一个短工,这已经难以避免。如果一个外国人不能马上意识到,几乎所有的中国人都缺钱,他就不可能长期与中国人相处。事情一开始做,他们就要钱,因为他们一无所有,给了钱,做事的人才有饭吃。即使是小康人家,急需用钱的时候,也很难筹集到起码的数目。中国有个意味深长的说法,用以形容办丧事、打官司时被迫借钱的窘状:“过贱年”,就是说好像一个饥饿的人,不顾一切地寻求帮助。除了境况较好的人家外,谁都不可以指望能在孤立无援的情况下,独立操办这类事情。令人绝望的贫穷是帝国最突出的现实,它使得人与人之间明显变得冷漠。在物质困乏的压力下,人们已形成一些固定的习惯,即使是直接的生活需求不再紧迫时,他们仍保持艰苦的生活水平。中国的生活就像一个椭园,钱和粮是它的两个圆心,一切社会生活都围绕着它们旋转。 
  帝国民众的极度贫困、他们为生活所需而进行的长期艰苦的抗争,以及在各种难以想像的条件下所遭受的令人同情的苦难,都是世人皆知的。中国人的慈善行为无论是出于何种动机,也都只不过是想从令人绝望的痛苦中解脱出来,哪怕是千分之一那么一点点。这些苦难一直沉重地压迫着他们,要是遇到灾荒年头,还不知要糟多少倍呢!中国的有识之士应该意识到他们那些缓和痛苦的办法是彻底行不通的。无论是靠个人的慈悲,还是靠政府的干预,即使做得再好,也只能改善表面的症状,对于根除疾病完全无效。就像发冰块给伤寒病人一样——每个人就这么多,没有医院,没有饮食,没有药物,没有护理。因此,一点也不奇怪,中国人没有变得更慈善,而是在全然缺乏制度、预见和管理的情况下,一直保持行善的习惯。我们都清楚,即使一个有教养的人,长期面对既无法阻止又无力帮助解决的灾难,会产生什么样的结果。现代的战争就是一个明证。第一次看见血,会精神紧张,产生难以消除的印象,但它很快就消失了,人也变得麻木了。对有经验的人来说,对血的恐惧一生只有一次。中国经常发生战争,人们对战争的结果也早已习以为常。 
  对残疾人的态度也能说明中国人缺乏同情;中国人一般认为,呆子、瞎子、尤其是独眼龙、聋子、秃子、斗鸡眼都应该避而远之。似乎生理上有缺陷,道德上也一定有缺陷。据我们观察,人们不会对这些人冷酷无情,但总是缺少同情。就像古犹太人认为的,这些人肯定暗中犯了罪,因此才遭到这样的惩罚。相反,西方人会对这种人产生发自内心的同情。 
  一个不幸残疾的人,无论是先天的还是后天的,不能忍耐嘲讽就不能活下去。对他最温和的方式是描述他的缺陷,以引起众人的注意,药铺的伙计会对一个病人说:“麻子老兄,你是哪村的?”一个斜眼人听到“眼斜心歪”也不足为奇,假如是个秃子,就会听到:“十个秃子有九个是骗子,最后一个如不哑巴,也一样。”这些不幸的人终生都必须逆来顺受,只有当他听到长年不断的嘲弄而不再温怒时,才能够安于生活。 
  对精神有问题的人,中国人同样坦率得过分。旁观者会说:“这孩于是个笨蛋!”可实际上,他也许并不笨。不断地重复说他不长脑子,很容易摧残他未发育完全的智力。以这种方式对待精神病患者或其他病人,也十分普遍。也许恰恰就是这种方式导致了疾病的产生,并使之更为严重。他们所有的毛病、生活的细节成了公众的谈资,而他们自己所能做的只是完全习惯于被称为“疯子”。“二百五”、“蠢货”等等。 
  在一个重视生男孩的民族中,因没有孩子而遭到谴责与嘲骂,一点儿也不奇怪。就像传说中先知撒母耳*的母亲,“为了激怒她,仇敌触动了她的痛处。”不管有意无意,一个母亲悄悄地闷死了她的一个孩子,人们对此并不大惊小怪,那一定是个女孩。 
  婚礼中新娘的遭遇也是中国人缺乏同情的典型例证。新娘一般都很年幼,也很害羞胆怯,突然置身于那么多陌生人当中,难免感到恐惧。尽管各地风俗差别很大,但都任凭众人盯着这些可怜的孩子,完全漠视她们此刻的心情。有的地方,人们可以随意拉开轿帘,盯着新娘看;还有的地方,新娘会成为尚未出嫁的姑娘们取乐的对象。她们站在新娘经过的道旁,大把大把地向她头上撒草籽或谷糠,新娘的头发是费了好长时间,仔细油过的,那些东西会牢牢地粘在上面。在公婆门前,新娘一下轿子,就立刻成了人们品评的对象,仿佛一匹刚买来的马,此时此刻, 
  *注:撒母耳,《圣经》中希伯莱的士师与先知。新娘的心情当然是不难想像的。 
  中国人一方面特别注意细节,另一方面又会做出对别人显然不合时宜的事。我的一位中国朋友,就说过一些失礼的话,可他一点儿都没觉察到。他描述他第一次见到外国人时,说他感到最惊奇的是他们脸上长满了胡子,像猴子一样,然后他还再三保证说:“我现在已经非常习惯了。”老师则会当着学生的面评价学生:靠门的那个最聪明,二十岁时一定会高中,而邻桌的那两个的确是他所见学生中最愚蠢的,这种评价会对学生产生何种影响,从来没人想过。 
  中国人缺乏同情还表现在他们的大家庭生活方面。尽管各家情况不同,我们仍然可以轻而易举地发现,他们的家庭生活并不幸福。他们也不可能幸福,因为缺少感情上的结合,而这一点在我们的现实生活中恰恰是至关重要的。中国人的家庭只是个人组成的团体而已,他们持久稳定地结合在一起,有共同的利益,也有不同的利益。这种家庭在我们看来根本不是家庭,因为家庭成员之间没有同情心。 
  在中国,女孩一出生,多多少少总不受欢迎。她们的遭遇中有大量有意义的事例,可以说明中国人缺乏同情。 
  在中国,母亲和女儿共同住在封闭狭窄的小院子里,难免会发生争吵,由于平时很少受到约束,她们便往往恶语相加。中国俗话说:“再骂总是亲闺女。”对于想了解中国家庭的人,这句话确实很有意味。女儿一旦结婚,除血缘关系,就与娘家没多少关系了,将她的名字从家谱中抹去,是出于一种根深蒂固的观念:她不再是我们家的女儿了,而是别人的媳妇了。但人的天性又促使女儿隔三差五回娘家走亲戚,这也是地方风俗,某些地方,女儿经常回娘家,而且住的时间很长;而另外一些地方,女儿则应尽量少回娘家,如果娘家人全死了,她就几乎再也不回去了。不管这些风俗有多少细微的差异,人们普遍认为,媳妇是婆家人。女儿回娘家,严格说来,是出于一种做活的考虑。她们常常带上婆家的一大堆针线活,而娘家的人必须帮她做完,每次还要尽量带上自己的孩子,这样,既可以避免自己不在时没人照看,最重要的是孩子能在姥姥家吃喝花销。对于女儿较多的家庭,频繁的造访会令全家人感到很可怕,简直是一种严重的盘剥。因此,父亲与兄弟常常阻挡女儿回来,母亲却暗中支持。但根据当地风俗,如正月里的某些日于,尤其还有节日,女儿回娘家是不能限制的。 
  女儿回婆家时,就像谚语中讲的贼,从未空手而归。她应给婆婆带些礼物,一般是些吃的。假如忽略了这一点,或者没能办到,婆婆就会演戏似地发一通脾气,女儿嫁到穷人家里,或者后来家道衰落了,假如她有一些结了婚的兄弟,她将会发现,回娘家就仿佛医生说的,“处置不当”。娘家的媳妇和已出嫁的女儿之间就会爆发战争,就像非利士人和以色列人一样,都把家看成自己的领地,把对方看成入侵者。如果媳妇在家里足够强大,她们就会像非利士人那样,对不能统统消失或赶走的仇敌索取贡品。媳妇在整个家庭中的地位,严格地说,形同奴仆。找仆人,当然要找健壮的,发育良好的,而且还要懂得烹调、缝纫等生活技艺,不论当地人如何谋生,她们总比没有力气和办事能力的孩子要强得多。因此,我们就明白了,为什么一个十岁左右瘦弱的男孩会要一个健壮丰满。二十岁的姑娘作媳妇了。婚后很长一段时间,姑娘还要尽心尽力照看生天花的小丈夫,天花是一种幼儿病。 
  中国媳妇的苦难简直罄竹难书。中国妇女一般结婚很早,她们一生中相当一部分时间是受婆婆的绝对控制,由此,人们大概可以想像出媳妇在倍受虐待的家庭中,遭受了多么令人难以忍受的痛苦。做父母的,在女儿遭受虐待时.只能对她的婆家表示抗议,或在女儿受虐自杀后,索取高昂的送葬费,除此之外,他们完全保护不了自己的女儿。如果丈夫严重伤害、甚至杀死了妻子,只要说她对公婆“不孝”,就可以逍遥法外了,我们有必要重复一遍,年纪轻轻的媳妇自杀,在中国司空见惯,有些地方,各村都会接二连三发生这类事。一位母亲曾责备已出嫁的女儿自杀未遂:“你有机会,怎么会死不成?”痛哉斯言! 
  在几年前的北京《邸报》上,河南总督偶然披露了一种情况:不仅父母杀死孩子不需要负法律责任,而且作婆婆的杀死媳妇只需交一笔罚金就行了。在报告的案例中,有一位妇女用香柱烧她的童养媳,用烧红的火钳烙她的双颊,最后又用滚烫的开水把她烹死。这位总督的奏折里还提到了其他类似的例子,其可靠性是勿容置疑的。这类极端野蛮的行径大概并不多,不过,残酷的虐待导致自杀或企图自杀却是常见的。 
  中国有许多妇女嫁给人作妾,她们的生活也十分痛苦。她们生活的家庭,极少是幸福的,总是不断发幸争吵和公开的打斗。一位在中国住了很久的外国人写道:“我所居住的那个城市的长官,是个大富翁、大学者,诗人,也很有才干,通晓经典教义;但他任意欺骗、诅咒、搜刮和体罚百姓,以满足自己罪恶的欲念。他的一个妾逃跑了,抓到后,被剥光了衣服倒吊在梁上,严刑拷打。” 
  在中国这样的国家,穷人可不能生病。女人、孩子病了,男人根本不把它当做一回事,任其发展,到最后常常都是病人膏盲,因为男人没时间照料他们,有时是因为“付不起医药费”。 
  我们前面讨论的孝顺观念把年轻人看得无足轻重。他们的价值只在未来,而不是现是。西方的许多做法在中国常常是被反其道而行之。三个旅行者当中,最年轻的要吃苦在前。最年轻的仆人也一律最辛苦。百姓的生活穷困难熬,孩子们经常会因苛刻的压制而离家出逃。在外头,他们一般都能发现生存的希望,因为可以与别人合伙谋生。。出逃的原因多种多样,但据观察,最普遍是因为不堪虐待。我知道一个男孩,最近斑疹伤寒初愈,很想吃东西——这种病人一般都是这样。他觉得家里的粗糙的黑窝头实在难以下咽,就跑到街上,非常奢侈地买了大约两毛钱的点心吃了,但因此受到父亲的严厉责骂,于是一气之下,跑到东北去了,从此,杳无消息。 
  乔治.D.普林蒂斯说,男人是支配者,女人只不过是“细枝末节”。这话用来形容妻子在中国家庭中的地位,非常恰当。人们认为,婚姻对女方家庭是为了不再抚养她,摆脱一个负担,对男方家庭则是为了传种接代。除非深究潜在的动机,人们对此都是闭口不谈。但是在中国,没有谁对此心里不清楚。 
  婚姻的这一目的,在较穷的阶层表现得更突出。寡妇再嫁,人们会说:“现在她不会饿死了。”俗话说:再嫁再娶,为了肚皮;没吃没喝,拆灶散伙。灾荒年头,丈夫抛弃妻儿,任其乞讨或饿死,己是司空见惯,有很多家庭把儿媳妇赶回娘家,由娘家赡养,或最终饿死。他们说:“你们的女儿,你们自己养活吧。”有时,发给哺育婴儿和妇女的特殊救济粮,会被男人吞吃,尽管这种事可能并不多,可总在发生。 
  仅仅通过灾荒年头的现象评价一个民族,显然有欠公允,然而,重要的是,特殊的岁月常常是检验社会基本原则的试金石,和平时相比,可能会更准确,更确实,在中国,卖妻卖儿,并不只发生在灾荒年头。只不过,这时人似乎忘记是在从事人口交易。了解真情的人都知道,早几年,很多灾区,买卖妇女儿童就像买卖牲口一样公开,唯一的区别就是前者不用赶到集市上去。1878年,大灾荒几乎席卷了整个东三省,并向南蔓延,买卖妇女随处可见,十分普遍。大量的妇女被运往内地。有的地方,运输都出现了困难,甚至连一辆马车都雇不到。人贩子千方百计转运刚买到的妇女,把年轻的从灾区或人口过剩的地方运往因造反而人口减少、或多年娶妻困难的地区。令人感到悲哀的是,这一奇怪的交易对买卖双方可能都是最好的出路。尽管卖方妻离子散,天各一方,但买者与卖者毕竟都能活下去。 
  我们说过,中国人之所以对病人熟视无睹,是因为他们“只不过是女人和孩子”。天花,在西方被当成可怕的灾祸,可中国人对它一点儿也不重视——尽管在中国经常有人染上这种病,而且几乎无人能逃,这也只因为害天花的主要是孩子。因害这种病而双目失明的人十分普遍。中国人对婴儿生命价值的忽视程度,令西方人难以想像。他们强烈反对毁坏人的尸体,但对婴儿的尸体经常不加掩埋。婴儿死了,人们都是说:“扔掉”,用芦席松松地卷了,抛到荒野里,不久就被野狗吃掉了。有的地方,还流行一种恐怖的习俗,把婴儿塞进乱坟岗的死人堆,以免“鬼魂”回家骚扰。 
  我们感到天花可怕,中国人却不在乎。可他们对斑疹伤寒与伤寒的恐惧,如同我们见了猩红热一般。一个人离家在外,得了上述其中任一种病,都难以得到妥善的护理,甚至一点护理也得不到。向其他人请求帮助,得到的回答肯定是:“那病传染。”尽管伤寒多少有些传染,可在云南的一些山沟里,它可能是最令人胆寒的灾难。巴伯先生描述说:“患者不久变得虚弱不堪,接着一连几小时,浑身疼痛难忍;随后神志不清,胡言乱语,患了这种病,十之八九,性命不保。”据当地人说:“病人房间的各个角落都被鬼占据了,桌子和床在里面四处移动,发出声音,清楚地回答人们的提问。”可是,很少有人冒险进屋。传教士向我证实,大多数情况下,由于害怕传染,人们像对待麻疯病人一样将病人丢弃不问。如果家里的老人患了这种病,最好的照顾就是把他挪进一间孤零零的小屋子,放上一碗水,锁上门。挂念他的亲人每天两次战战兢兢地从门缝往里看,用棍子捅捅病人,看他是不是还活着。 
  中国人性情温和,在这样一个民族,每个家庭中肯定存在友爱行为,不过我们没有发现而已。疾病与灾难尤其容易唤起人类天性中最美好的一面。在一家为中国人开的西方医院里,我们耳闻目睹了很多实例,不仅父母与子女,丈夫与妻子真诚相爱,就连陌生人之间也彼此爱护。一位中国母亲见到失去母亲的婴儿,很愿意用自己的乳汁喂养他,因为不忍心看着他饿死。 
  除非有特殊的原因,没人愿意帮助别人,这是中国多重社会关系表现出来的一个特点。比如,一个聪慧的男孩,经过考虑,想去读书,即使他没机会入学堂,这也非常合乎清理。可他周围的很多读书人,宁愿闲着无事,也不愿教他识字。他一流露出读书的愿望,就会招来无穷的嘲讽,这些人曾经年累月待在学堂里,他们似乎认为:“这家伙凭什么走捷径,我们费了好多年时间辛辛苦苦学来的东西,怎么能教他,让他很快学会呢?还是让他和我们一样请老师吧。”尽管个别人可以自学,但是很少有人能真正学到知识,哪怕是最基本的识字也不可能。 
  见人落水,竟袖手旁观,所有在中国的西方人都对此大为震惊。几年前,一艘外国汽轮在扬子江上着火,岸上挤满了观望的中国人,但没人营救落水的乘客与船员。最后,那些拼命游到岸边的人,很多都被抢劫一空,甚至身上的衣服也被剥去了,还有一些人被当场杀死。不久前,英国也曾发生沉船事件,但没有出现不营救的现象,我们应该将这些事比较来看。1892年秋天,英国一艘庞大的汽船在中国海岸搁浅,当地渔民和政府官员都尽全力救助幸存者。不过,中国人对灾难麻木不仁,这是个普遍的事实,尤其离家在外,俗话说:在家千日好,出门一时难。 
  在中国旅行,人们普遍发现,沿途的人对陌生人缺乏友善与帮助。夏天遇到暴雨,无法继续旅行时,需要前进的人会发现,这时天公和人在合伙捉弄他。即使你走的路通向泥潭,也没人会提醒你。你走入泥潭,与附近修路的人无关。我们说过,中国人不重视公路建设。所有的路在任何时候都布满了深坑,旅行者一旦陷进去就难以自拔。这时,周围立刻会聚满看热闹的人,他们都像一句成语所说的:“袖手旁观”。直到答应给钱,旁观者中才会有一位站出来,帮你一把。不仅如此,当地的居民还经常故意在难走的地方挖一个深坑,这样,陷进去的旅客不得不花钱请他帮忙。在这种情况下,一个人若不了解道路情况,最好不要听当地人的劝告,只管照直往前走,只要不能肯定所走的是一条绝路,就比接受他们的“帮助”好得多。 
  可是,我们还听说过,一家外国人搬到中国内地的一个城市,受到了人们的热诚欢迎,邻居甚至主动借家具给他们,直到他们把家具备齐。类似的事情无疑还有,不过,谁都明白,这只是例外。人们一般除了对新搬来的人感到好奇外,更多的是表示冷漠,就好像肥鹅注定会招来贪婪与阴沉的敌意,最终被拔光羽毛。还没听说过,外国人遇到天灾人祸,中国人自愿帮忙的先例,当然,也可能出现过。我们只听说,曾有一些海员尝试从天津到烟台、从广州到汕头作陆上旅游时,自始至终没人给过他们一碗饭,或留住一宿。 
  在中国,将客死他乡的人运回家,途中住店非常困难,一般是住不成,我们曾听说,一位死者的兄弟,因店主不让住店,不得不在街头过夜。请摆渡者将尸体运过河,也会被狠狠地敲一笔。我们还晓得,有些人为免引起怀疑,就把尸体层层包裹,再外扎草席,使它看起来像一包货物。据说,前几年的一个寒冬,山东维县的一家店主因为怕几个快要冻僵的旅客死在店里,拒绝他们住店。结果,这几位旅客都冻死街头。 
  中国人作恶犯罪,很少有人告发,部分原因是没钱告状,另外也不愿惹人注意,通奸案一般私下了结。插足者会遭到一大帮人的毒打,中国人相信“人多势众”。有时,这个人的腿会被打折,有时是胳膊,更多的情况是被用生石灰弄瞎双眼。笔者知道几个这方面的例子,这类事情一点儿也不罕见。有一位聪明的中国人,他不了解西方人的思想“方式,当他听到外国人抗议这种极为残酷的做法时,毫不掩饰他的惊讶,他说,这种处理方法在中国已是“非常宽容”的了,就像他自己,仅仅残废而已,否则,早被杀死了。 
  “为什么老是到我家吃饭?”作嫂子的会对小叔子这样说,他已离家多年,在外头干了见不得人的事,双眼被人用生石灰弄瞎了,“这儿没地方让你住,硬的,有刀;软的,有绳,你只配要这些!” 这是那位无法医治的盲人偶然告诉我的,如果有希望,他还想获得一丝光明;若是没希望,他暗示说,无论“硬的”,还是“软的”,都可以让他解除痛苦。我们很少听说过,这类暴行的受害者告官成功过。对他们不利的证据已经压倒了一切,而且官员们十之八九认为他们活该,罪有应得,甚至还应该加重惩罚。即便他打赢了官司,处境也不会有所改善,只会变得更糟。他的邻居会更加愤怒,那时,他连命也难保了。 
  中国人把人视为神圣的,但生活中很少重视人的价值、人的尊严。在中国,偷盗是最易惹人愤怒的罪恶之一。因为人口众多,而且经常濒临无法生存的境地,偷盗就被视为对社会的严重威胁,其危害仅次于谋杀。在一次救灾中,一位分发救济品的人,发现一位妇女像疯狗一样被锁在石磨上,她是个盗窃狂,早已精神错乱。如果一个人被发现是小偷,或因某种原因而被公众唾弃,他就可能在简单讯问后被公众处死,这和弗吉尼亚早些年治安维持会的做法没什么两样。有时用刀子刺死,更多是活埋。有人形象地称之为“吞金”,其实,这非常残酷。笔者认识四个人,曾差点被这样处死。有两例是已被捆上,有一例是坑已挖好,后来由于族人中一些长者的干预,才没有被活埋。另有一例,发生在笔者很熟悉的一个小村子里,一个年轻人偷东西,已经不可救药,人们也知道他神经不正常。他本家的一些人和他母亲“商量(!)”了一下,就在村口的小河上砸了个冰窟窿,把他捆紧,塞了进去。 
  太平天国起义闹得最凶的那段日子,到处都很紧张。一张生面孔,一旦有嫌疑,就会被抓起来,遭到严厉的盘查。若不能交待清楚,使抓他的人满意,马上就会遭殃。在离盘查点几百码远的地方,文告上写着将近二十年前发生的两件惨事。当时,官吏们发现,他们自己几乎无力执法,就发布了一个半官方的告示,让百姓捕捉所有的可疑人物。一次,村民们发现,一个人骑着马向村子里走来,不像是本省的。盘问中,那人怎么也说不清自己的来历,接着又发现他的包裹中塞满了珠宝,这显然是偷来的,村民们就把他捆起来,挖坑活埋了。这时,又看见一个人惊恐地从田野中跑过,有人猜测他可能是同伙,索性连他也一起埋了。有时,陌生人还被迫自己挖坑。在无法无天的时代,所有的人都会变得胆大妄为。一些老人回忆说,那时候,像这类事数不胜数。1877年,爆发了一场不可思议的剪辫运动,当时,大半个帝国都被白色恐怖所笼罩,许多有嫌疑的人都被活埋了。当然,特殊情况下,任何民族都会产生这样的恐怖时期,我们也不能太苛求中国人。 
  中国人缺乏同情,最突出的表现是残酷。他们一般认为中国的穆斯林比他们自己更残酷。尽管可能真的如此,但了解中国人的人,肯定都认为,对别人的痛苦漠然置之,世界上几乎没有任何文明国家能与中国相比。就拿孩子来说,在家里,他们几乎无拘无束;一旦开始上学,这个充满温情的天国就消失了。《三字经》是帝国最常用的启蒙教材,这本书中有句话,叫做:“教不严,师之情。”老师的性情与学生的天资都会影响老师对学生的态度。不过一般来说,都非常严厉。我们曾见过一个刚被老师惩罚过的学生,那情形就像在街头打了一架,头破血流。老师让他掌握写应试文章的秘诀,他没做到。老师发火,学生挨骂,更是常事。另外,不幸受罚的孩子还会遭母亲的毒打,一位平时拿孩子出气的母亲,遇到特别刺激时,更会残酷地对待自己的孩子。 
  中国人缺乏同情还表现在他们的刑法制度中,根据帝国的法典,很难判断哪些刑罚合法,哪些刑罚不合法,因为有一些不符合法令条文的做法会得到社会习俗的认可与支持,最能说明这一点的是打板子的数目,它们常常高出法律规定数目十倍,有的多达百倍。这里,我们没机会公正地评价中国人对囚犯惨无人道的严刑拷打。在像《中央王国》或者《胡克游记》这类有关于中国的优秀著作中,这样的事例不胜枚举,《胡克游记》的作者提到,他曾亲眼看见一批囚犯手被钉在囚车上,押往衙门,因为解差忘了带脚镣。囚犯没有钱来打通关节,平时就会受到蓄意的残酷折磨,中国人虽有“心肠”,却肯定没有“慈悲”,还有比这更有力的证据吗?几年前,上海的报纸报道了一个案子。两个老囚犯向一个新囚犯索取“孝敬费”,结果地方官员判他们重打两、三千大板,又用铁锤敲碎他们的脚踝骨。中国有谚语云:死不进地狱,活不进衙门。我们大概不会对此感到奇怪吧?* 
  既然上文中,那些出人意料的结论是从表面上可靠的 
  *韩因章(HANYINZHANG)先生,一位在美国学习法律的中国留学生,他曾在一家重要的宗教杂志上发表过一篇论文,论述中国法治。前面在讨论中国人“不紧不慢”时已经引用过这篇文章。该文认为中国人并不把自己的刑罚当成残酷的。可我们对此不敢苟同,不能忘了,他们是中国人,他们的法律、习俗也是中国人的。他们在个人权利方面不进行彻底改革,他们的刑罚也许永不会有任何实质性的改善。在道德力量有条件充分发挥作用之前,一定不能放弃物质力量。例证得出的。下面我们将引用1888年2月7日北京《邸报》译文中的一段: 
  “据云南总督报告,该省的一些农村,流行一种可怕的陋习:抓到偷粮食的人,要活活烧死。同时,还强迫他的亲人书面表示同意这种做法,并要亲自点火,以免日后归罪于他人。有时,只不过折断庄稼的一个枝茎。有的出于怨恨,仅凭莫须有的罪名,就把别人置于死地,乍一听,这种残酷的做法实在令人难以置信。它也曾助长了云南的叛乱。政府一直努力铲除这一陋习,至今仍未成功。” 
  福州附近的一个地区,还有强迫寡妇自杀殉夫的恶习。几年前,当地的中国报纸曾作过详细的描述。乡人先是逼迫寡妇自缢,然后焚烧尸体,并建造一座牌坊,以彰其节。政府不断努力阻止这一残酷的做法,除了个别地方一时奏效外,基本上徒劳无功。 
  中国需要的东西很多,政治家认为需要海军、陆军和兵工厂,友邦人士认为显然需要货币、铁路和科学指导,但若进一步分析帝国的境况,难道她最深切的需要不是多一些人类的同情心吗?她需要对孩子同情,尽管人类从前没发现它,可十八世纪以来,它已成为人类最宝贵的财富。她需要对妻子和母亲同情,这种同情十八世纪以来已经获得长足发展,并深入人心。她需要把人当做人来同情,懂得仁慈之情有如天国的甘霖,既降临于祝福者,也降临于被祝福者——只有它才使人类最接近于上帝,塞内加称这种神圣的情感为“智力的缺陷”,但基督教培育的仁慈之花,要一直等到开满全世界才会停止。 

\chapter{社会风波}
在人口异常稠密的中国,一个大家庭拥挤地住在一起,难免会发生口角。你问邻居:“你家有多少人?”他会回答:“十好几口吧。”你问:“所有的东西都是公有吗?”最普通的回答是:“是的。”十五或二十口人的大家庭,大概有三、四代,全靠一个商号或一块土地生活,收入都归家庭公有,所有家庭成员的消费也由公共财产满足。兄弟们为大家庭奋力劳作,而作为家庭重要成员的妯娌们,却很难团结在一起。她们年长的欺压年幼的,年幼的嫉恨年长的。每一个都竭力想使自己的丈夫觉得,他在大家庭里是最吃亏的。 
  家庭不和睦总是由年轻一代引起的。哪个社会能承受得起这种生活条件的压力呢?西方秩序良好的家庭也不足完全没有这种矛盾,更何况复杂严密的中国家庭呢?人与人之间存在着大量的分歧,就像人们的动机与兴趣一样。金钱、食物、衣服、孩子,以及他们往日的口角、鸡零狗碎的小事,都可能导致纠缠不清的争吵。 
  汉语中有个极不可思议的词,通常表示发火的意思。英语中委婉地译为“愤怒的物质”,这个词就是“气”。在中国的哲学与现实生活中,它是最重要的概念之一。一个人愤怒了,气就产生了。中国人认为,“愤怒的物质”和人体系统之间有着十分紧密的联系,强烈的愤怒会导致各种疾病与不适,如失明、心脏功能衰退等等。中国大夫问病人的第一个问题通常是:“什么事又惹你生气了?”在中国,富有经验的西方医生也乐于相信,气的确会导致中国人所说的那些疾病,下面的例子非常能说明问题:山东中部山区里的一个人,他有个老婆和几个孩子,其中两个还很小,1889年10月,他的老婆死了。这使他感到非常生气,他自己解释说,这并非因为他特别依附老婆,而是因为不知道如何照管孩子。一气之下,他抓起一把剃头刀,在自己的肚子上狠戳了三个大口子。他的朋友用棉线把伤口缝了起来。六大后,他再次生气,又把口子撕开了。不过,他那可怕的伤口居然愈合了。六个月之后,他已能够步行几百里到一家外国医院去接受治疗。伤口大部分已经长好,只剩下一个小瘘管,然而肠子的功能已经紊乱。前面我们说过,中国人富有生命力,这也是个典型的例证。 
  中国人喜欢大声喊叫地命令或批评别人,这一习惯已经根深蒂固,似乎难以彻底改变。对中国人来说,用正常的语调规劝别人,不时停下来,听一听对方的回答,从心理上几乎不可能接受,他不能不喊叫,不能不插嘴,毫不客气,如同一条兴奋的狗,非叫不可。 
  在东方,中国人的辱骂艺术已发展到登峰造极的境界。争吵一开始,骂人的话就像污水一样喷涌而出,在这方面,英语望尘莫及,其刻毒与持久,令人不由想起伦敦毕令奇街的卖鱼妇。哪怕最细微的摩擦,都会引来滔滔不绝的辱骂,就像两根电线一碰,马上就会产生电火花。无论何时何地,无论男人女人,无论哪一社会阶层,情况都一样。人们普遍抱怨女人骂人比男人更恶毒,更持久。俗语说:女人不裹脚,舌头利如刀。父母常在孩子呀呀学语时,教他们用土话骂人,并把自己被孩子骂当成最大的乐趣。骂人己成为中国人的第二天性,广泛地存在于社会各阶层中。文人及各级政府官员,甚至最高层官员被激怒时,都会像他们自己手下的苦力一样随意骂人。普通百姓在街上相遇,甚至以骂人的活来打招呼,这可以表明他们之间关系的亲密。 
  西方人的咒骂,声音不高,但能令对方痛人肺腑。中国人的咒骂若声音不高就毫无意义。英语中的诅咒是带翅的飞弹,中国人的诅咒是肮脏的皮球。他们骂人的话大部分被视为一种咒语。一个人发现自家田里的谷穗被人掐了,就会在村子里高声喊骂那未知的贼——尽管经常有怀疑对象。人们认为,这样做有两种作用:第一,可以告诉众人,他已恼怒了。骂人能让他发泄一下。第二,骂人还可以防止再次被偷。偷东西的人在暗处,(理论上)听着对他的可怕的诅咒,虽然一般不会被发觉,可他毕竟不能保证。被偷的人可能很清楚是谁干的,但他更乐意在大庭广众之中谩骂,作为对偷东西人的正式警告:他已被发觉或被怀疑,以后最好不要再做。假如被偷的人过于被激怒,这显然就是在声明:他要报复。这就是中国的骂街论。不过,他们坦率承认,这样做不仅不能防止被偷,而且也不能防止再次被偷。因为在众多的人当中,小偷不一定知道自己被骂了。 
  女人比男人喜欢骂人,她们经常爬到平房顶上,扯着嗓子叫骂,一连几个小时,有时直到嗓子嘶哑,方才罢休。在一个有社会地位的家庭中,倘能制止,是不会出现这种行为的。但是中国与其他地方一样,妇女一旦被激怒,是最难约束的。一般情况下,骂街很少有人注意,或者没人理睬。有时会在巷口遇到一个男人,或在屋顶上发现一个女人,已骂得面红耳赤,周围却一个人影也没有。如果天气较热,他(她)拼命叫骂一阵之后,会挥着扇子休息一下,再继续叫骂。 
  中国人吵架吵到一定程度,不动手就不能收场。在欧洲南部旅行的英国人发现,他们打架时将拳头从肩膀处击出的习惯,令拉丁民族十分惊奇。中国人和他们一样,从未受过拳击训练,即便学过,也是不伦不类。他们怒不可遏时,首先是抓住对手的辫子,尽力扯他的头发,这也是他们最主要的手段。倘没有第三者加入,双方又都没有其他武器,这样的“战斗”十之八九,仅仅是一场扯辫子比赛而已。 
  中国人的争吵,也是对骂比赛。比赛声音的高低,结果除了嗓子喊破之外,没什么严重伤害。中国极少有人对交战双方火上浇油。我们只见过“战斗”发生时,很快有人挺身而出,充当和事佬——这也是我们一直期望的,他们通常有两、三个。他们将叫骂的人拉开,好言相劝。但叫骂的人一旦发现自己处在和事佬的安全保护之下,就会骂得更凶,可他心里却很谨慎,对方有人准备僵旗息鼓时,他也明智地逐渐收敛,这无疑对双方都没有坏处。中国人即使在最愤怒时,仍非常理智,不管在理论上,还是在实践中,都不会忘记这一点。谁见过吵架的人转而扭住劝架的人,责怪他多管闲事呢?那可是紧要关头。中国人愤怒时,仍渴望和平——在抽象的意义上——只不过在自己的具体事情上,难以实现罢了。和事佬劝解他们,几乎总是扯走好斗的一方,后者则边撤边骂,表示对对方恶毒的蔑视。 
  中国人骂人,非常令人难以理解,他们并不揭露对手实际的过错,而勿宁污辱他的祖先,嘲笑他卑贱的出身。被骂的人则认为这是对自己尊严的严重伤害,其原因不在于是当着别人的面,甚至不在于自己被骂,而在于骂他的那些话令他太丢)“面子”了。骂人者感到自己做得不对,也不是认为自己的行为不光彩,有失身份,而是认为自己不该在那个时候、用那种话骂对方。 
  幸亏中国人没有随身携带武器的习惯,假如他们随身带着手枪,或像从前日本的武士,佩着剑,真不知他们发起火来,会酿制多少惨剧。 
  中国人一旦觉得自己受了莫大委屈,会马上气得暴跳如雷,失去控制,据说有个人,请求一位有经验的老传教士为他施洗,遭到了委婉的拒绝。于是,他就拿着刀子,逼迫老传教士为他举行仪式。幸好大多数初做修士的人对这种凭暴力进入天国的方式不感兴趣,可是这条原则普遍地存在于中国的社会生活中。一位乞讨的老太婆,遭到了拒绝,会躺倒在你的马车前。要是被你的车轧了,她就要谢天谢地了,因为现在她有理由要你永远赡养她,为她养老送终。笔者住的那个村子里,有个老泼妇,尽管邻居们乐于帮助她,可她总不满足,经常以自杀相威胁。有一次,她终于跳进了池塘里,想淹死自己,却发现水只能没到她的脖子,她怎么也不能一直把头没在水里,结果恼羞成怒,对全村人破口大骂。不过,她第二次跳的时候,村民们答应给她更大的帮助。 
  中国人有冤无处诉时,常常私了。比如婆婆过分虐待儿媳妇,法律管不着,社会习俗又认可,就要靠儿媳妇的娘家去讨公道。这时,若遭拒绝,肯定会发生一场战斗。如果没遭拒绝,但施虐者逃之夭夭,娘家的人就把她屋子里一切能打碎的东西,全部打碎,像镜子、水罐之类的。出完气之后,扬长而去。假如婆家人事先知道了风声,就会先把那些东西搬到邻居家里。据中国的一家报纸说,北京曾发生过这么一件事:一个小伙子和一个漂亮的姑娘订了婚,结婚时,却发现娶来的是个又老又丑的姑娘,而且还是个秃子。失望的新郎勃然大怒,狠狠地打了媒人一顿,大骂那些骗于,并砸毁了新娘的所有嫁妆。在这种情况下,任何一个中国人,只要有胆量,都会这样做。“怒气爆发,总会平息,这就要看“和事佬”的了——他们在中国社会生活中的作用可非同小可。这些重要的人物都热衷于和平,即使事不关己,也会主动出面,两边劝慰,促使双方互相谦让,协调一致。 
  社会纠纷不能用普通的方式解决时,或者说当事人怒气太盛,无法发泄时,就要打官司了。在中国,打官司是件大事。极度的愤怒会使人失去控制,大吵大闹,最终决定将冒犯者告官,以求“法办”,在西方,这是鲁莽的做法,在中国,完全是发疯。中国有些格言,明确表明,宁死也不愿打官司。狗被别人打死,乃小事一桩,我们会笑置之。可一位中国移民的狗被打死之后,却声明要告到法院。他的朋友问他:“一条狗能值几个钱?”他说:“狗系(是)不及(值)钱,可那家伙太狠了,他要赔我全价。”西方的法庭宁愿以高价拒绝受理,可在中国,它会导致两败俱伤,并结下世仇。不过,双方通常都会找些说情的人。这种人无处不在,价值也无可估量。他们的参与使成千上万的案子在审判前就了结了。我知道一个小村子,住着上千户人家,已经几十年没人打过官司了,究其原因,原来是当地衙门中一位有地位的人物一直在制约着他们。 
  *据来自北京的报告,现在的皇帝并不喜欢为他选择的妻子。太后作出的选择常常与皇帝的意愿相违背,令他很不满意据人们私下里说,宫庭中的婚礼情况与民间相似,“上行下效”嘛。 
  一个复杂如中国的社会机器一定会经常咯咯作响,在巨大的压力下扭曲变形,可中国社会却一直安然无恙。这些压力并没有使中国社会破产、毁灭。中国的政治机体也像人的身体一样,存在着大量的润滑液囊,在最需要时。最需要处,往往会及时渗出一滴来,加以润滑。爱好和平的品质使每个中国人都成为有价值的社会分子。他们热爱秩序,尊重法律,甚至在不值得如此时仍惜守不渝。所有支那民族中,中国人是最容易统治的,只要统治方式符合他们的习惯。当然,其他文明,在很多方面或大多数方面,都优于中国。不过,能像中国社会这样承受如此之巨大压力者,大概寥寥无几,其中,最为功不可没者,当数那些和事佬。


\chapter{株连守法}
中国人有个典型的特征,可以用“负责”一词来概括,西方很少有一个词能像这个词那样重要,那样具有丰富的意义。在西方,个人是社会的基本单位,社会是个人的集合体。而在中国,社会却是由家庭,村落或宗族等构成,这些通常又是一致的。中国有千千万万个村子,每个村子的居民都源于同一个祖宗,同姓一个姓,共享一块热土。他们搬到现在居住的地方,可以追溯到几百年前的一次政治变动,比如明朝灭亡,甚至在明朝建立的时候。在这样的村子里,堂兄弟几乎是最远的关系了,男性长辈,不是父亲,就是叔伯,或者什么“爷爷”。有时,一个小小的村子,竟会住着十一代人。他们并不像我们所想像的,寿数越高,辈份越高。中国人年龄很小就结婚,以后甚至到晚年还娶妻纳妾,一辈了不断地生孩子,结果就造成了错综复杂的亲属关系。如果不特别询问或仔细注意名字中表明辈份的字,实在难以分辨出谁是晚辈,谁是长辈。一个年近七旬的老翁会叫一个三十岁的男人“爷爷”。所有的堂兄弟之间都可以互称“兄弟”,假如外国人对此感到困惑不解,坚持要搞清楚,问他们到底是不是“自家兄弟”,回答经常很有意味:他们是“自家的堂兄弟”。笔者曾经这样问过,那人几乎毫不犹豫地说:“嗯,当然,你可以称他们为自家兄弟。” 
  这些都是中国人社会团结的具体表现。正是这种团结决定了中国富有责任心。父亲要对儿子负责,不单在儿子“成年”之前,而且永远负责。儿子也永远对父亲负责,俗话说:父债子还。兄长要对弟弟的一切负责,“家长”一一通常是长辈中年龄最大的男人一一要对整个家庭或家族负责。不过,这些责任会随环境的变化而变化。 
  风俗习惯不同无关紧要,个人是重要的。这一点,理论上很难论述清楚。在一个显赫的大家庭里,尽管有很多知书达理的人,也有一些是当地的头面人物,或科班出身的,但“族长”却可能是个头脑糊涂的老头,大字不识一个,甚至一辈子连离家十里远的地方都没去过。 
  家庭中兄长对弟弟或年长者对年幼者的影响,最直接,也最绝对。这与我们所提倡的自由势如水火。弟弟就像个仆人,整天盼望改变自己的地位,而哥哥偏不许他这样做。弟弟想买件棉衣,哥哥认为太贵,不给钱。笔者正在写这本书时,又接到一个报告:一个中国人,手头上有些罕见的古币,有个外国人很想买。为防止钱主不愿卖——在中国,一个人手里有另一个人想要的东西,情况常常如此——中间人就建议,送些西洋糖果和小玩艺儿给钱主的叔叔,让他对钱主施加压力,最后迫使钱主把古币卖掉。 
  有这样一个滑稽的故事,一位从西方国家来的旅行者,途中遇到一个长着长长白胡子的老人,在伤心地哭泣。旅行者感到很意外,就停下来问老人,为什么哭泣。老人告诉他,自己刚被父亲用鞭子抽了一顿。“你的父亲在哪儿?”旅行者问他,“那儿。”老人指着前面。旅行者便顺着老人指的方向往前走,又遇到了一个胡子更长更白的老人,“那是你的儿子吗?”旅行者问,“是的。”“你用鞭子抽他了?”“是的。”“你为什么要打他呢?”“因为他对爷爷无礼。如果他再这样,我还会再用鞭子抽他。”假如将这个故事的背景换成中国,这可就不是一个滑稽故事了。 
  家庭成员应该彼此负责,邻里之间也应这样。不管他们是否有亲属关系,都不该例外,因为住处相邻嘛。中国人认为善良与邪恶会传染。近朱者赤,近墨者黑。孟母三迁就是为了找个理想的邻居。而接受了盎格鲁-撒克逊民族共和思想的人,对谁是他的邻居毫不在乎,在城里某个地方住上一年,他甚至还不知道隔壁邻居的名字。不过在中国,情形就完全不同了。倘若有人犯了罪,其邻居也逃不脱干系,犯了类似英国法律说的“包庇罪”。因为他们知道罪犯的企图,却不向政府报告。说“我不知道”,丝毫不起作用。你是他的邻居,就应该知道。 
  对杀死父母案子的处理,很能说明中国人的负责观念。在“孝悌为先”一章 里,我们提到过,这类罪犯一般是疯子。如不自杀,就应该心甘情愿地接受凌迟之刑。几年前,北京《邸报》上的一份奏折中,中部某省的总督报告,他在处理一件杀死父母案时,命人推倒了罪犯邻居的房子,因为他们没有给罪犯良好的道德影响,以令其改邪归正。一般的中国人可能认为,这种处理方式合情合理有时,某地方有人犯了罪,除了对人进行惩罚外,还要拆毁一段城墙,或者修正一下城墙的样式,比如,将方角改成圆角,把城门换个地方,甚至干脆封死。要是一个地方老发生犯罪,据说该城就要被夷为平地,在别的地方另建新城,不过,这种事,我们还未遇到过。 
  村子里,地位比普通老百姓略高的称地保,管一个或几个村子,职责也十分繁杂,不过,总的来说,是充当沟通地方政府与百姓的媒介,地保经常会陷入麻烦中。任何一种纠葛都会给他带来麻烦。假如遇到一个吹毛求疵的地方官,有时甚至会因为没有汇报他不可能知道的事情而被打得血肉模糊。 
  地位比地保再高的是县官。在百姓眼里,他们是中国最重要的官吏。在百姓面前,他们是老虎;在上级面前,他们又是老鼠。一个县官至少要处理六大方面的事务,他既是民事、刑事司法官,又是行政司法官、验尸官、财政长官和税务官。一个官员要处理这么多事务,当然不能细致入微,明察秋毫。无论从生理上还是心理上说,这都是超负荷的,使得所有的事情都不能处理好。况且,很多县官只一门心思想着如何捞油水,对任何公务都不感兴趣。因为公务繁多,彼此又不协调,即使县官有良心,也难免犯不少错误。一些事情处理失当,他总是难逃其咎。大部分县官要依靠师爷或随从来帮助处理日常事务,与所有中国官员一样,县官总被想像为对辖区内的一切都了如指掌,也能随时防患于未然。为做到这一点,每个城市或乡村中,每十户划为一甲,每甲设保甲长。每户门前挂有一个小牌子,上面注明户主姓名和该户人口数目。这种户藉制度,有点像古代撒克逊人十户区或百户区制,它有利于确立责任区,某个保甲区内一旦出现可疑的陌生人物,第一个发现的就迅速报告保甲长,保甲长立刻报告地保,地保再报告给县官,县官马上采取措施,“严密搜捕,严厉惩处”,这种简易的保安措施,使所有的地方犯罪,还未发生就被发觉了。这不是靠陌生人长相可疑,而是靠住户固定。这一制度还使良好的民风代代相传。 
  显然,这一措施只有在住户固定的地区方能奏效。然而,即使在中国这样人口最为固定的国家,保甲制度在很大程度上也只是个法律上的假定。有时,在一个城市,以前从未见过门前挂牌子,可突然有一大,每户门上都挂上了。这就说明县官来了,他想加强这方面的管理。有些地方,只有冬天才挂上,因为冬天最危险,坏人最多。不过,据我们所知,该措施只是昔日的经验,现在徒具形式而已。实际上,也几乎已经完全消失了,连续走几个月,几千里路,沿途挂牌子的住居,不足百分之一。 
  前面可能说过,中国的保甲制度和所谓的人口调查紧密联系在一起。假如每户的门牌一直都准确地标明该户的人口数目;假如每个地保都有一份其辖区内人口的清单;假如每个县官都准确地将这些清单上的数字汇总——对整个帝国人口的准确统计就会非常容易,只要将这些一连中的数字加起来就行了。可惜,这些都是“假如”,而事实上,几乎没有一个可以实现。门牌根本不存在。当某个地方官偶尔需要人口总数时,无论是他自己,还是他完全依赖的众多地保,都不能保证提供一个准确的数字,他们对此都毫无兴趣,因为人口调查没油水可捞。因此,对中国人口的准确统计,只能从想像中虚构了。即使在文明程度较高的西方国家,人们也总是把人口调查与税收联系起来。在中国,它更是令人们疑神疑鬼。如果各地不能持久投入地实行保甲制度,就绝对不可能准确地统计出人口的数目。 
  地方官犯点小罪,可能平安无事,也可能遇到大麻烦。即便如此,只要找有势力的朋友说说情,或者明智点,花些银子,也就完事了。就算丢了乌纱帽,也会把原因归结为他的辖区内发生了不可避免的事。在中国,这现象极为普遍。 
  接下来有必要阐述一下官僚阶层是如何实行责任制度的。在翻译过来的北京《邸报》中,这样的例子每期都层出不穷。几年前就曾披露过这样一件事:一个值班的士兵偷了自己看守的大约三十箱子弹,卖给了一个做罐筒盒的。后者认为那些子弹是部队多余的次品。案发后,士兵被打了一百大板,流放边疆服苦役;负责仓库的小官,虽然允许交钱赎罪,减轻处罚,仍被打了八十大板,革去官职;买主因考虑是出于不知情,免于处罚,不过按常规,打了四十小板;管理这些士兵的连长,因为“纵容”犯罪,也被撤职,听候审判,但这家伙很聪明,及早悄悄地溜走了。上表奏折的刑部受命决定对该部队最高指挥官的处罚,他对此案也有责任。由此可见,每个人都是这条锁链上的一个环节,谁都不能以不知情或难以防止这类犯罪为借口,逃避责任。 
  北京《邸报》中,每年都有上报河流泛滥的奏折,像这类事情更能说明中国人互相负责的品质。1888年夏,直隶省附近的永定河泛滥,河水从山上冲下来,一泻千里。官员们看来是闻讯就赶到了现场,冒着生命危险,奋力抢救。可惜,人难胜天,他们不过像暴风雨中的蚂蚁一样,栖惶无助。尽管如此,李鸿章也不为之所动,仍请求皇帝立即摘去他们的顶戴花翎,或者保留官职,取消贵族身份(这是朝廷不满时最常用的手段)。直隶总督也一再上表请求将自己交付刑部,依罪处罚。同样的河水泛滥后来还发生了几次,每次都有同样的表章,皇帝也经常命令有司记录“备案”。几年前,河南省修复河堤以使黄河回归故道的工程失败了,自巡抚以下的大批官员遭到罢黜和流放。 
  中国人的责任感十分强烈,即使天子本人也不例外。他经常发布诏书,承认自己的缺点,把暴发洪水、饥荒和农民造反的责任归咎于己,并乞求上天宽恕。他要对上天负责,这和他的臣子对他负责一样,十分现实。皇帝失去了皇位,就表明他失去了“天意”,上天要将皇位交付应该登基的人。 
  中国人的责任观念与西方观念最相抵触的是一人犯法、株灭九族,太平天国起义中有许多这方面的例子,最近,土库曼斯但穆斯林起义的首领雅库·贝哥也是被满门抄斩。这种做法并不仅限于镇压起义上。1873年,“一个中国人被指控盗取皇陵中的陪葬品,结果全家四代,上至年近九旬的老人,下至几个月大的婴儿,全被杀掉。在这个案子中,除罪犯一人外,其余十一人全是无辜的,根本没有证据表明他们参与了或知晓罪犯的行动。” 
  中国人的责任观常被视为其各项制度永恒不变的原因之一。它就像脚镣手铐,束缚着每个人,各级官员也因之为他们从未参与或根本不晓得的事情受到处罚,这就不能不破坏各种公正原则,并直接导致了上上下下所有的官吏都掌握了一套弄虚作假的手段,如实汇报情况,还要受到严厉惩处,完全颠倒了公正原则,违背了人性。因此,官员们发现不能控制犯罪现象或觉察得太迟的时候,即使本该负责,他也要掩盖真相,以逃避责任,过分要求人们彼此负责足以说明中国为什么会出现弊政和缺乏公正。我们对此一直都很关注。 
  每个致力于研究中国问题的人都会发现,中国官僚体制中还存在着另一个弊病,那就是官吏的俸禄不能满足其基本的生活需求,一点微不足道的津贴也很少能全领,并且还要作为各种罚款交回去,最后,连衙门中日常的开支都不足以维持。做官的,没有其他门路可走,只好贪污受贿,以摆脱困境。 
  中国人的责任观不符合公正原则,已是昭然若揭的事实,也令我们时刻难忘。可是,我们并不能因此忽略了其优点。 
  在西方,一个人被证明有罪之前,是清白无辜的。你也很难将责任强加到某人头上。一列满载旅客的火车,因超重压断了桥梁,不可能归咎于某一个人。一座高楼倒塌了,压死了很多人,尽管建筑师会受到指责,可他表示当时他已做出最大努力,也没听说过他将因此受到处罚。一辆装甲车翻倒,或者一次军事行动因准备不足,遭到挫败,人们也只是指责整个体制,从不针对某个人。中国人在社会公正方面远远落后于我们,可是,难道我们就不应该学习他们古老的经验吗?它可以便我们每个人都严格地为自己的行为负责,从而维持国家的安全。 
  中国人的责任感对居住在那儿的外国人也十分重要。家里的“僮仆”能随时取出任一把勺子、叉子,任一件古玩;负责家务的总管,除自己可以欺骗你之外,不允许任何人欺骗你;那些买办,虽有大权,但又对每一分钱,每一个职员负责——只要我们和中国人打交道,就永远会遇到这类人。中国客店的老板,很少有善良的,尤其对待外国旅客方面。可是,我们听说一个老板,为了把一个空沙丁鱼罐筒盒还给一个外国旅客,竟追了半英里,他以为那是什么重要的东西。他认为应该那样做,可不像美国的旅店老板,他们总是冷冷地通知旅客:“本旅馆对丢在大厅里的脏靴子,概不负责。” 
  要是举荐了某人,就要对其品质、行为和欠债负责,中国人普遍承认这是一种社会义务。外国人要想与中国人共事,就不能忽略这一点,一个中国监工,不论处在哪一位置,都会对每一次录用或解雇下属负责,这种情况会对事情的各个环节产生特殊的影响。在与中国人相互交往的漫长历史中,外国人一直本能地对这一点非常欣赏。传说从前有一个银行里的买办头,因为“男僮”让蚊子钻进了经理的蚊帐,竟将他叫去好好说明原因,如果中国人看到外国人对下属从不负责,或者不重视“应尽的责任”,肯定认为极不适宜,假如想学会,更要花很长时间。 
  中国人有许多令人赞叹的品质,其中有一种是天生的尊重法律。我们不知道,是社会制度造就了这一品质,还是它造就了社会制度。但是,我们知道,中国人无论从先天的本性,还是从后天接受的教育上说,都是一个尊重法律的民族。在讨论民族的忍耐美德时,这一点已有所涉及,不过,它与中国人责任感之间的联系值得特别注意。在中国,每个男人、女人和孩子都对他人负责,这一重要事实要时刻牢记。虽然一个人应该“远走高飞”,可他逃脱不了自己的责任;即使他逃脱了,他的家庭仍不能逃脱,这是铁的原则,它虽不能保证使一个人改邪归正,却常常可以使他不致于变得十恶不赦。 
  中国人很怕进官府,打官司。它也能说明中国人对法律的尊重。尤其是文人,他们一被召到官府,就吓得胆战心惊,噤若寒蝉,大气也不敢喘一口,即使事不关己,也会如此。我们就确实知道一个文人,被请去作证时,吓得像患了癫痫病一样,浑身抖个不停,最后竟昏倒在地,回家不久,就死了。 
  中国人对法律的尊重,与共和政体所表现的精神构成了鲜明的对比,这种精神是由历来就追求共和政体的人们创造的,学院法规、市政法令、国家法律,全都遭到默默地抵制,仿佛追求个人自由不是当代最大的危险,反而是最大的需要。不过,个人或社会完全应该把阻止,揭露虚伪和欺骗当成应尽的职责,并将这一点视为对中国人处理各种社会事务之方式的最大抗议。可是,在基督教国家,无论目不识丁的人,还是举止文雅,有教养的人都有意无意地轻视法律,仿佛不需要法律维护公众的利益,并且违抗法律要比遵守法律更能体现法律的尊严,这难道很光彩吗?我们的法律既没有被取消,也没有贯彻实施——这种既存在又不存在的反常状态已使所有的法律都遭到了公众的蔑视,我们对此能作何辩解?还有,在过去三十年,犯罪率迅猛增长,很多地方,人类生活的神圣感已经显著淡化,我们对此又作何解释?对于统计学无能为力的事,做出武断的评价,完全徒劳无益。我们必须承认,中国的城市生活比美国的城市生活更安全——北京就比纽约安全。我们也相信,在中国旅游比在美国旅游更安全。应该记住,从总体上说,中国人和美国移民一样无知,怀有偏见。他们也容易受盅惑,聚众滋事。可是,令人奇怪的是,这种事并不经常发生,对外国人也没有生命威胁。 
  中国人相信,人的思想、行为会影响上大的意志。为了给父母治病割自己身上肉的做法,就体现了这一观念。在讨论中同人的孝顺性格时,我们已经谈过了。我们不准备坚持这种观念是正确的,可有一些支持这一观念的事实仍值得一提。中国18个省的面积与地理环境和美国落基山以东的地区很相似。美国气候变化无常,就像小麦乔里.弗莱明对乘法表的评论:“令人难以忍受。”霍桑评价新英格兰时,也说那儿“没有气候,只有各种人气的范例”将波士顿、纽约、芝加哥的气候与中国同一纬度地区的气候相比,就可以看出,同一些地理书对美国气候的判断一样,中国也有“严寒酷暑”,因为在北京所处的纬度上,年温差大约有l00华氏度,这必然会产生各种不同的气候温度。 
  可是,在中国,同样的冷温变化并没有导致像伟大的共和国——美国那样的变化无常、难以预测,而是宁静平稳、井然有序,很适宜于她那古老稳定的社会体制。钦定的帝国历书体现了天、地、人三者的和谐统一。我们不清楚,在辽阔的帝国疆域内,是否各地百姓都同样信服它。不过,在我们所熟悉的地区,它的确能告诉人们有关天气的信息。“立春”那天,春天会翩然而至。在不同的几年中,我们都发现,“立秋”一过,气候会明显发生变化,再也没有了夏天的燥热。而在西方国家,不期而至的霜降会给人们突然造成危害,一年十二个月它都会不定期地出现。为避免这一点,中国历书将“二十四节气”之一定为“霜降”,日期为12月23日。在这一天之前,一点霜花都看不见,而到了这天早晨,地上就会蒙上一层薄薄的白霜。以后的每天早晨也都会有霜,我们观察这个现象好几年了,很少看到有提前或推迟三天的。 
  在中国,这些非生物性东西的出现有规律,合乎理性,生物的出没更是如此。很多年,我们都注意到,在早春的某一天,窗棂上点缀着几只苍蝇,已经有好几个月没在那儿看到苍蝇了。每逢这时,只要打开帝国的历书,就一定会发现这一天是“凉蜇”。 
  据说,讲英语的民族,人的血管中流淌的是肆无忌惮的血液,它使我们蔑视法律,不服约束。布莱克斯顿说:“我们强健的英国祖先认为,只有在特定的时间,人们才能自由的活动。”不过,也正因为我们勇敢的祖先,个人自由观念和人权经历了很长时间才得到确立。 
  但是,虽然这些权利已经很好地确立了,难道我们就不需要多强调个人意志服从公众利益、不需要维护法律的尊严吗?在这方面,我们不是有很多东西应该向中国人学习吗? 

\chapter{相互猜疑}
没有一定的互相信赖,人就不能在有组织的社会中生存,这是个勿容置疑的事实。对中国这样一个组织高度严密、复杂的社会,更是如此。尽管人们都承认这一点,仍有一些现象需要注意。这些现象并不符合我们的观念,可对于了解中国的人来说,却是十足的事实。我们所要讨论的主题是中国人的相互猜疑,这一性格特征其实并无特别之处,所有的东方民族都具备。不过,中国的天才们无疑大大地改变了它的表现形式。知道一些与己无关、但可能引起严重后果的事,就会十分危险,它会引起极大猜疑。中国人如此,其他民族也不例外。 
  相互猜疑,在中国经久不衰。最引人注意的是帝国各地的城中均围着高墙。汉语中,“城”一词本身就包含着被墙所围的意思,就像拉丁语中的“军队”一词也有训练、锻炼的意思一样。帝国的律法规定每个城市必须用一定高度的墙围起来,不过,它和许多其他法令相同,没有形诸文字,坚决要求实施,因为有很多城墙没有任何保护设施,任其颓毁。在太平天国起义中,有一个城市曾被起义者攻破,并被占据了好几个月,尽管城墙没有被全部摧毁,可从那以后,十几年都没重修,还有许多城墙不过是薄薄的一层泥墙,连狗都可以任意爬进爬出。所有这些颓败的现象只反映了帝国的贫困,一旦有危险警报出现,首先就是修城墙。而修城又成了官吏或暴发户掠夺的最便捷的途径。 
  中国之所以有那么多城墙,是因为政府不信任百姓。尽管从理论上说,皇帝是百姓的父亲,他的臣僚也被称为“父母官”,但所有的人都清楚,那只不过是一种说法而已,就像说“加”或“减”一样,百姓与统治者之间真正的关系是孩子与继父间的关系。整个中国历史充满了起义,如果中央政府及时采取适当行动,大多数起义显然可以避免。可是,政府并不想及时采取行动,也可能是它不希望这样做,或者有某些原因使它不能这样做。起义正在悄悄地准备着,政府也知道,可官员们只是像乌龟一样地把头缩进壳里,或者像刺猬一样团成球,立刻躲进现成的防御城堡中,把动乱留给军队去收拾。 
  与其他东方城镇一样,中国居民住处周围也建有高墙,这是他们相互猜疑的另一表现。外国人对中国人谈起伦敦、纽约这类城市,若故意说这些城市是“有围墙的城市”,会感到十分为难。使一个可能对西方感兴趣的中国人理解,西方人的住处周围没有任何防护设施,也并不容易。中国人会立刻认为,那些国家没有多少坏人,尽管他没有什么根据。 
  在中国农村,人们一般拥挤地住在一处,这也可以说明中国人相互猜疑。这些农村实际上是微型城市,它防御的不是外来敌人,而是彼此防御。据我们了解,只有一些山区例外。那些地区土地贫瘠,养不了几户人家。他们又实在太贫穷,根本不用怕贼。巴伯先生描绘了四川的情况:“地主和佃户各自住在自己的田舍里,他们宁愿分开住,而不愿将住处挤在一起。”如果这个例外是因为古老的四川比其他省更期望和平,那么,它就恰恰证实了巴伯先生所说的:这种期望已经历了太多痛苦的失望,特别是太平天国那段日子,尽管在此之前曾有过很长一段和平时期。巴伦·梵·瑞恰斯芬也很赞成巴伯先生的观点。 
  中国人,包括其他东方人,在观念上和实践中对待妇女的态度,也是他们相互猜疑的最重要的表现。其观念已经人人尽知,就是花上一整章也讨论不清其中的一点。女孩子一到青春期,就变得像“私盐”一样危险。订婚之后,就更加不能外出见人了。极细小、单纯的事都会招来恶毒的流言蜚语。“寡妇门前是非多”,也是公认的社会真理。尽管中国妇女比印度、土耳其的妇女享有更大的自由,*但仍不能认为中国妇女能获得较高的尊重。妇女普遍遭到歧视,处于从属地位;一夫多妻制和纳妾制也一直存在一一这些都表现出对妇女的不尊重,可是在西方,尊重妇女是再平常不过的事。中国表达对妇女看法的俗语也许被视为长期经验的总结,随处都可以听到。女人被说成是天生下贱、目光短浅、不可信赖的货色,还被当成嫉妒的化身,人们常说:“妒莫过于妇人。”这里的“妒”想传达的意思,是和它读音相同的一个字:“毒”,这种观念,有诗为证: 
            竹林蛇口 
            赤蜂尾上 
            狠毒莫若 
            妇人心肠 
  另外,歧视妇女的观念还渗透在文字中。作为客观的表现形式,它经常引起人们的注意。一位杰出的中国学者,为了回答笔者的问题,仔细研究了一百三十五个以“女”字为偏旁的常用字。结果发现,其中十四个为褒义,如“好”、“娴”等;其余三十五个为贬义,八十六个为中性。那些贬义字囊括了汉语中最恶毒无耻的意义,如虚伪、欺诈、堕落、不忠、自私之类。三个“女”字组成的 
  * 可这种自由不能以表面现象来判断。一位在印度德里居住了若干年的妇女,来到山西省首府定居,她评判说,通常中国街道上的妇女人数要少于印度。不过,事实与这段注释并不矛盾。 
  奸字,表达了“与未婚者私通、通奸、诱奸”等等意思。 
  据说,不信任别人有两个原因:一是不了解对方;二是了解对方。原因不同,中国人的处理方式也不同。中国人天生具有联合的本领,如同化学原子化合一样。他们彼此不信任是以含蓄的方式表达的,只要在恰当的时间,以恰当的方式,我们就很容易发现这一点。媳妇煽起家庭成员之间的不断猜疑,为了分配共同的劳动成果,她们总是使出浑身解数,挑拨丈夫与家庭间的关系。 
  不讨论家庭生活了,它可以写上整整一章 。现在让我们看看没有复杂家庭关系的人。家里的仆人,假如不是由某位富有责任心的人介绍来的,彼此之间总是保持武装中立。可假如其中一位有劣迹传出来,他首先不是问自己:“主人是怎么发现的?”而是问:“谁告诉他的?”即使他心里清楚,有很多证据可以证明是他干的,他的第一个念头仍是别的仆人在排挤他。我们认识一位中国妇女,有次她听到院子里有人高声谈话,脸色就马上变了,怒气冲冲地从屋子里奔出去,她认为,人们是在愤怒地议论她。可事实上,只是有人在买一堆谷草,嫌卖主要价太高。 
  某个仆人被意外辞退,他肯定会满腔仇恨,这也是由猜疑引起的。他怀疑除他自己之外的每一个人,即使他知道所有的理由中,任何一条都足以使他被辞退,他仍会坚持有人说了他的坏话,坚持说辞退他是毫无道理的。他必须挽回“面子”,他猜疑的天性必须满足,外国家庭的仆人也会发生这类事,不过程度不同,因为中国仆人知道如何欺骗善良的外国人。但在中国主人那里,他想都不敢这样想。因此,很多外国人一直雇用着早该辞退的仆人,他们不敢那样做。他们也知道,单单提出辞退就会招怨树敌,其中主要是那些受过指责、“不光彩”的仆人。外国人没有勇气将他们赶走,以免失败后,情况更糟。 
  有一个故事,讲的是中世纪奥地利的一座城市遭到了土耳其人的围攻,眼看城池就要被攻破了。在这危急关头,一位姑娘突然想起了自己的很多箱蜜蜂,就把它们搬到城墙上。这时土耳其人已快爬到城垛子上了。群蜂飞出,敌人潮水般地退却了,城市被挽救了。中国人的策略常常和这个奥地利姑娘一样,成功对于他们仅是一种标志,一位拉丁教授说过,人们宁愿“面对风暴的警报”,也不“面对风暴自身”,中国人对待骚乱也如同对待风暴一样。虽然中国人说:“用人不疑,疑人不用”,可经常只是睁一只眼,闭一只眼,假装没看见,而对于外国人来说,可没那么简单,容易处理。 
  孩子到了独立闯世界的年龄,我们认为有必要告诉他们:最好不要过分相信陌生人。中国的孩子不需要如此告诫,他们早已从母奶里汲取了这一经验。有句俗话说:一人不进庙,两人不看井。我们感到迷惑不解,为什么一个人不能进庙呢?原来是和尚可能会乘机谋财害命。两人不看井,因为假如一个人欠了另一个人的债,或另一个人身上有他渴望得到的东西,他也许会趁机把这个人推入井中。 
  另外一些相互猜疑的例子来自人们的日常生活。在西方国家,有自由,无压抑,而中国明显缺乏自由。在我们看来,处理一件事情,理所当然应该采用最简便的方法,可在中国完全不同,需要考虑很多因素。无论遇到什么事,中国人考虑最多的是两种东西——钱和粮,它们是大部分中国人生活的两个核心。中国人很难相信,一笔钱若交到另外一个人手里,能够按既定的方案分配给众人。他们没有那种分配经验,只认为,钱到了另外一个人手里,他就会千方百计地从中克扣。同样,安排一个中国人为他人分配食物也很困难。表面上,怎么也看不出接受食物者会怀疑分配者从中克扣。此时,不满的情绪可能被完全压抑了。但我们不能据此认为,没有猜疑存在。其实,只有外国人才把它当成一个问题,中国人认为,只要机器中存在摩擦,人与人之间就存在猜疑。 
  中国旅馆的侍者有个习惯,他们总对即将离开的旅客大声报出清单上的每一款项。这可不像一些旅客所认为的,是在称赞他的阔气,它有更实际的目的,是为了使其他侍者知道,报单的人并没有私藏小费或“酒钱”,尽管实际上他们个个盼望能够这样。 
  假如一件事接近完成时,需要重新磋商或修正,中国人就不能像西方人,一封信就可以把事情办妥。当事人要亲自到负责人家里去。如果时间太晚,负责人不在,还必须再次登门,直到见到为止。假如通过中介,谁都不敢保证事情不被歪曲。 
  人们经常讨论中国人的团结。有些时候,整个家庭或家族会干预属于家庭成员个人的事。一个明智的外姓人。这时会格外小心,以免介入,惹火烧身。有句很妙的格言说的就是外姓人的忠言难以被接受。“我们的事情,这家伙搀和什么?一定是居心不良!”对朋友和老邻居都如此,更何况外来户和没有特殊关系的人。 
  “外”这个词在中国还有远近之分。外国人办事不顺利,因为他来自“外国”;乡民办事不顺利,因为他来自“外乡”。一个外来者,背景不明,又不想让别人知道,情况一会更糟糕。谨慎的中国人免不了会想:“谁知道这家伙葫芦里装的是什么药?” 
  一个旅行者碰巧迷路,来到了一个村庄。假如天黑了,尤其时间太晚,他会经常发现,没人出来给他指路。笔者有一次就曾来来回回转了几个小时,花钱也雇不到向导,甚至听不到一句指路的话。 
  中国学生上课时一律扯着嗓子念,既损害他们的发声器官,也令外国人心烦意乱。这是一种“传统”,如果想刨根究底,人们会告诉你,听不到读书声,老师就不知道学生是否在专心学习。学生背诵时,要背对老师,老师以这种奇怪的做法来确保学生不偷看。 
  并不是所有的文明都主张要款待陌生人。和东方人实际接触之后,所罗门关于对陌生人要谨慎的箴言获得了新意。但中国人的谨慎已到了高不可攀的地步。一位中国老师受雇于外国人,收集童谣。一次,他听到一个小男孩正在哼一支不清楚的歌谣,就让他再唱一遍,可是,孩子吓得慌忙逃走了,再也没露面。小男孩的行为是中国这种环境中典型的产物。一个人精神失常,离家出走,他的朋友四下里打听,希望能得到一点儿有关他的消息,其实,他们很清楚,这样做,希望是非常渺茫的。假如有人说曾见他来过,后来又走了。寻找的人会很自然地问:你当时做了些什么?这样,麻烦就来了。所以,如果询问者是个陌生人,人们就一定会回答:不知道。这也是最安全的办法。 
  根据我们的经验,在中国,陌生人寻找当地一位有名的人物,也会出现类似的情况。有一次,一个看似来自邻省的人,到某个村子去找一位名人,最后却失望地发现,全村人众口一辞,都断然否定认识这么一个人,而且还信誓旦旦地表示,连听说都没听说过。这些谎言并不是事先串通好,编造出来的,因为村民们没有串通的时间,他们不约而同地这样做,就像北美的草原犬鼠,一见到陌生的东西,就一头扎进洞里,是出于一种本能。 
  在所有的这类事情中,一句简单的招呼,都可以显示出它与当地方言的细微差别。乡下人会经常遇到盘问,他家住在哪儿,距离某某地有多远等等,似乎在确保他不是在骗人。同样,学生入“闱”时,不仅要询问他的学历。还可能要盘问他写的文章,以及是如何完成的。用这种方法,欺骗就很容易被识破,事实上也经常如此。一个人不要企图冒充当地人,因为口音会泄露他的籍贯。陌生人不仅很难获得某人的下落,而且他的行为还会引起普遍的猜疑,就像前面说的那个例子,整个村子都在猜疑。有几个中国人曾长期在一家外国医院接受治疗,笔者让另几个中国人去找他们,结果,一个也没找到。有时,即使一个人鼓起勇气和陌生人交谈,也至多只说出自己的姓,绝对不会泄露自己的名字,因为同姓的人毕竟很多。还有时,送信者苦苦寻找的村子就在眼前,却会莫名其妙地找不到,甚至连最后的一点线索也消失了。就在前面说的那个例子中,陌生人在方圆一、两里路内都没有找到的那位名人,其住处实际上距他只有几十米远。 
  笔者认识一位老人,他有一个富有的邻居。两人从前曾同是中国某一秘密教派的成员。可当人们询问起他邻居的情况时,却发现这两个从小一起长大,相邻而居六十余年的老人从未接触过。“怎么会这样呢?”“因为他老了,很少外出。”“你为什么不常去看望他,谈谈过去的时光呢?你们相处得不好吗?”老人不自然地微微一笑,然后摇摇头:“不,我们相处得很好。但他富有,我贫穷,如果我去他家,就会惹人说闲话:他去那儿干什么?” 
  中国人相互猜疑有一个明显的表现:他们从内心里不愿被单独留在房间里。否则,一定会觉得不自在,还可能会溜出房间,到走廊里去,似乎在说:“别怀疑我,瞧,我没拿你的任何东西,我不在屋子里。”自重的中国人拜访外国人时也会这样。 
  没有什么能比非正常性死亡更易引起强烈的猜疑。典型的例子是已婚女儿的死亡。虽然,像前面说的,女儿活着时,父母无力保护她;可假如她死得可疑,她的父母在一定程度上就获得了主动权。女儿自杀后,他们就不再像过去那样俯首屈从,而是盛气凌人地提出一些苛刻的条件。这种情况下,拒绝和女方娘家达成一致,就会引起一场持久、恼人的官司。娘家的目的首先是为了报复,不过,最终目的只是为了保住“面子”。 
  中国有句古话:瓜田不纳履,李下不整冠。这句箴言表达了一个普遍的真理:在中国,走路都要小心翼翼。这就是中国人生性沉默的原因,我们对此有时十分难以忍受。他们都知道,一言不慎,就可能酿成大祸,而我们对此一点也不懂。 
  中国人的商业活动表明中国有各种各样的猜疑形式。买方、卖方彼此都不信任,只有严格保持中立的第三者才能促使买卖成交,他们只有通过讨价还价才能获利。而且直到付款,交易才算做成。情况更复杂时,还需要形诸文字,因为“空口无凭”,必须“立此存照”。 
  中国金融市场的混乱,部分原因在于钱庄对顾客有根深蒂固的不信任,顾客也有充分的理由表明自己不应该相信钱庄。南方的假外币,各地的假银元,都是这个伟大的商业性民族生性猜疑的明证。他们决心做成一桩交易时,非常精明;不愿做时,表现得更精明。顾客出于猜疑,总希望天黑以后使用银子,这一点中外皆然。如果城里的商店建议等到第二天,这是明智之举,也不足为奇。 
  中国的银行系统似乎包罗万象,错综复杂。我们从马可·波罗的游记中可以知道,中国很早就使用银行票据了。但并不普遍,好像被严格地限制在一定的流通领域。两个相距仅十里的城市,各自的钱庄都有充足的理由不收对方的票据。 
  中国的利润率很高,在24%一36%之间,甚至更高,这也是中国人彼此不信任的表现。这种暴利大部分不是钱本身的利润,而是巨大冒险的保险费,我们所熟悉的西方的投资方式,他们几乎没有,这不是因为帝国资源开发不足,而是因为人们普遍彼此不信任,“人生无信不立”,根据这一点,在将来的很长一段时间内,在许多问题上,中国人仍将表现出猜疑的特点,这必定会极大地损害他们的利益。 
  几年前,有家报纸对纽约华人的情况作过详细的报道,其中有个荒谬的例子可以说明中国商业场中的猜疑。中国人在其他城市建立的机构大概也是大同小异。在纽约,他们有自已的市政府,有十二个中政领导。这些人把钱和中政府的文件锁在一个大保险柜中,为确保绝对安全,他们不用美国银行用的那种复杂、美观的号码锁,而是用十二把铜挂锁(中国式的)。每人只掌一把钥匙,要想打开保险柜,必须十二人全部到齐,每人开一把锁。不幸的是,一位杰出的高级中政官突然去世了,整个市政事务立刻陷入了极度混乱中,因为那位市政官的钥匙找不到了。即使找到,也没人敢代他开锁,人们相信死者会嫉妒他的继承人,让他也患上自己的那种病,这一迷信的观念太令他们恐惧了。直到经过特殊的选举补了空缺之后,市政府才取出钱支付丧葬费。这件小事,的确是一扇窗口,通过它,人们可以发现中国人的一些主要特征一一富有组织才能、商业才能,互相猜疑,极度的轻信以及对西方制度和文明不言而喻的蔑视。 
  中国的政府机构中也不乏相互猜疑的例子。宦官是亚洲典型的现象,中国古亦有之。但在目前的这个王朝,满族人采用了卓有成效的办法,使这批危险人物不再像过去那样有权力来危害人民了。 
  满人在中国是征服者,汉人是被征服者,双方在政府中难免相互猜疑,产生龃龉。六部长官及副职由哪族人来担任必须妥善安排,这样才能维持国家机器的平衡。检察院在很大程度上,也起到同等作用。 
  对那些熟悉中国政府内部情况的人,我们不能不承认他们说得对;中国人的普通社会生活中充满猜疑,官场也不例外。它不可能是另外一种情况,实际上乃中国人的本性使然。上级害怕下级竞争,时刻提防下级;下级的宦海浮沉又随时会受上级影响,他们也在猜疑上级。而整个官僚阶层又对强大的文人阶层和普通百姓存有戒备之意。中国有许多宗教团体是半政治性的,这就是后一种情况的显著表现。它们已使整个帝国变得像一团马蜂窝。地方政府禁止节欲者团体集会,比如有名的三星会,它只不过想禁止鸦片、烟和酒,打翻衙门里贪婪的“虎狼”的筵宴。他们并不企图谋反,可官府一直这样认为,他们也就只能如此。所有的秘密宗教都企图谋反,包括三星会,这样猜疑,会使事情变得容易处理。无论什么时候只要有异常情况,政府立即行动,把为首的抓住,或流放,或杀掉,恐惧就可缓解一阵子了。 
  强烈的猜疑使中国人变得十分保守。他们不了解人口调查,政府偶尔需要,也会因猜疑而不能实行,哪怕是名义上也不行,人们总是立刻怀疑调查是别有用心。笔者邻村发生的一件事,可以证明这种猜疑真实不虚。有兄弟二人,听说政府要进行新的人口调查,便断定这是强行迁民的预兆。按常规,迁民时,兄弟二人会留一人在家看守祖坟。弟弟料想自己很可能被征走,为了逃避长途跋涉的折磨,他立刻自杀了。这样就将了政府一军。 
  猜疑与保守,使中国青年从美国留学归来后,一直步履维艰,困难重重。它们也同样阻碍了中国对铁路的引进。中国所需要的改革也因政府的猜疑而长期遭到禁止。三十多年前,一位著名的政治家,听了铸造小银币之重要性的意见时,非常坦率地说:帝国的货币,永不可改革,“如果试图改变,百姓立即会认为政府想从中牟利。”事实上,也的确如此。 
  开矿也同样不可避免地遇到了巨大的阻力。如果可以成功的话,它会使中国变成一个富庶的国家。地下的“泥龙”,地上的猜疑和侵吞公款的行为,使得整个行动连第一步都难以迈出。无论新事物会带来多大的益处,益处有多明显,只要引起猜疑,就别想引进。已故的内文斯博士在烟台,为了将外国水果的优良品种引进中国,做了大量的工作,这些水果明显会带来巨大的收益。但他每前进一步,都要被迫同猜疑作斗争。缺乏善心或稍不耐烦,就可能早已取消这项计划了。不过,效益一旦得到确证,猜疑就会自然地渐渐消失。调查养蚕和种茶对帝国的海关非常实际,可是对此感兴趣的人们又怎么能违背过去的经验,认为这些调查不是为了征税,而是为了促进生产或提高技术劳动的收益呢?谁听说过这种事情?即便听说了,谁又会相信呢?古老的荷兰有句谚语可以形容中国人对这类事的态度:“狐狸跳进鹅毛笔管时,却说:‘各位早晨好’。” 
  下面我们将探讨一下这个问题与外国人间的特殊关系。中国人在强烈地不信任外国人时,还经常伴随着一种根深蒂固的观念:他们能够轻而易举地完成最难办的事。假如一个外国人在某个他从前很少去的地方散步,中国人就会认为他在察看风水;假如他凝视一条河,他就是在测定其中是否有金子。人们认为他能够看穿地表,发现最值得攫取的东西。如果他在赈济灾荒,人们就会认为他最终是想掠走大批当地人,到外国去做苦力。出于“风水”上的考虑,外国人经常被禁止到城墙上去,他们的建筑物也必须严格控制,像帝国的边界线一样明确。中国人似乎缺乏自然一致性的观念。巴伯先生曾提起四川某山区的一句谚语:外长罂粟内藏煤。这并不仅仅是一种无知的观念。帕普利教授说,北京的一位高级官员也告诉过他同样的话,并且在不知开采速度的情况下,把它作为反对过快开采煤矿的根据。己故的政治家文祥,曾读马丁博士的《基督教的证明》一书,当有人问起他的看法时,他回答,他准备接受该书科学的部分,但宗教部分,关于地球绕着太阳旋转的断言,则会令他难以置信。 
  外国人进入中国完全超出了他们目前的承受能力。梵·瑞恰斯芬男爵骑马在乡间游历,在四川人看来,完全是一种漫无目的的行为,因此把他想像成一个亡命之徒。很多中国人第一次见到外国人,会产生一种神秘的恐惧感,后来才发现这些野蛮人原来也都挺不错。许多中国妇女受到告诫,她们一旦进入外国人的住所,外国人就会念动致命的咒语,使她们着魔。如果她们最终被引诱进去,她们千万不能踩门槛或照镜子,否则会不安全。 
  几年以前,从内地某省来的一位年轻学者一一该省对外国人其实一无所知一一经过笔者的努力,答应帮一位新来的外国人学汉语。他在那人家里住了几星期后,想起他的母亲需要他的照顾,就回家去了。临行前,与笔者约定,某日赶回,可是,直到现在他也没回来。住在外国人家里的那段日子,这个聪明的孔门弟子,从未喝过一口茶或吃过一样东西,唯恐吃进了迷魂药。有一次,他写信给他的母亲,告诉她,自己一切都很好。另一个老师就送给他个外国信封,并且还告诉他,只要用舌头舔湿就可以封口。他却急中生智,温和地请那位老师帮他封口,因为他对此不在行。 
  中国人拒绝接受外同人印刷的汉语书,也是由这种观念导致的。人们普遍相信,书中放了迷魂药,油墨味就是它发出来的。药是在排版印刷时就掺进去了。有时,还会听到读了外国人的书,就会成为外国人的奴隶的传闻,据说有个小伙子对此不太相信,就读了一本小册子的开头,马上惊恐地把书扔掉,跑回家告诉他的朋友,假如谁读了那个书后,说了谎言,将来就会下地狱。小贩子也经常发现,这些书送都送不出去,并不是因为书中不为人知的内容受到敌视,而是因为人们担心送书者会以此进行敲诈勒索,这种做法在中国相当普遍。 
  如果外国人不慎重,试图记下一些孩子的名字,就会引起一片恐慌,而且也确曾使一所正在兴办的学校解散了。中国文字的罗马拼音体系一开始引入,就遭到了怀疑与排斥。为什么外国人希望教学生写一些他们的朋友读不懂的文字?世界上任何解释都不能消除中国老一辈人的疑心,他们认为,汉字一直很完美,对下一代也有好处。外国人连自己的祖先都不知道是谁,和他们的发明相比,中国人的汉字不知要好多少倍。几乎可以说,外国人的一切建议都会受到普遍的排斥。其原因很明显,就因为是外国人的建议。这种“顺而不从”的性格使你的中国朋友,以最温和而又最明确无误的方式让你确信,你的建议非常令人敬佩,不过,也非常荒谬。 
  讽刺是西方人手中的一种便利的武器,可它完全不适合中国人的口味。外国人对待中国人决不是根据自己的愿望或需要。有个外国人对仆人的失职和过错深恶痛绝,就用英语骂他“骗子”。仆人就向一位汉语很好的女士请教,当他得知这个用来说他的词的意思时,说他“受到了沉重的打击”。清朝的高级官员曾盗用汤姆先生《伊索寓言》泽文的版本,他们的思维模式和这个仆人一模一样。不过,其中会说话的鹅、老虎、狐狸和狮子不能不令他们想到背后隐晦的含义,为防患于未然,他们便查禁了所有的版本。 
  对外国人最顽固的不信任的例子可以在遍布大部分中国的医院和诊所里发现。在人数众多的患者中,许多人对外国医生的善良和医术表现出不言而喻的诚意和令人感动的信任。但也有不少人,仍相信荒谬的谎言,说外国人用人的眼睛和心脏做药,外科医生有将人切成碎肉的嗜好,外国人还将中国儿童藏在地窖中,进行可怕的处理等等。除了小心探问,这些人的感情我们一般了解得很少。一、两年之后,这些机构的广泛成功可望能够像大风吹散尘土一样,驱散所有的这些无稽之谈。不过,它们一有机会,就会疯狂地蔓延,像温暖潮湿八月中的霉菌。 
  虽然在中外关系史上,外国人存在着严重失误,但整个历史是中国人怀疑与搪塞的历史。这是一段令人厌倦的回忆,其间的教训完全是由徒劳无功的交往者造成的。但在中国,私人常常被迫充当外交家,他们都很清楚应该怎样做,我们可以举一个典型的例子进行说明。一个外国人提出要在中国内地某城市租赁一处住所,当地官员则提出种种借口加以搪塞,在一次安排好的会见中,这个外国人身着中国服装,带着纸和笔,到了见面地点。初步交谈之后,他慢慢地取出文具,摆好纸,拧开笔,检查一下墨水,神情严肃专注。中国官员怀着强烈的兴趣看他做完这一切,好奇地问:“你在做什么?”外国人解释道,他只是准备好文具,“仅此而已,没有什么。”“文具?准备文具干什么?”“记下你的答复。”这位官员急忙向外国客人保证,住所一定会解决,这样做完全没有必要。可他下次再听到这个神秘的记录时,怎么能保证承认,其中的内容都是他说的? 
  中国是个谣言泛滥的国家,它们经常使人心中充满恐惧。几年前,新加坡的中国人报告说,苦力们坚决拒绝天黑以后走某一条街道,因为在那里会突然神秘地被砍去脑袋。帝国可能永远也不能从恐惧中解脱出来了,对于有关的人,这些恐惧就像1789年法国革命中的巴黎人感到的一样真切。无节制的轻信和相互猜疑是恐怖的谣言产生、弥漫的沃土。当它们与外国人有关时,痛苦的经验表明,绝不可掉以轻心,在它们刚一产生时,就应该查清。如果当地官员认真查处,就不会导致严重的后果。如果不进行制止,任其漫延,就会产生像天津大屠杀那样的惨剧。整个中国都适宜于谣言的迅速传播,几乎没有一个省没有产生过谣言。为了彻底铲除谣言,时间就应像地质新纪元一样被视为至关重要的因素,最好的办法,是以勿容置疑的实例,使中国人相信,外国人是他们真诚的祝愿者,一旦树立了这一坚定信念,“四海之内皆兄弟”将会在人类史上第一次成为现实。


\chapter{缺乏诚信}
“信”,英语里一般译为“sincerity”;在汉字中,它是个会意字,由“人”和“言”两部分组成,其意义也是这两部分字面所表达的。“五常”中,它位列最后。许多了解中国的人认为,“信”在天朝上邦,事实上可能是最罕见的美德。他们也将会同意基德教授的看法。基德教授在谈了中国人“信”的观念之后,接着又说:“如果在民族性格中有一种美德,不仅在行动中受到蔑视,而且也和现有的处世态度形成强烈的反差,这一特征非信莫属。中国人公开的和私下里的表现,都与信背道而驰,他们的敌人也以此讽刺他们,虚伪矫饰,欺骗、不真诚和趋炎附势是这个民族的显著特征。”这种评价多大程度上符合事实,我们最好在详细地考察了下面的事例后再作判断。 
  我们完全有理由认为,现代中国人和古代中国人没有多少差异,而且我们还深信,有资格的学者也会支持这一观点。在信的标准上,中国人不同于西方人。一些思想敏锐的学者,在仔细推敲中国的古代经典时,会从字里行间发现很多拐弯抹角、含糊其辞的地方。他还会发现,对西方人的直率,中国人有句很有意味的话:“直率而无分寸就成了无礼。”《论语》中孺悲与孔子的故事,西方人觉得意味深长,而儒生们却一点儿也不理解。下面一段选自莱格的译文:“孺悲想拜见孔子,孔子托辞有病,谢绝见他。但传话人一出房门,孔子便取下瑟,边弹边唱,故意让孺悲听见。”孔子不想接见孺悲这样的人,便以中国的方式来解决。 
  孔子的做法后来为孟子所仿效。孟子曾在某国作为客人被邀请上朝,但他希望国王能给他以第一次召见的荣誉,因此托病不出。第二天,为表明这只是个借口,便在别处觐见国王。陪伴孟子的官员,夜里与孟子就孔子的上述行为,进行了一次长谈,但讨论只局限在礼节惯例方面,没有涉及到为方便而撒谎的道德问题,也没有任何证据表明有人思考过这一问题,现代的孔门弟子在给学生解释这一段时,也没有超出上面的讨论。 
  在保存典籍的本能方面,古代中国人远远高出许多国家的当代人。他们历史虽然冗长;但包罗万象。很多西方学者似乎最推崇中国的历史,言辞中常流露出过分的信任。维也纳大学教师基·辛格博士1788年7月在《中国评论》上发表了一篇论文,其中有这样一段:“科学考据很早就认识到,并且越来越证明了中国古典文献的历史真实性。”例如,最新一代中最广博的中国研究者——瑞恰斯芬,在讨论中国人性格中惊人的矛盾成分时,发现一方面他们在统计记录历史事件时,具有忠实精神和探索真理的强烈愿望,另一方面在日常生活和外交谈判中处处充满谎言与欺骗,此二者形成了鲜明的对比。精确地记录历史有两种不同的途径:一,按特定的顺序和比重叙述事件;二、根据一定的精神和动机分析。阐释事件。一些广泛地研究了中国历史的人认为,从前者看,这些历史著作无疑大大地超出了撰写的时代;从后者看,它们绝没有辛格博士所认为的谨严。对不了解的事物,我们不发表意见,只是想让人们注意,一个民族沉溺于谎言,同时又能培养出尊重事实的史官,即使不是史无前例,也是独一无二的。强烈的爱或恨扭曲其他国家的历史,在中国,难道它们就不起作用吗?在世界其他地区发挥作用的因素难道在中国会失效吗? 
  不仅儒家思想本身存在较大缺陷,而孔大圣人也不严格尊重史实。莱格博士并不紧盯着“圣人生平的暇疵”不放,而是重点研究孔子编篡《春秋》时处理历史材料的方式。这部著作记录了鲁国二百三十四年的历史,向后延续到孔子死后两年。下面一段引自莱格博士有关儒教的演讲,发表在他的多卷本《中国宗教》中:“孟子把《春秋》视为孔子最伟大的成就,说它的问世使乱臣贼子惧。作者自己也说过同样的话,并说世人因此了解他,也因此毁谤他。”但是当孔子谈到世人因此毁谤他时,不知他心里是否充满了疑虑。事实上,这部书不仅极为简约,而且含糊其辞,具有欺骗性。《春秋》问世后,不足百年,公羊便对之作了修正与补正,说《春秋》“为尊者讳,为长者讳,为贤者讳”。我在《中国经典》第五卷中指出,“讳”包含三种含义——省略,掩盖和篡改。对此,我们能说什么呢?……我常常想快刀斩乱麻,干脆否认《春秋》的真诚性和真实性。但是孔子生活在他记录的那个时代,他把历史与自己的笔法紧密地联系在一起了,如果一个外国学习者采用曲解的办法,使他看不到大圣人不尊重史实的缺点,中国的统治者和大多数学者可不会怜悯他,也不会同情他的苦恼。孔子及其弟子一直倡导真实性,但《春秋》使他们的同胞在可能损及帝国或圣人名誉的情况下,学会了掩盖真相。 
  我们已经看出,宣称中国历史真实的人只准备承认,在中国,真实仅存在历史的记述中。当然,不可能证明每一个中国人都撒谎。即使有可能,我们也不愿那样做。等到中国人的良知苏醒,开始关注自己的信义时,自己会提供最有力的证据。他们在谈论自己的民族时,我们经常可以听到,像海南岛首领所说的:“我们一开口,谎言就诞生。”可是,对我们来说,中国人并不像一些人认为的,是为撒谎而撒谎,撒谎是为了获得谎言之外的某种利益。巴伯先生说:“他们不说真话,同样也不相信真话。”一位学过英语的中国小伙子在拜访笔者的朋友时,为增加词汇量,希望学会说“你撒谎”的英语表述方式。我的朋友就告诉他,这句话最好别用来说外国人,否则,肯定会挨揍。小伙子毫不掩饰地对此表示惊讶,他觉得这句话就像说“你骗人”一样,不会伤害人。库克先生,1857年在作伦敦《泰晤士报》驻中国记者时,谈到西方人最讨厌被称为说谎者,“但是,如果你对中国人说同样的话,他一点儿也不会气恼,也不会感到受了侮辱。他不否认事实,只回答道:‘我可不敢对阁下撒谎’。说一个中国人‘撒谎成性,眼下正在撒谎’,就像对英国人说,‘你这家伙就爱说俏皮话,我保证现在你脑袋里装满了糟透了的俏皮话’。” 
  中国人平时说话缺乏诚信,虽未达到作伪的程度,但他们所说的每一件,几乎都不是真相,真相在中国是最难获得的,谁都不敢保证,自己获得了事实的全部真相。即使有人寻求你的帮助,比如打官司,他希望你全权代理,你仍会发现,他向你隐瞒了重要的事实。这显然是支吾搪塞的本能所致,而非蓄意如此,尽管这样做,受害者只能是他本人。无论你从何处着手处理,整个事情一直要到最后才会显露出来。较为了解中国的人不会听了一方陈述就觉得掌握了全部情况,他宁愿把听到的和其他情况结合起来,最后找来几位他最信任的人,就那些陈述再调查一番,才判断事情的真相。 
  缺乏诚信,再加上猜疑,就足以解释为什么中国人经常交谈了很长时间,却没有谈出任何实质性的内容,对外国人来说,中国人不可理喻,主要归咎于他们虚伪。我们不知道他们在希求什么,但总觉得他们的言谈背后隐藏着更多的东西。因此,当一个中国人走到你跟前,贴在你耳边,神秘地告诉你一个你感兴趣的中国人的事,你不可能不心头一沉。你不能确定他是在说事实,还是在诬陷那人。你也从来不能保证中国人的最后通牒真的就是最后的。对于生意人、旅行家、外交官来说,这个很容易阐释的命题,包含着诸多令人烦恼的因素。 
  所有事情的真正原因几乎都难以预料,即便知道,也不能确保是事实。每一个中国人,即使没受过教育,其本性也像一头狡猾的乌贼,受到追踪时,立刻能喷出大量的墨汁,使自己退到最安全的地方。如果你在旅途中,受到拜访,请求捐款给一些穷人,他们希望开发新的土地,你的仆人不会像你一样,干脆说:“你花钱不关我的事,随你的便。”而是“面带孩子般的笑容”解释道,你袋里的钱只够你自己用的。这样,你就无法捐款了。我们也很少发现某个看门的人,会像外国人对待他那样,对一群中国人说:“这儿你不能进。”他只是在一旁悄悄地看着,等他们一进去,他就放狗。 
  中国人能自觉守约者,寥寥无几。这与他们误解的天赋、淡薄的时间观念有关。不管失约的真正原因是什么,你将有趣地发现他们会寻找各种各样的借口。一般,中国人被指责爽约时,会说道,这个约会无足轻重,重要的约会,他总会守约。如果谴责他的某个缺点,发誓改正的话就会像流水一样从他嘴里喷泻而出。他承认错误很全面——实际上是太全面了,除了信用之外,你再没别的可期待了。 
  一位中国先生,曾被雇来抄写,注释一些格言。在一些古老的警句之后,他解释道,不能马上拒绝别人的请求,相反,即使实际上不想帮忙,也要表面上答应。“拖到明天,接着,再一个明天,这样,请求者心里会得到安慰。”负债的人一般也采用这种方法。谁也别指望一次就可把债讨回,要债者也不会因此失望,欠债者会信誓旦旦地说,下一次还。然后再下一次,再下一次。 
  最能说明中国人虚伪的,是他们对待孩子的态度。孩子们从小就学会不诚实,而且无论孩子本人,还是施教者竟都没有意识到这一点。孩子还在牙牙学语、朦朦胧胧懂话的时候,大人就告诉他,要是不听话,藏在大人袖子里的怪物就会出来咬他。外国人也常被比做未知的怪物,这也能较好地说明为什么中国人经常对我们说脏话。孩子们很小就对我们怀有模糊的恐惧感,长大后,一旦意识到我们并不可怕,只是可笑而已,怎么会不在街上哄赶我们呢? 
  车夫拉着外国人,后面跟着一群高声喊叫的顽童。他被激怒了,向他们吼道,他要捉几个,绑在车后面拖死。船夫遇到这种情况,也会吓唬道,用开水浇他们。“我揍你”、“我砍死你”这类话,对懂点事的孩子来说,就等于“别那样做”。 
  中国人要想装得“懂礼”,必须掌握一大套词汇,他们能表现出说话者的谦卑,听话者的高贵。“懂礼”的人提到自己的妻子,如果必须称呼,就说“拙荆”,或其他类似的文雅的谦称。农村人,虽然不会文雅的辞令,也能抓住“礼”的精髓,称和自己患难与共的伴侣为“臭婆娘”。中国人自己的一个故事,可以恰当地说明他们注意礼节的特征。一位拜访者身穿最好的礼服,坐在客厅里等候主人的出现。一只老鼠正在梁上嬉戏,把鼻子伸进梁上的油罐中,客人的突然到来吓了老鼠一跳,它转身就逃,结果碰翻了油罐,正打在客人的身上,华丽的外衣立刻沾满了油污。正当客人气得脸色发青时,主人进来了。一阵寒喧之后,客人解释道:“鄙人来到贵舍,坐于贵梁之下,不慎惊动贵鼠,贵鼠走,贵油罐落于鄙人寒服之上,狼狈之极实令足下见笑。” 
  不用说,很少有外国人能以中国人的方式招待中国人,这需要长期的锻炼。主人走向宴席时,要热情地向一群客人鞠躬,温和地招呼:“诸位请坐,请用膳。”或把茶杯举到唇边,环视左右,郑重地对客人说:“诸位请用茶。”更令人难以接受的是在不同场合的“叩头”、“叩头”,来表示“我能、我会、我可以、我必须、我应该”(视情况而定)“拜倒在你脚下”。偶尔还会插入这样的话:“我该打,我该死”,意思是礼数不周,忽略了小细节。或者,骑着马,中途遇到熟人,就停下来,对他说:“我下来,你骑吧。”一点也不考虑你往哪儿走,或他的做法是否符合情理。即使是最没教养、最无知的中国人也会经常摆出邀请的姿态,迫使最无同情心的西方人无意识中对此赞叹不已,因为别人会,而他不会。我们在各种场合不断看到的小小的礼仪,是个人对整个社会的奉献,它使得社会摩擦减少了。如果拒绝作出这种奉献,就会遭到惩罚,因为他走上了歧途。车夫停下来问路,假如忘了取下辫子,很可能会被指错路,另外,还可能会遭到辱骂。 
  判断送什么礼物最合适,在东方,这也是一门学问,其他国家可能也如此。对于收礼物的人,有些东西,绝对不能接受,而另外一些东西则不能全部接受。假如外国人在这方面自作主张,一定会做错事。一般情况下,有人送礼,要慎重对待,特别是在出乎意料的情况下。即使是生儿子这样的喜事,也要小心,“我怕希腊人,即使他带着礼物”,这句格言在世界各地都具有永久的生命力,在中国也一样。送礼背后总有文章,像中国歇后语说的“老鼠拉木锨,大头在后面”,或者,换句话说(实质上是),要求的回报要比送的多。 
  许多居住在中国的外国人,对送礼的虚伪性,都有所体会,我们曾有幸熟悉了送礼的全部细节。为了对几个外国人表示尊敬,一个小村庄为他们搭台唱戏,当然,谁都明白,外国人应该设宴回敬。可是村民们对此执意拒绝,请求他们捐一笔款,哪怕是一点点也行,可以用于公共设施的建设。在这个村子,他们照做了。此后不久,又有十“一个村子,说是被外国人救济灾荒和医疗救助的精神深深地感动,接二连三地派代表请他们去看戏。这些村子都清楚,邀请肯定会被拒绝。每个代表听到被拒绝的消息时,脸上都露出同样悲哀的惊愕神情,然后又全部转向捐款问题,仍然是公共设施。他们每个人都是点到为止,没有再作进一步的表示。 
  不单单是外国人在这方面受到困扰。富有的中国人不幸遇到喜事时,邻居就会拿着一点儿不值得一提的礼物前来祝贺,比如为新生婴儿买的不值钱的玩具,但是主人必须设宴答谢——在中国,这是一种永远合乎时宜的方式。这时,即使最不了解中国的人,也会赞叹中国格言的精妙:“吃自己的,吃出泪水;吃别人的,吃出汗水。”主人还要被迫装出一副真诚欢迎的样子。为了不丢“面子”,满腔怒火全都压抑在肚子里,丢“面子”可比损失食物更要命。 
  这表明,中国人有许多行为都是为了“讲面子”有意做出来的。在受雇时,中国人对待外国人的大部分礼节,只是一种虚伪的客套,尤其是在大城市中,将一个人在公共场合和私下里的行为比较一下,很容易发现这一点。据说,有一位中国先生,在他的外国主人家里,向来被奉为遵守礼节的楷模。可假如他在北京街头遇到主人,就会怒目而视,好像要“杀掉他”,因为倘若和主人打招呼,就会让别人看出,这位饱读诗书的先生在某种程度上要依靠野蛮的外国人来混碗饭吃——尽管这情形已是众所周知,但在表面上,尤其在公共场合不能承认。几个中国人进屋时,只给屋里的中国人行礼,完全无视其中外国人的存在,是很正常的事。中国老师会称赞他的外国学生听力准确,发音完美,在接受语言方面会很快超过其他同学。可同时,这位学生的一些奇怪的发音错误,却会成为这位老师与他同事间的笑料。一般情况下,雇来教汉语的老师理所当然被视为最有权决定他汉语语音正确性的人。 
  另一个可以说明中国礼貌的表面性与虚伪性的现象,是口头上应承,而实际上不做。送礼没有带来预期的好处,送礼者也不灰心丧气,因为早就知道事情可能办不成,但送礼者的面子保住了。同样,假如在酒馆里,你和老板在付款上发生了争执,你的车夫可能会站出来调停,决定不足的部分由他来付,然后将手伸进你的钱袋。如果他掏自己的钱,最后账仍会算在你头上。倘若提醒是他自己要付的,他会说:“你能指望参加葬礼的人一同被埋进坟墓吗?” 
  尽管表里不一,中国仍有许多人是真正谦逊的,不过,无论男女,肯定也有不少人的谦逊是假的。当人们清醒地意识到,某些观念难以直接表达时,他们就谈论一些不愉快的事间接地来表达。可这些谈吐优雅的人,一旦被激怒,连最难听的话都骂得出。 
  虚假的谦逊与虚伪的同情同是由空话组成。但是,中国人不应因此受到谴责,因为他们没有足够的财富长期维持对如此众多人的同情。最令人恶心的倒不是空洞的同情,而是对死者假装同情时,又流露出兴高采烈的神情。巴伯先生提到四川的一个苦力,看到两条野狗在纤道上吞吃死尸,竟止不住哈哈大笑。梅杜斯先生告诉我们,他的汉语老师听到自己的好朋友死得很有趣,竟也捧腹大笑。最疼爱的孩子夭折,长时间的悲痛会使父母变得表情麻木,这与上面的情形不同,因为沉默的悲痛和违背人性、对自己自然感情的粗鲁愚弄之间有着巨大的差异。 
  如前所述,西方人和中国人实行贸易往来已有数百年的历史了,在此过程中,中国人的商业信誉也多次得到验证。为不使结论显得有如空穴来风,应该举些例证作基础。下面就是一个范例。香港和上海银行经理卡麦伦先生在他的告别发言中,有这样一段:“我已谈了西方商团的高标准的贸易原则,在这方面,中国人一点也不落后。实际上,没有谁能像中国商人和银行家那样,很快赢得我们的信任。当然,任何事情都有例外。但为了表明我有足够的理由得出这一有力的结论,我可以告诉诸位,过去二十五年内,我们的银行在上海与中国人做了很大一笔生意,总额达数亿两白银,迄今为止,我们还未遇到过不守信用的中国人”。对这段发言最好的评价可能是三年后发生的另一件事:这家银行在香港的一位买办,使银行在蒙受损失,没有保障的情况下,仍能每年赚取一百万元的利润。 
  中国商业活动中的零售与整卖是否有本质区别,我们无从得知。但为了使我们的思考显得更全面,确实应该考虑一下,上述结果是否完全没有中国人令人赞叹的责任感的作用——这是一个西方人应该好好学习的优点,中国人在与西方人做生意时,讲信用可以使他们获得最大限度的利润,所以,我们获得中国人毫无疑问的最大限度的信任,本来就是自然而然的事。尽管如此,长期广泛的观察仍只能证明:中国人的商业活动是这个民族缺乏信用的最大例证。 
  一位聪明的学者,写过一篇很有意思的论文,论述中国人的普通商业活动只是一方欺骗另一方的活动。这两个人之间的关系,一般来说,就是雅各布和拉班之间的关系,或者像中国说的,是铁刷遇铜盆。没有谁不知道,让一个孩子做生意,其实就等于毁了他。假秤、假尺、假钱和假货——所有的这些现象在中国都难以避免。即使一些大字号,挂着醒目的招牌,告诉顾客,本店“货真价实”,“绝无二价”,实际上绝不是这么一回事。 
  我们无意于表明中国无诚实,不过,根据我们的观察和经验,很难保证可以找到。和一个不重视事实的民族交往,还会出现别的情况吗?一个衣冠楚楚的学者,大言不惭地告诉外国人,他不识字。可如果递给他一本小书,他会毫不迟疑地悄悄地从人群中溜走,钱都不付,虽然那本小书至多只值三个铜板。对此,他一点儿也不觉得羞耻,反倒沾沾自喜,把愚蠢的外国人骗了,那家伙竟然相信一个完全陌生的人。中国人向外国人买东西,经常少付一个铜钱。他宣称,身上没钱了。如果你告诉他,,他的耳朵上正夹着一枚铜钱呢,他会极不情愿地取下来交给你,那情形就跟受了骗似的。同样,一个人会磨蹭“老半天”,想免费从你那儿得到点东西,理由是他一个钱也没有。可是最后却会取出一大串铜钱,满脸不高兴地递给你,叮嘱你只取他应付的钱。假如你相信了他,让他不付钱就把东西拿走,他会心花怒放,就像杀死了一条蛇。 
  中国人一向有向亲戚“借东西”的习惯,而且总是有意无意地不打招呼,这大概也是社会团结的一种表现吧。“借”来的东西大部分被立即送进了当铺,主人想要时,必须自己拿钱去取。教会学校的一个中国男孩,在偷一个管学生宿舍的单身女士的钱时,被发现了。在不容置辩的证据面前,他抽抽搭搭地解释说,在家时,他一直习惯于偷妈妈的钱,而这位外国老师太像他的妈妈了,于是,他不由地想偷一偷。 
  中国社会明显存在许多邪恶,西方无疑也存在,但最重要的是,要清醒地意识到两者之间的本质区别。我们前面说过,中国人缺乏信义,其事虽不常见,但经常可以找到。一些例子,在我们讨论其他论题时,已经列举了,还有一些应作详细的论述。 
  要是具备必要的知识,可以就中国人的敲诈勒索写一套非常有趣的书——上至龙椅上的皇帝,下至最卑贱的乞丐,人人都那样干。中国人具有注重实际的智慧,他们惜以使敲诈勒索形成一个完备的行为体系,每个人就像离不开大气层一样,脱离不了这一天罗地网的笼罩。它是如此恶毒,堕落,除非对整个帝国进行彻底整顿,才能将其铲除。 
  中国人的性格,以及中国的现状,必然导致西方人很难以务实的态度在最大范围内同中国人交往,同时还能保住“上等人”的名誉——假如他有幸获得这一名誉。人们经常说,车夫、船夫、酒馆老板、苦力、买卖中间人,不论犯什么罪,按常规,一律杀头。他们,以及与他们地位相当的人,和外国人间的关系很特殊。因为外国人宁愿忍受欺诈,也不愿引起社会风波。这方面,他们一般既没有兴趣也没有能力。然而中国人内部如果破坏了公正原则,却只有通过社会风波才能使社会最终达到平衡。 
  一个人难以做到不偏不倚时,努力做到这一点,他一定是个非凡的人物。既不猜疑,又不轻信,是中庸之道最完美的体现。如果我们对似乎必需的不诚实表示不满,敏于判断人的性格的中国人,就会把我们归入“性情人物”那一类。佛的涅盘境界,对于易激动的人来说,想时刻保持,并不容易,即使我们能够保持这种宁静的品性,也会被当成被进一步任意勒索的最佳对象。有一个典型的中国人,受雇为外国人做事。有一天,看见一个小贩沿街叫卖泥捏的外国小人,那些小人造型精巧,服饰得体。他就停下来,看了一会儿,对小贩说:“啊,你玩的不过是玩具,我玩弄的可是真家伙。” 
  勿需赘言,就我们所知,中国政府似乎是我们正在讨论的这一特点的重要例证。在整个中外关系史上,以及有名的中国官僚与民众的关系史上,也都可以找到这种例子。各级官员经常颁布的文告,就是一个独特、简明的例证。这些文告篇幅冗长,文辞华美,内容繁杂,表现出高尚的道德境界。唯一缺少的就是真实,因为这些华美的命令并不准备让人们去执行。这一点,写的人和看的人都很清楚,从不会发生误解。“中国政客的生平和公文,就像卢梭的《忏悔录》,情感高尚,而行为卑鄙。他砍下十万颗头颅,却引用孟子的话论述生命的神圣。他把修筑堤坝的钱塞进自己的腰包,导致河水淹没一个省,却为人民背井离乡而哀叹。他高声痛斥发假誓的人,却在签定一项协议后,私下里说,那不过是一时骗人的玩艺儿。”勿容置疑,中国也有公正无私的官员,不过很难找到而已,而且,他们的生活环境,使他们处在孤立无援的境地,无法如愿以偿。把最有机会了解中国经典的人的处境和这些经典的教义比较一下,明显可以看出,他们在引导社会走向更高一种境界时,是多么的无能为力。 
  “你知道多少值得信任的中国人?”这里仅指受过正规教育的中国人。不同的人,有着不同的经验和评价中国人的标准,因此回答也千差万别。大多数外国人会回答:“很少”,“七、八个”,“一打”,视情况而定。有时,也有人回答:“很多”,“多得记不清”,可是,我们深信,在有见识和辨别能力的人当中,做出这种回答的肯定极少。 
  观察被一个民族视为理所当然的事,是一种富有智慧的做法。在探讨中国人相互猜疑的特征时,我们已经看出,中国人把不信任别人看成很自然的事,其理由他们心里都很明白。这种状况使得中国的前途充满了不确定性。这个民族不是由精英分子来统治,相反,掌握全部权力的是帝国中最卑鄙、无耻的家伙。一位聪明的道台,对外国人说:“皇帝以下的所有官员都是坏蛋,全该杀掉,但是杀了我们没有用,下一任仍会和我们一样坏。”中国谚语说,蛇知道自己的窟在哪儿。另一个很有意味的现象是,中国的官僚阶层受到商人阶层的极度不信任。他们知道,所谓的“改革”,不过是一层表面的外壳,不久,就会脱落。一个中国的泥瓦匠,花了很长时间,用没调和好的灰浆抹平盖得很糟的烟囱与屋顶,可他心里很清楚,第一次生火,烟囱会四处冒烟;第一次下雨,屋顶会漏水。在中国,这不过是一桩极平常的事。 
  中国有足够的实力开发各处的资源,只要有信心,羞怯的资本就会从隐蔽的地方走出来。在中国,开发资源所需要的各种知识都十分丰富,各类人才应有尽有。但是,假如没有建立在真诚基础上的彼此信任,这一切都不足以使帝国复兴。 
  几年前,一位善于思考的中国人来向笔者请教,如何解决某个地区打井难的问题。中国人打井,一般是井内从上至下都用砖头砌上。可在当地,井打好后,过了一段时间,整个地面就会下沉,井壁也随之坍塌,只剩一个小洞,井也渐渐地干枯了。治疗中国长期忍受的病痛,与对直隶省的这个不幸地区的补救相同,一切药方都难以将其根治。所有的治疗都是表面的,整个帝国最终只能像满载珠宝的大车陷进绝望的泥潭。 

\chapter{多元信仰}
儒教,作为一种思想体系,是中华民族最伟大的智力成就之一,而儒家经典对于西方读者来说,又十分枯燥乏味。不过,仅仅仔细地阅读,只能得到皮毛的印象,不思考其影响,简直永远不可能了解它们。中华民族是世界上最伟大的民族,“其有记载的历史一直可以追溯到传说中的远古,她是世界上唯一没有异化或崩溃的古老民族,也从未被任何民族,从她自古生存的那片土地上驱逐出去。”一切都仍是那样古老。对这一空前绝后的事实,我们该如何解释呢?中国人口之众多,在世界上无与伦比,他们自从开大辟地以来就居住在中华大地上,直到今天。到底是一种什么样的神秘力量在支撑着这个古老的民族?在所有的民族都必然走向衰落、灭亡的宇宙命运面前,中华民族为什么成了一个例外,一直保持着如此顽强的生命力? 
  所有对此作过彻底调查的学者一致认为,其他民族依靠物质力量生存,而中华民族依靠的是道德力量。一位学习历史的人或善于观察的旅行家,只要了解人的本性,无不对中华民族奇迹般的道德约束力肃然起敬,这种约束力从古至今都发挥着巨大的作用。威廉姆斯博士说:“儒教对中华民族在追求理想人格,善良人性方面的影响,无论作何评价,都不过分,它所描绘的极高的道德标准对后世产生了不可估量的影响,以至于整个民族都要接受这一标准的评判。”莱格博士说:“儒教在作为人的责任方面的精彩教诲,实在令人赞叹,它所推崇的四个方面的教诲——文质彬彬、谨守道德律令、关注精神、诚实,其中有三个方面是和摩西律法及福音书教诲是一致的,以此为标准建立的世界,必将是一个美好的世界。” 
  中国经典中,完全没有使人堕落的描写,这一点经常有人指出,它是中国经典最伟大的特征之一,也是与印度、希腊、罗马经典最主要的区别之一。梅杜斯先生说:“无论是古代的,还是现代的,没有任何一个民族拥有如此圣洁庄严的作品,里面完全没有放荡的描写和粗鲁的语言。整部书没有一句话或注释不能在英国任何一个家庭中大声朗诵。在其他所有非基督教国家,偶像崇拜都是与人的牺牲及堕落的神圣化联系在一起,并伴随着狂欢放荡的仪式。可在中国,所有的这一切,都找不到半点踪影。” 
  皇帝就自己的统治直接向上大负责;民心振奋要比统治者的精神更为重要;统治者应该德才兼备,其统治也应该以美德为基础;人与人之间五种关系的复杂理论;己所不欲,勿施于人——所有这些观念像山峰一样,从普通的中国思想中脱颖而出,也吸引了所有观察家的注意。在即将结束对中国人的论述时,我们想重点强调一下儒家思想体系的优点,只有真正理解了这些优点,我们才能真正地理解中国人。它们使中国人具备了一种服从道德的卓绝能力。每年的文官考试,都是就这些经典出题,因此,中国人的思想统一,也达到了不可思议的地步,每一位候选者都把政府的稳定当成自己成功的前提,这无疑就是中华民族繁衍至今的首要因素。 
  中国人是否确实信仰过上帝,一直是个相当令人感兴趣的问题。那些严格考察过中国典籍的人向我们保证,学者们倾向于肯定回答。而另一些自称具有独立判断能力的人则持否定态度。即使中国人确曾认识到真正的上帝,那些观念也全被遗忘了,犹如一枚古币,上面的文字早已被积存的斑斑锈迹遮盖了。对于提问者,这一问题可能非常重要,但对我们目前的研究来说,完全可以不考虑。我们目前所关注的既不是历史问题,也不是理论问题,而是现实问题,也就是说,中国人和他们的神之间到底存在着一种什么样的关系? 
  通过一些实例,我们不难追溯古代英雄和杰出人物从受尊敬到被纪念、再到被崇拜的各个阶段。中国所有的神几乎都是死去的人,祖先崇拜,在某种意义上,可以证明所有的死人都是神,在皇帝的恩准下,各地不断为生前就闻名遐迩的人物建庙立祠。随着时间的流逝,很难说其中没有人会成为整个民族的神,不管怎样,作为一个民族,中国人是多神论者。 
  认为人都有崇拜自然的倾向,这只是陈辞滥调。人们认为那些不可抗拒的未知力量有感觉,因而便把它们拟人化,并加以崇拜,所以风神庙、雷神庙等等随处可见。北极星也是人们长期崇拜的对象。北京还有与皇帝崇拜有关的日坛与月坛。有些地方,对太阳的崇拜成了一种有规律的朝拜。二月的某一天被定为太阳的“生日”。这天一大早,村民们就起身东行,去迎接太阳;傍晚时分,再转身向西,护送太阳踏上归程。一年中对太阳的朝拜这时就算结束了。 
  自然崇拜中最平常的是崇拜树,在某些省(例如河南的西北部),经常可以见到大小几百棵树,都挂着小旗子,标明此树为何神居所。有时即使没有这种外在的标志,人们仍会笃信不疑。如果破旧的草屋前长着一棵遒劲的老树,简直可以肯定,树的主人一定不敢砍伐,因为中间住着神灵。 
  皇帝经常被认为是唯一有权祭天的人。祭祀大典独特而有趣,由皇帝一人独自在大坛上举行。但对于全体中国人来说,他们自己不祭把大地,也是一个新闻。每家朝南的正墙上都设有一个小小的神龛,有些地方称为天地龛。大多数中国人证实,他们举行的祭把活动(祖先崇拜除外)只在初一或十五举行,向大地祭拜或上供,有时是在新年伊始。祭祀时,没有祷告,过一会儿,供品就被撤下,或像其他祭祀一样,全部吃掉。这时,人们祭祀的是什么呢?有时,他们说是“天和地”,有时又说是“天”,也称为“老天爷”。后一种称呼经常使人们认为,中国人确实感觉到一个人格神。可是当你知道,这个假想的“存在物”经常与另一个被称为“土地婆”的神相匹配时,这个推断就带来了严重的问题。有些地方,有六有十九祭祀“老天爷”的风俗,因为这一天是他的生日。向给“老天爷”定生日的人追问:“老天爷”的父亲是谁,他的生辰八字是多少,完全是多余的,因为他们对此也一无所知。很难使一个普通的中国人理解这些问题的实际意义。他只是接受传统,做梦也不想提这些乱七八糟的问题。我们很少遇到一个中同人,除了知道“老天爷”掌管”气候和收成之外,还能知道他的身世与品质。“老天爷”这个同在中国广泛流传,似乎表明他具有人格,但是,就我们所知,人们既没有为他建庙,也没有为他塑像,对他的祭祀和对“天与地”的祭祀也没什么不同,似乎都是未经解释而保留下来的。 
  中国典籍中经常用“天”这个词来表示人的观念和意志,但有时这两方面的涵义都没有。当我们看“天即道”的注解时,感到其意义已模糊到了极点。这个词在古代富有歧义,在日常生活中也一直非常含混。一个一向祭天的人,被强烈要求回答“天”是何意时,他经常回答说,就是头上那蓝蓝的天空。这表明,他的崇拜是与崇拜自然力量相一致的,只不过有的崇拜个体,有的崇拜群体而已。他们所信仰的,用爱默生的话说,是“萧萧细雨,萋萋芳草”,他们是泛神论者。这种缺乏明确涵义的人格化,正是中国“苍天”崇拜的致命缺陷。 
  中国上层社会流行的似乎纯粹是无神论,这与下层百姓的多神论、泛神论形成了鲜明的对比。从那些对此颇有研究的人提供的证据中,从无数的表面现象和“前定可能性”中,我们不能不得出如下结论:世界上没有任何一个有教养的、文明人团体像儒家学者那样是彻底的不可知论者和无神论者。* “前定可能性”指的是宋代唯物主义注释家对知识界的著名影响,中国经典的注释家、大学问家朱熹则是绝对的权威,任何对他的观念的怀疑都被视为异端邪说。他的注释不仅是唯物主义的,而且就我们理解,也是彻头彻尾的无神论,它的影响遮蔽了原有经典的教诲。 
  黄河从陕西和山西的深山峻岭中流出后,继续向东奔流数百里,注入大海。多少年来,它曾数次改道,跨越六、七个纬度,从扬子江口流至渤海口。但它流经哪里,哪里哀鸿遍野,留下一片不毛的沙丘。宋代的注释家带来的唯物主义洪流冲击着中国思想的大河,情形与黄河别无 
  * 梅杜斯先生指出,任何一位思想前后一贯的儒家学者都是一个彻底的无神论者,但人类的本性很少有绝对的。一致性,不少儒家学者也信神,或者以为自己信神。二致。它泛滥了达七百年之久,留下的只是一片无神论的荒漠,再也不能为这个民族的灵魂提供任何有益的食粮了。道教退化成降妖捉怪的妖术,它曾从佛教中汲取大量的营养,以弥补自己的先天不足。佛教的引入是用以满足儒教无法满足的人的先天需求。它们的教育方式彼此影响,都产生了很大的变动。任何一个提供行善途径的机构都会得到人们的赞助,假如他手头上碰巧有点积蓄,或者认为赞助与某些行为一样高尚的话。任何在某一方面似乎对人有利的神灵部会受到人们的垂青,就像个偶尔需要伞的人,遇到了一家伞店。一个英国人头伞,绝不会去问 伞是何时发明,何时开始普遍应用的,中国人对自已崇拜的神也同样不过问其家世、经历。 
  经常有些学术讲座探讨中国有多少佛教徒和道教徒。我们认为,这个问题就像探讨在联合王国有多少人抽十便士一包的香烟、多少人吃菜豆。谁想抽十便士一包的香烟,又能弄到,谁就抽;谁想吃菜豆,又能买起,就吃好了。中国的两种“教义”与此相同。谁想请和尚做法事,又能付得起钱,就去请和尚,他也因此就成了“佛教徒”。如果他想请道士,他也同样可以去请,这也会使他成“道教徒”。如果他既请了和尚,又请了道士,那也无关紧要,人们可以说他既是“佛教徒”,又是“道教徒”。因此,一个人可以同时是儒教徒、佛教徒和道教徒,这并没有什么不和谐的。佛教融合了道教,道教融合了儒教,最后儒教又融合了佛教与道教,因此,“三教合一”。 
  中国人同他们的“三教”之间的真正关系可以用盎格鲁一撒克逊人同他们的语言间的关系来说明。他们在描述自己的语言构成时说:“撒克逊语、诺曼底语和丹麦语就是我们的语言。”即使可以确定我们的祖先为谁,我们的血管中流动的撒克逊人和丹麦人的血液的比例并不能影响我们对语词的选择,它只受思维习惯和我们所期望的用途影响。学者会使用大量的拉丁词语,混杂着很多丹麦语。而农民则主要使用朴素的撒克逊语。担二者都以撒克逊语为基础,其他语言只是补充。在中国,儒教是基础,所有的中国人都是儒教教徒,所有的英国人都是撒克逊人。佛教和道教的观念、用语和教规对儒教产生的影响,随环境的变化而变化。但是对于中国人来说,“三教”融合而成的仪式中,并没有什么不和谐或矛盾的地方,就跟我们在同一句话中使用了来源于不同民族的词汇一样。 
  两种不同形式的信仰常常是互相排斥的,让中国人相信这一点,并不容易。他们不懂什么是逻辑矛盾,也很少关心。他们本能地学会了一种调和不同命题的技巧。对于两个本质不相容的命题之间的关系,他们可以置之不理,强行把它们扯到一起。他们所接受的思维训练,也为融合不同形式的信仰作了充分的准备,就像液体可以通过内渗和外渗相互融合一样。他们已把这种“信仰亲和”推到了逻辑消亡的境地,即使告诉他们这一点,他们也不会明白,而且也无人可使他们理解。 
  教义的机械融合有两个非常显著的特点。第一是与中国人天生的喜欢秩序的本能相违背。中国人喜欢秩序,闻名遐迩,官阶的精心划分可以鲜明地体现这一点。帝国的所有官员,分为九品,每品都有严格的身份标志和权限。但是中国的神灵世界却找不到这等级森严的秩序,若问中国人“玉皇大帝”和“如来佛”谁权力大,简直是白费口舌。即使在“万神殿”中,诸神排列的秩序也是偶然的。暂时的,经常交换不定。中国人的精神世界中,权威的地位也不固定,这种十足的混乱状态,如果出现在地球上,一定是个无政府主义的世界。在供奉孔子、老子和如来佛的“三教堂”,排列秩序问题仍很突出。尊者位于中间,我们认为,这个位置应归孔子,如果不是他——既然他不信神——就应该是老子。可以肯定,这个问题在过去一直令人们争论不休,但在我们听到的所有的争论中,总是佛祖受惠,尽管他是个外来户。 
  另一个重要特征即是中国所有的信仰都把人的道德本质贬得极低,犹如假货币顶替了真货币。儒教高尚的箴言一点也不能使人们消除对于道教经常提到的妖魔鬼怪的恐惧。人们常说,世界上没有任何文明民族比中国人更迷信、更轻信的了,这也确实不假。富有的商人和知识渊博的学者竟然每个月都要花两天时间祭拜狐狸,黄鼠狼、刺谓、蛇和老鼠,它们被标在一张纸上,又被称为“大仙”,据说它们甚至可以左右人的命运。 
  数年前,中国一位著名的官员曾跪在一条被当成水神的大蛇面前,据说该神是前朝的一位官员,他曾奇迹般地制服过泛滥的黄河。在洪水泛滥时,将蛇当成神加以崇拜的现象十分普遍。在离黄河较远的地方,人们会不分青红皂白把一条生活在旱地的普通的蛇当做神。如果河水退去,为纪念神恩,人们会做出一些非常富有戏剧性的事件来,他们把蛇放在盘子上,抬进庙里或其他公共场合,县官和其他官吏每天都去烧香磕头。在离黄河近的地方,河神一般认为就是水神,但在稍远的内地,战神关帝则被当成雨神,有时,这二者会被大慈大悲的观世音代替。在中国人眼里,这似乎并不是非理性的,因为他们从不考虑本质融合的前提,即使告诉他们其中的荒谬,他们也不能理解。 
  我们还经常注意到与求雨有关的另一个古怪而又极有意味的事实。在中国名著《西游记》中,有一个主要角色是一只从石头里诞生的猴子,后来渐渐演化成了人,很多地方将这个想像之物当成雨神来崇拜,以便排除河神和战神。中国人从来就分不清真实与虚构,还有什么例子能比这更具说服力?在西方人的观念中,原因与结果相互关联。但是中国人向一只并不存在的猴子求雨,他们的因果观念是怎样的呢?我们无论如何也捉摸不透。 
  中国人对神有各种各样的描述,他们是如何对待这些神的呢?这个问题有两个答案:崇拜与忽视。中国人每年在香烛、纸钱上要花多少钱,经常有人作出估计。这种估计当然是先把某个地区当成一个样本,计算出确切的数字,再以此推算帝国的其他地区,没有什么比这种所谓的“统计”更不精确了,就像有人统计一大片蚊子,“数累了,接着就开始估计”。 
  把中华帝国当成一个整体下结论,很容易犯错误。中国人到庙里拜神就是个突出的例子。从广州登陆的旅行者,看到庙里香客如云,烟雾镣绕,会认为中国人是世界上最盲目崇拜的民族之一。假如让他先别急着下结论,等他游览了帝国的另一端再说。他会发现,大量的庙宇早已荒颓无人,大部分时间,包括初一、十五也没人进香,甚至在上香最盛行的时间——春节,也可能没人进去。他会发现成千上万被人们遗忘的古庙,尽管偶尔有人做些修复,但已无人知道它们修于何时,为何而修了。他会发现,一块方圆数百里、人口稠密的地方,找不到一个教士,无论道士,还是和尚。在有些地方的庙里,他一般看不到妇女,孩子从小到大,没有人教导他们皈依神的必要性。在帝国的其他地方,情况则截然不同,表面的崇拜仪式渗透到人们日常生活的一举一动中。 
  中国的宗教势力可以和造成夏威夷群岛的火山相比。在夏威夷最北部和最西部的岛屿上,很久以前,火山就死亡了,昔日的残破不堪的火山口现在已长满了茂盛的草木。但在东南部的岛屿上,大火仍然在熊熊燃烧,不时地从岛中传来剧烈的地震。在中国最古老的地区,也最少有人烧香拜佛,而在中国文明最辉煌时仍处于野蛮状态的地区,偶像崇拜却极为盛行。这些表面现象最容易产生误导,在没有进一步充分调查之前,很可能会被赋予言过其实的意义。 
  孔子曾说:“敬鬼神而远之”,他的现代门徒也因此认为,对中国五花八门的众神敬而远之是最明智的。与蒙古人、日本人相比,中国人相对没有宗教偏见。在一些庙宇的门媚上,我们还经常看到古老的格言:“敬神如神在”。以“如”字来传达模棱两可的含义,完全是中国人的本能使然。下面这一流行的说法,表现得更具体: 
          敬神如神来, 
          可来可不来。 
          敬神如神在, 
          不敬神不怪。 
  比敬而远之更进一步的是仪式崇拜,它有一定的程序和方法,这样做的目的,无非想获得外在的利益。 
  若说中国人似乎与神圣感毫不相干,也仅仅是一种礼貌的表述。我们已经认识到,中国人所有的信仰,要么是常规的仪式,要么是交易一一供给神多少就得到神的多少恩赐,对“老天爷”的崇拜最能表现这一本质。问一个中国人,为什么要定期祭拜“老天爷”,他会告诉你:“因为我们从他那里得到粮食和衣服。”即使他对“老天爷”的存在茫然无知,仍会按仪式照行不误。祖先们这样,他也这样做,至于是否有用,“谁知道呢?” 
  这种对待宗教仪式的态度是浅薄的。在一些被人遗忘的庙宇的门柱上,我们经常可以看到一副具有讽刺意味的对联,较能说明这一点: 
          古庙无僧风扫地 
          空室有情月作灯 
  中国人崇拜神,仿佛西方人参加保险。一般人认为,“最好相信神存在”。也就是说,他们不存在,相信了也没害处;假如确实存在,又被人忽略了,他们可能会生气、报复。人们认为神和人一样,也受一定动机的支配,有句俗语说,一个羊头(作供品)可换来一切。那些没有特殊可以赐给人的神,例如“三圣”,常常是穷神,而观世音菩萨和关帝则既尊贵又富有。 
  中国人对神的崇拜不仅仅建立在纯粹假设的基础上:信神“有益无害”,而且走到了令我们难以理解的地步。他们经常说:“信,就有,不信,就没有。”也似乎认为确是如此。这种表述方式(很难称为思维方式),就像一个中国人说:“相信皇帝存在,就存在;不相信,就不存在。”这样类比,中国人很乐于接受,可他们自己就好像不能通过一定的推理认识到这一点。 
  在中国,可以看到许多朝圣者每走一步、磕一个头。他们要在这种沉闷而单调的朝圣历程中花掉很长时间。如果问他们为什么要这样,他们会说,有许多人的信仰是假的,信仰者有必要以这种苦行的方式表达他们的虔诚。无论怎样评价这种例外,我们仍会毫不犹豫地断言,他们彼此间缺乏信义,在信仰方面表现得更突出,对北京附近一座庙里和尚的描绘,简直是描绘阴险狡诈方面的杰作。人,长了一张什么样的脸,就过一种什么样的生活。 
  与其他国家的非基督教徒一样,中国人把自己的神想像得和自己一样,因为有不少神灵是他们的同胞。笔者曾看到过一张以菩萨名义贴出的告示,它晓谕世人,人类逐渐变得邪恶的消息已上达天庭,玉帝获悉,大为震怒,大声怒斥他的神臣,因为他们没有强迫人类从善。中国人认为,人们周围到处都是神灵精怪,他们一样可以贿赂、奉承、容易欺骗。中国人讨价还价时,很想占对方的便宜,对他所祈祷的神,如果可能,他同样想占便宜,他可以通过捐钱修庙换取好运,但假如他捐二百五十个铜钱,却可能在功德簿上记一千!神灵只能从簿子上知道他捐了多少。修庙时,每尊神像都会用红纸遮住眼睛,这样,他就不能看到周围混乱的场面和不敬的举止了,如果庙位于村外,常常会成为盗贼分赃的窝点,因此,人们就把庙门封上,让神灵独自待在里面,去和宇宙尽情地交流吧。 
  年底,灶君要回到天上,汇报他所在家庭的行为,但他的嘴巴早已被抹上粘糖,不能说出不好的事情了。这个风俗,人们都很熟悉,它是中国人智胜天界神仙的典型例子。一个男孩子有时会取女孩的名字,这可以便愚蠢的妖怪认为他真是女孩,从而放过他。巴伯先生谈到,在四川,女婴被溺死后,人们总是大烧纸钱,供她的鬼魂使用,以此来安抚她们。送子菩萨的庙和其他的庙不同,进去的一般都是妇女。很多庙会为她们提供泥做的小男孩,它们有时由菩萨抱着,有时像货物一样摆在架子上。妇女烧香拜佛时,会把泥人的小鸡鸡掰下来吃掉,以确保生男孩,这己成为一种习惯做法。庙里一般有许多这样的小泥人,是特地为去庙里进香的妇女准备的。不过,只能悄悄地偷走,不能公开拿走。假如果真生了男孩,这位妇女就要在上次偷走泥人的地方,再放上两个,以示感谢,中国的水手认为,中国海上可怕的风暴是恶毒的妖怪制造的,它们躺在水底,静静地等待着,一旦有船的动静,马上兴风作浪。风暴十分猛烈时,据说水手会做一个和他们船一模一样的纸船。等到最危急的关头,放入海中,这样可以欺骗狂怒的水怪,使真船脱身。 
  霍乱之类的瘟疫发生时,中国常在六、七月份庆祝新年,帝国的大部分地区都盛行这一风俗。人们认为,这样做可以欺骗瘟神,它会很惊讶地发现,原来自己算错了年历,然后就离开了。这也可以使人们很容易理解,为什么“秋二月”实际指的是“永远”。人们还有一种欺骗神的做法,就是爬到供桌底下,将头从一个专门的圆孔中伸出来。神会以为人真的把头献给了自己,便赐以相应的好运。而那人将头一缩,就回家享受将来的好运去了。 
  有一次,我们偶然发现,一个村庄想把神像挪走,将庙改成学校。村民们本指望能从佛像的肚子里掏出些“银子”,补充开支。可这些头脑简单的乡下人根本不了解佛和塑佛像的人,结果发现那宝贵的心脏只是一团锡块。不过,确有些憎人曾把财宝藏在佛像里,使庙里遭了抢劫,佛像要么被搬走,要么被当场打碎。但是这帮粗鲁的家伙仍会相信神。据说,有个县官在审理一桩与僧人有关的案子时,牵涉到庙里的佛祖,县官便将它召到堂上,令其跪下,可它不跪,愤怒的县官命人把它重打五百大板,结果打成了一堆泥土,并以缺席判其败诉。 
  每逢土地干枯,不能播种时,人们便向雨神求雨,希望它大施神威,普降甘霖。假如求了很长时间,仍无结果,村民经常会将神像抬到最热的地方,让它亲自去看看,而不是只守在庙里听人说。人们也经常不掩饰对神的不满,有一句流行的俗语为证:“三,四月不修屋,六、七月骂涝神。” 
  我们听说中国的一个大城市,遭到了一种严重的传染病的侵袭。人们判断,这是当地的一个神在作怪。于是,他们就联合起来,严然对付一个现实的恶霸,把神像打成一堆碎土。我们没有证据保证这一描述的真实性,只是听说而已,不过,这己足够了,因为整个过程很符合中国人的神灵观念。 
  我们列举的这些实例,很容易使不了解中国人性格的人认为,中国完全不可能有宗教。确实有人这样直截了当地断言。梅杜斯先生在他的《中国人及其信仰》一书中,批判了胡克先生的概括,认为那是对“人类高尚生活的毫无根据的诬蔑”。他坦然承认,中国人既不关心纯粹的神学争论,也不关心把争论结果当成信条的民族的行为。但他断然否认中国“缺乏对不朽的渴望,缺乏对美好、伟大事物的由衷赞叹,缺乏对伟大、善良的人物的持久、毫不动摇的热爱,缺乏向往神圣、高尚事物的灵魂”。托马斯·韦德爵士曾对中国和中国人有过长期了解,对于“中国有无宗教”这一普通的问题,他应有资格作出权威性的回答。最近,他发表了如下观点:“如果宗教是超越于道德之外的东西,那么,我拒绝承认中国人有宗教。他们确实有祭拜活动,或更确切地说,是一种混合的祭拜活动,但没有信仰:他们随时都可能嘲笑本民族那些形形色色的偶像崇拜,但他们绝不敢漠然置之。” 
  我们觉得没有必要探讨这个有趣而又不容易回答的问题,详细讨论并不困难,可是,不能保证会有结果。我们有比讨论更有效的、更现实的途径来解决这个问题。道教和佛教对中国人影响很大,然而中国人既不是道士,也不是和尚。他们是孔夫子的信徒,无论给他们的信仰增加点什么,或减少点什么,他们仍不会改变。我们应该努力探寻儒教到底为什么不能成为中国人必需的宗教。为此,我们打算引用一个著名的中国问题专家的话,他的研究是不能忽略的。 
  恩斯特·费伯博士《孔子思想体系类编》一书的最后,有一节名为“儒教的不足与失误”,它指出,儒教中有关人与人之间关系的精彩论述,在基督教的《启示录》中也得到回应。下面的二十四条即是引自其中,我们偶尔加上了几句评论。 
  1.“儒教自认与现存的神没有关系。” 
  2.“人的灵魂与肉体之间没有区别,无论从生理的角度,还是从生物学的角度,都没有对人进行明确的界定。” 
  在人的灵魂方面缺少明确的说法,这很令学习儒家学说的外国学生迷惑不解。对于广大的普通百姓来说,这种教诲的最终结果是使他们除了在肉体的生命力方面之外,根本不了解什么是人的灵魂。人死之后,传统的说法是,他的“灵魂”升天,“肉体”化为泥土。但时常还出现一种更简单的理论,认为“灵魂”或生命的气息消融在空气中,肉体化为尘埃,这一观念与真正的儒教的唯物主义不可知论完全一致。问中国人,他有三个灵魂,一个灵魂,还是没有灵魂?几乎难以引起他的兴趣。对他来说,这种问题就像问他,人体的哪块肌肉带动了咀嚼一样。只要咀嚼舒服,他才不管是哪块肌肉呢。同样,只要他有好胃口,还可以养家糊口,他也不去管什么“灵魂”,除非它与米价有关。 
  3.“没有解释为什么有的人天生就像一个圣徒,其他人却是普通人。” 
  4.“每个人都能成为圣人,却不能解释每个人都没有成为圣人的事实。” 
  5.“儒教对罪恶的态度坚决而认真,但除了道德上的惩戒外,没提任何惩罚措施。” 
  6.“对罪孽和邪恶缺少深刻的认识。” 
  7.“因此,儒教发现无法解释死亡。” 
  8.“儒教中不存在一个中介,使人的原初本性与自身的理想重归和谐。” 
  9.“祈祷及其道德力量与儒教无缘。” 
  10.“尽管一再强调信任,现实中却很少鼓励作为信任前提的诚实,而且恰恰相反。” 
  11. “认为一夫多妻制天经地义。” 
  12.“赞同多神论。” 
  13.“相信算命、看日子,预兆、做梦和其他的幻像(如凤凰等)。” 
  14.“伦理道德和其他的仪式搅在一起,成为一种十足的专制形式。” 
  15.“孔子对待古代制度的态度反复无常。” 
  16. “断言某些美妙的音乐会对道德产生荒谬的影响。” 
  17.“夸大楷模的力量。孔子本人就是最好的例证。” 
  如果像儒教所宣称的,君为器,民为水,器圆则水圆,器方则水方——似乎就难以解释为什么中国的伟人没有对他的研究者产生强有力的影响。如果楷模确实像儒教所说,有那么大的影响力,为什么现实中恰恰相反,看到的都是苍白无力的现象?对“圣人”的神化(第20条将提到)正反映了第8条所言的中介的缺失。无论“圣人”多贤明,他也只能提些好的建议。假如人们对他的建议置若罔闻,他也无可奈何,最多以后不提而已。 
  我们觉得,孔子的一句话非常富有启发性。他说:“不愤不启,不悱不发,举一隅而不以三隅反,则不复也。”这话是针对圣贤们说的,很精彩,可决不是预防针,只能算一剂补药。眼睁睁地看着旅行者受盗贼抢劫,却大谈加入互助旅行团的好处,说他之所以头破血流,就是因为没有加入该团。而受伤者对此全都知晓,可他现在大量失血,早已昏晕过去。他最需要的不是反思过去违背了常规,而是油、酒和可以供他尽量恢复的避难所,而且首先要有一个聪明、乐于助人的朋友。对于肉体有残疾的人,儒教或许可以做点什么,可如果是道德或精神方面的,它也无能为力。 
  18.“儒教教义中,社会生活由暴政控制;女人是奴隶;孩子没权力,只能绝对服从长辈。” 
  19.“绝对孝顺父母,把他们奉若神灵。” 
  20.“孔子思想体系的最终结果是崇拜天才,例如人的神化。” 
  21.“除了没有伦理价值的祖先崇拜,不存在关于不朽的明确观念。” 
  22.“现世现报,无形中鼓励了利己主义,不是贪婪,就是野心勃勃。” 
  23.“中国历史表明,儒教不能使人们获得新生,努力追求一种更高尚更神圣的生活,现实生活中,儒教己与道教、佛教相融合。” 
  24.“整个儒教对死者、生者都不能给予安慰。” 
  对于中国各种不同形式的信仰的融合,我们已经作了论述。中国人自己也早已认识到,无论儒教还是其他宗教,都不能“使他们获得新生,努力追求一种更高尚更神圣的生活”。有一则传说鲜明地表现了这一点,传说的作者不详。 
  据说有一天,孔子、老子、如来佛三位圣人在永恒的神界相遇了,一致哀叹世风日下,人心不古,他们的教义在天朝上邦无人听取。经过一阵讨论,共同认为,他们的教义本身虽然精妙绝伦,令人赞叹,但没有一个永恒的楷模引导人性朝这个方向发展。于是,一致决定下凡人间,物色一位合适的人选。说完,他们就分头到人间去了。孔子首先遇到了一位老人,老人看上去令人肃然起敬。见到孔子来了,老人却端坐不动,只是请孔子坐下,和孔子谈起古圣人的训诫和今天人们对它的忽视。交谈中,老人表现出对古代圣言了如指掌,其渊博的学识和敏锐的判断力令孔子大为惊叹。谈了一阵后,孔子告辞,老人仍是端坐不动,并不起身相送。看到老子和如来佛一无所获,孔子就讲述了自己的奇遇,并建议他们也轮流去拜访那位端坐的哲人,看他是否像熟悉孔子思想一样,熟悉他们的思想。老子先去,令他兴奋的是,老人对道教教义也十分熟悉,仿佛他就是道教的创始人,其口才和热情也堪为楷模。同样,如来佛也获得了令人惊喜的成功,不过,令老子和如来佛惊奇的是,老人对他们也非常尊敬,但也同样都没起身相送。 
  三位圣人又相聚了,他们一致认为,这位举止罕见的老人正是他们的理想中人,不仅精通“三大宗教”教义,而且还能论证“事实上三教归一”。于是,他们一起又找到老人,向他解释了初次拜访的目的,希望老人能重振三教,将它们付诸实践。老人静静地坐着,听他们讲完,然后答道:“尊敬的诸位圣人,你们的善行如日月齐光,你们的计划重比泰山,令人赞叹。可不幸的是,你们选错了完成这一伟大使命的人。诚然,我曾拜读过诸位的大作,对它们的崇高与一致性也略知一二,可你们也许没注意到,我的上身是人体,下身却是石头。我擅长从各个方面论述人类的责任,却由于我自身的不幸,永远不能将它们付诸实践。”三位圣人听了,长叹一声,就从地面上消失了。从此以后,再也不企图寻找可以传播三大宗教的凡人了。 
  我们常常将目前中国与一世纪的罗马相比,事实上,目前中国的道德状况要远远高于罗马帝国,可二者有一共同之处,即它们的宗教信仰都濒临崩溃的边缘。我们也可以像吉本评价罗马那样,来评价中国:对普通百姓来说,所有的宗教都一样真实;对哲学家来说,所有的宗教都一样虚伪;对政客来说,所有的宗教都一样有用。中国皇帝,也和罗马皇帝一样,“既是高级教士,又是无神论者,和至高无上的神。”造成中华帝国这一现状的就是那融合了多神论与泛神论的儒教。 
  对无神论是否正确的问题漠然置之,要比纯粹的无神论更可怕。中国存在多神论与无神论两种迥然不同的信仰,可很多受过教育的中国人却感到二者没什么矛盾之处。最令人悲哀的是,中国人从本性上对最深奥的宗教真理是绝对冷漠的,比如,他们接受没有灵魂的肉体,接受没有心灵的灵魂,接受无条件的和谐,接受没有上帝的宇宙。 

\chapter{中国的现实与时务}
中国像一艘庞大的航船,儒家经典就是中国统治者驾驶这艘航船的航海图。它是人类设计的最完关的蓝图,或者如已故的威廉姆斯博士,莱格博上及其他一些学者所说的,在某种意义上,说它出于天启,也许并不过分。中国人利用这份航海图创造了多少业绩,航行过哪些海域,目前正朝哪个方向前进一-这些都是非常重要的问题,因为中国和西方许多国家的交往越来越密切,将来也可能对它们产生越来越大的影响。 
  据说,社会道德生活有六项指标,每一项都十分重要;它们共同构成检验社会性格的可靠证据。具体如下:1.工业水平;2.社会风俗习惯;3.妇女的地位和家庭的特征,4.政府的组织形式和统治者的品质;5.公共教育状况;6. 宗教信仰与现实生活的关系。 
  上述各项指标,我们在讨论中国人的各种性格特征时,都附带作了阐述,虽然还不够充分,也没有对各自所占比重作必要的安排。在考察中国人的性格时,有大多方面需要注意,有时不得不忍痛割爱,被迫放弃。我们只想通过自己的选择勾勒出中国人性格的大致框架。如果真要完全展现,还有许多其他特征应考虑在内。 
  我们在阐明中国人的性格特征时,列举的例证大部分都具有说服力,因为经过权衡,它们似乎更为典型。它们就像组成一副骨架的骨头,每一块都应事先放在各自的位置。除非是冒牌货,否则完全不能忽视。确实可能有人反对,每块骨头都放错了位置,而且另外一些可以改变整体结构形态的骨头也没放在恰当的位置。这种批评极为公正。对此我们不仅承认,而且还要特别说明,这些选择的“性格”不可能使人全面认识中国人,就像描绘某人的眼睛、耳朵和下巴,不能让人形成对他的准确印象一样。但同时,我们必须提醒读者,那些结论并非仓促之间形成的,实际上,我们观察的事实远远多于本书所提到的,即使稍微难以肯定的观点,也都得到充分地论证。这些事实比比皆是,就像北方起大风时的尘沙,灌满了人的眼、耳、鼻、头发,衣服经常遮天蔽日,有时中午也需要点灯。这种现象,人们也许会搞错起因,但对它的描述是完全正确的。不过,观察物理现象和道德现象有重大的差异:前者每一个人都可以观察到,而后者只有幸运者才能遇到,而且还要善于观察。 
  中国人的生活充满了矛盾的现象,只看一面,而忽视另一面,肯定会作出错误判断,同时还永远认识不到自己是错误的。将两个明显不和谐的观点融合起来,不是件容易事。然而时常又必须完成这一任务,世界上也没有任何地方能比在中国更需要这样做的了。在中国,完全了解事物的一个方面已是相当困难,更何况两个方面。 
  我们已经谈了,儒教具有极高的道德品性,而且相信,它能造就许多品德高尚的人。这也正是人们对它奇妙的道德体系的企盼。可是它如何使大部分人的品德都变得高尚呢?有三个方面的问题,可以揭示人的真实性格:他与自己的关系如何?他与别人关系如何?他与自己的信仰关系如何?通过这三个互相联系的方面,就可以对一个人的性格准确定位。读过前面各章的读者,已经知道了现代中国人在这三个问题上的答案:他们对自己和别人缺少真诚和信义;对别人缺少利他主义;他们的信仰是多神论。泛神论和不可知论。 
  中国人并不缺乏智慧,也不缺乏耐心、现实性、快乐,这些方面他们都是杰出的。他们缺乏的是人格和良心。许多中国官员受不了贿赂的诱惑,就做了错事,还以为永远不会被发现,因为“天知,地知,你知,我知。”有多少中国人能抵制得了压力,不推荐公认的不称职的亲戚呢?想像一下抵制在家庭中带来的后果吧,中国人害怕面对这一后果,难道还有什么奇怪的吗?把道德律令引入这样的领域,中国人是怎样想的呢?看到中国的民政机构,军队机构、商业机构中充满了寄生现象和裙带关系。难道还会对中国门卫和警察的失职感到奇怪吗? 
  想了解中国人道德的真实情况,会得到中国人的帮助。尽管他们竭力掩盖自己及朋友的缺点,却经常对民族性格的弱点直言不讳。他们对其他中国人的描述,时常让我们想起卡莱尔在《弗雷德里克大帝的一生》一书中以快乐的笔调描写的一段对话。这位君王很喜欢一位学校监督员,总爱跟他谈点什么。一天,君王问道:“苏泽先生,你的那些学校近来怎样?我们的教育事业发展得如何?”“当然啦,不错,陛下,最近几年好多了。”苏泽答道。“最近几年?为什么?”“啊,陛下,从前,人们相信人天生邪恶,学校实行严格的管理制度;可现在,我们认识到人天生向善,校长采用了更为宽容的管理方法。”“天生向善!”弗雷德里克摇着他那苍老的头,悲哀地笑了笑:“哎,亲爱的苏泽,我看你一点儿也不了解这该死的人类。” 
  中国社会就像中国的许多风景胜地。远看,具有诱人的魅力。可是,再近点,总会发现很多破烂不堪、令人讨厌之处,空气中弥漫着难闻的气味。照片绝不能客观地反映中国的风景胜地,虽然照相机被认为具有“无情的公正”,但有关中国的照片却不如此,肮脏和难闻的东西都被遗漏了。 
  在中国,象征幸福的东西如此之多,可谓举世无双。可是,不用过太久,我们就会发现,中国人的幸福只是徒具其表,我们相信这是个真实的评价,就像说亚洲不存在家庭生活一样。 
  在对中国进行理论分析,并探讨如何使这种理论与现实相适应时,我们总是想起那些石碑,它们立在大路与河流交叉的地方,以“纪念”修桥的人。有时,这块碑旁边会有半打同样的石碑,它们已经缺头少角,残破不堪。对逝去的岁月和历代的纪念物,我们一直很感兴趣,当我们问起过去修的那些桥时,人们回答说:“啊,它们嘛,好几代以前就不存在了一一一谁知道什么时候。” 
  几年前,笔者在大运河上游玩时,遇到了逆风,被迫停下。我们在岸上闲逛,看到农民们正在田野里劳作。时值5月,田野里一片翠绿清明的景象。此时,任何游客都会对精细,不知疲倦、辛勤劳作的农民表示赞叹,因为他们把大片田野变得像花园一样美丽。然而,和他们稍稍交谈,才发现,他们刚刚度过一个艰苦的冬季。去年的洪水和干旱使他们颗粒未收,附近村庄的人都快饿死了一一也就是说,目前他们正在忍饥挨饿。政府发的一点点救济,只能是杯水车薪,零星的一点点,还要受到无耻的侵吞。这些可怜的农民毫无办法,一点儿也不能保护自己。可是从表面上,这一切完全看不出来。而其他地方是丰收年景,人们安居乐业。北京的《邸报》和中国的西方杂志都没有报道过任何有关消息。忽视现实,并不能改变现实。无论其他人是否知道这件事,当地人仍在忍受饥饿。即使断然否认这些事实,也不能证明采取了有效的救济措施。经验地认为中国人应是什么样子,是一回事;而仔细观察他们实际上是什么样子,完全是另外一回事。 
  我们很清楚,中国社会存在的许多弊病,在西方“有名无实的基督教国家”也同样存在。或许读者会感到失望,因为我们没有对这一事实作出更明确的结论,也没有进行系统的比较。我们确曾这样想过,但最后不得不放弃。笔者熟悉的西方国家十分有限,难以完成这项任务。请读者自己比较吧,不过要尽量摆脱“爱国主义的偏见”。在证据不足的情况下,还要暂认为中国人是无辜的。经过比较,我们至少可以看出,西方国家面对的是充满黎明曙光的未来,中国面对的却是充满黑暗的漫漫过去。我们想请读者好好深思一个意味深长的事实:这到底是怎么造成的呢? 
  再重复一遍,中国需要的很少,只有人格和良心。也可以说,二者是一个东西,良心本来就是人格。有人称赞一位著名的钢琴制造家,说他“像他的钢琴一样——宽厚、正直、高贵”。在中国,谁遇到过这样的人? 
  有一本关于一位英国作家的传记,在结尾处,他的妻子对刚去世几年的丈夫这样写道:“外界把他当成作家。传教士,一名社会成员;但只有每天和他亲密生活在一个家庭的人,才知道他是一个怎样的人。在他人眼里,他那浪漫的一生,温柔细腻、缠绵悱恻的私人信件,必定为一层面纱所笼罩。但只要稍微揭开这层面纱,我可以说,假如在人世间最高尚、最甜蜜的感情中,有一份永不褪色的爱情一——六十三年,纯洁、热烈依旧——无论生病的时候,还是健康的时候,无论是阳光明媚的日子,还是凄风苦雨的日子,无论是白天,还是黑夜,从未出现过一个仓促草率的字眼,一个不耐烦的手势,或一个自私的举止。如果这份高尚的爱情可以证明骑士时代永不会过去,那么,对于一位有福永远享受这份爱情的女人来说,查尔斯·金斯利是一位真正完美的骑士。” 
  基督教文明最美好的果子,就是它创造的完美的人生。如此人生,并不少见,当代就有数百个记录,更有千千万万不为公众所知的。每位读者至少知道一个把全部生命献给他人的例子,有些读者可能有幸在自己的经历中遇到更多这类例子。我们怎样解释这些人生呢?他们的动力来自何处?我们不希望过分怀疑,但经过反复考虑之后,我们确信,如果使中国变成现有这个样子的那种力量,能塑造一个像金斯利一样的人,这在道德方面,将是一个伟大的奇迹,比道家典籍中所有寓言里的奇迹都要大。任何人类制度,都不能逃脱无情的规律,《圣经》上说:“看他们的果子,便知道他们。”儒教有足够的时间获得其最终结果。我们相信,可做的,它已全做了,以后再也不会有更大的果子。它已使人的能力发挥到了极致,而且超过了其他地方、其他条件下人类所能做的一切。耐心地考察了中国的这些现象之后,即使是最友善的批评家也不得不悲哀地承认:“是儒教造就了中国。” 
  在中国改革问题上,存在三种不同的态度。第一,没必要改革。虽然不是所有的中国人都这样想,但无疑有不少中国人抱着这一态度。某些不了解中国的西方人也这样认为,第二,改革不可能成功。真正的、长期的改革尚未开始,就必定会遇到巨大的障碍,许多有机会了解到这一点的人,都持有这种悲观的论调。他们认为,对庞大的中国进行彻底的改革,就像给木乃伊注入活力使其复活一样,毫无希望。不过,如果没有我们前面的论述,这一观点就显得论据不足。 
  还有人认为,中国不仅需要改革,而且也可能成功。他们认为,问题的关键在于以何种方式进行改革。这方面,也有几种观点。 
  我们首先面对的问题是,中国是否能够自我革新?认识到改革之必要的中国政治家认为,中国当然应该自我革新。最近,北京《邸报》的一份奏折中,就有一个自我革新的例子。写奏折的官员抱怨内地某省的百姓骚动不安,并说他己派出一批得力人员奔赴各地,向百姓宣讲康熙皇帝的《神训广谕》。他显然是希望以这种强有力的方式教化百姓,移民易俗。尽管一无所获,但宣讲道德箴言(对基督教传道的原始模仿)在改良人的道德品行方面,仍不失为一种最有希望获得成功的方法。教化失败后,没有别的办法,只能像过去一样,再次进行同样的努力。长期的经验表明,这一做法必然会失败,事件变化,但结果依旧,全部努力都会化为泡影。那个石腿,雄辩的老人的寓言已充分表明这一点。 
  既然箴言无效,人们便寄希望于楷模。这一点,前面已作过讨论,这里重提,是想指出为什么最好的楷模没有产生预期的结果。其原因在于他们无力使更多的人接受他们生命中的最初动力。比如,山西省前任巡抚张之洞,据报告说,他曾采取强有力的措施制止官吏吸食鸦片,禁止百姓种植鸦片。但他的下属中有多少人能与他通力配合呢?没有这种配合,其结果可想而知。任何一个外国人,如果他必须依赖的中国人不支持他的改革,他不能不承认,在中国问题上,他无能为力。对于一个中国人,无论他位居何职,难道不同样会感到束手无策?最多是在目标确定之后,便着手处理面前的问题(只是表面上的),仿佛一只猫待在阁楼上,就要清除上面的老鼠。这位官员一旦调任,甚至还未开始走,老鼠就已经开始活动了,一切照旧。 
  中国政治家应该怀有亲自改革祖国的希望,这不仅可信,也极为自然,因为除此之外,他也别无选择。如果一位精明的不列颠官员,了解了“东方人特有的可怕的冷淡和宿命观——对这种极端的愚蠢,席勒说,即使上帝,也无计可施”——并且知道长期“改革”的方方面面,他可能早就把结果准确地预测到了。巴伯先生在谈及中国西南开采铜矿暴露出来的弊病时说:“铜矿还没有完全开采之前,云南必须补充人口,必须平等对待劳力,必须修公路,必须改善扬子江上游的航运设施——一句话,中国必须开化。除非有外来的动力支援,否则,想完成这一过程,一千年的时间都不够。”* 企图改革中国而不“借助外力”,就像在大海上造船,难以驾驭的海水和海风会使这一切化为黄粱一梦。始于并终于机器内部的力是不能使机器前进的。 
  北运河在北京和天津之间,有一个转弯,在那儿,游客会看到岸边有一个倾圮了一半的庙,那一半被大水冲走了。靠水的一边有一道精心修筑的栅栏,由拴在桩上的一捆捆芦苇组成,用来挡水。神像立在外头,任凭风吹日晒,河床中积满了淤泥,周围的田野没有任何阻拦洪水的设施,这是一幅荒凉破败的帝国残景。中国有一句经典格言:“朽木不可雕。”只有将朽木全部砍掉,老树才能发新芽。中国想从内部改革是不能成功的。 
  不久前,西方国家广泛认为,中国可以通过加入“联盟”而获得新生。不过,这种希望没有多少切实的根据。世界主要国家在北京派驻代表已有三十多年了,它们到底为苦难的中国带来了多少有益的影响?而且,令人悲哀的是,大国间的关系并不对中国格外有利。中国人敏于事,西方人有什么证据可以使中国人相信,它们发展自己国家的动机能比中国人改革的动机更高尚?既然中国自己正在成为一股“力量”,她就忙于挑拨其他国家之间的关系,从中取利、却没想到其他国家是在“掠夺”她,而不是在进行道德教化。因此,即使中国要改革,也不能通过外交途径。 
  * 已故的巴伯先生这段意味深长的话,最近为1890年8月北京《邸报》的一篇奏折证实了,云南矿务执事唐奘报告了工作与运输的条件,他说:“人们大量进行非法开采,官员们害怕独揽开采权会带来不良后果,就想了一个办法,他们低价购进非法开采的铜矿石,较有效地利用了人们的额外劳动,这一方法也颇受当地人的欢迎。我认为,这种方法既可以使采矿正常进行,也不会给外来侵入者提供借口。”不过,皇帝只命令税务署将这份奏章“记录备案。” 
  奏折附文中,巡抚报告说,每月可以从非法采矿者手中买进一万斤铜矿石,但“不付钱,只供给他们油和大米。”最后,他还说:“矿区的整体情况非常令人满意。” 
  皇帝并不是每天都能收到巡抚一级官员的汇报。许多人故意违反法令,而地方官又不敢动他们,不过,油和大米可以使他们满足,一点点钱就足以使他们交出偷采的矿。正是由于藐视皇帝及其他官员,帝国的采矿业才“非常令人满意。”无怪乎要让税务署“记录在案!” 
  也有人坚信,中国不仅需要加入国际大家庭,而且需要自由交流、自由贸易,需要人们彼此相爱、情同手足。只有商业主义才是中国问题的灵丹妙药。她需要更多的进出口,更低的关税,需要取消通行税。二、三十年前,我们也许不能听到这些观点,那时中国人已充分地渗透到澳大利亚和美国,可他们并没有学会“自由交流”和“彼此相爱、亲如兄弟”。不是早就听说中国的茶和草缏质量不合格吗?它们在某种程度上还不如从西方进口的货物。 
  商业作为文明的辅助手段,其价值是无法估量的。但它本身并不能作为改革的手段。现代经济学的伟大倡导者亚当,斯密把人定义为“商业动物”。他说,任何两条狗都不知道交换骨头。即使假设它们知道,而且在一个大城市里,群狗建立了一个骨头交易市场,这又会对狗的性格带来什么必然的影响呢?古代那些伟大的商业国家,并不是最好的国家,相反,总是最差的。它们的现代继承者,情形完全不同,并不能归因于贸易,完全是由其他原因造成的。有句话说得好,商业如同基督教,目标广大无边;但商业又像雨后彩虹,总弯向金色的一边。 
  只要看一看非洲大陆就行了。猖獗的酒类走私和奴隶贸易,哪一种不是由基督教国家引入的?这些无法形容的灾难,难道不说明,商业并没有给非洲带来革新吗? 
  许多了解中国现状的朋友,为中国开的药方要比上面复杂多了。他们认为,中国需要西方的文化,西方的科学,和梅杜斯先生说的“物质文明”。中国文明已有数千年的历史,我们的祖先还在森林中寻找食物时,她己进入文明社会数百年了。只要是地球上能吃的东西,她都试着烹好过,这种文明如何能改革呢?文化是自私的,它总是有意无意地强调“我,而不是你”。正如在中国,我们引以自豪的文化,却经常遭到嘲弄和非理性的讥笑。如果中国文化对此不适当加以控制,难道外国引入中国的事物不会遭到同样的命运? 
  科学,无疑也是中国最迫切需要的。他们需要各种科学来开发帝国潜在的资源。他们已清楚地看到了这一点,不久的将来,将会看得更清楚。但掌握科学就一定有利于改善帝国的道德状况吗?这要通过何种方式来实现呢?化学是与现代社会发展联系最紧密的学科,然而,化学知识在中国的广泛传播就是中国人获得新生的道德手段吗?难道在生活的各个领域就不会传人新的、意想不到的欺诈与暴力行为吗?按照中国人的现有性格,如果他们掌握了制造现代炸药的配方,而且对化学药品不加控制,难道人们还能过着安全的日子吗? 
  发展“物质文明”就意味着将具备西方高度发展的物质成果。包括以蒸汽机和电力所创造的各种奇迹。人们以为,这才是中国真正需要的,也是她的全部需要。连接各个城市的铁路、内陆河上的汽船航运、完备的邮电系统、国家银行,银市作为通讯中枢的电话与电报一一一这些都是美好的新中国的明显标志。 
  这也许就是张之洞的未成型的想法。他在主张修铁路的奏折中,断言铁路将会消除河运中很多可能的危险,“比如水手偷盗”等等。那么,物质文明的发展就能消除道德上的邪恶吗?铁路能保证雇员,甚至是老板的诚实吗?我们不是读过《伊利城的一章 》吗?那里整段的国际铁路被盗走,股东们束手无策,找不到“该负责的人”。物质文明是自己发展起来的,还是由一系列复杂的条件,经过长期协调,缓慢地发展起来的?引人投票箱,就能使中国成为民主国家,建立共和制度吗?如果中国不想创造西方那样的条件,她就不能获得同样的结果,也不能发展更多的物质文明。这些条件不是物质的,而是道德的。 
  中国人为什么不能学习香港、上海及其他通商口岸的经验,在内地城市设立“租界”呢?因为他们不希望这样的变革,如果设立,他们会难以忍受。在近三分之一的世纪中,他们亲眼看到帝国海关实行正规管理的成效,可为什么不在其他地方实行同样的管理方法呢?因为在中国目前的情况下,中国人对中国人采用这种抽税方法,在道德上是难以接受的。英国人的人格与良心经历了一千多年才发展到目前的水平,中国人不可能立即接受,并实行这一切,不可能像克虏伯大炮一样,架起来就可以发射。 
  盎格鲁一撒克逊民族培养人格和良心的动力就像裘力斯. 凯撒在不列颠登陆或威廉大帝入侵的历史一样确凿无疑,它诞生于基督教,又随着基督教的发展而发展。随着基督教在人们心中扎下根,它们也变得枝叶繁茂了。 
  让我们听一下伟大的文化倡导者马歇尔·阿诺德是如何说的吧:“每一个有教养的人都热爱希腊,感激希腊。希腊是艺术与科学的旗手,如同以色列是正义的旗手一样。现在,世界上离不开艺术与科学。伟大的希腊人是那样热衷于艺术与科学,反倒使品行成了普通的家常事。辉煌的希腊因不注重品行而在地球上消失了,因为人类需要品行、沉静、人格……不仅如此,它也成功地向世人启示,即使在知识受到高度尊重,世界需要越来越多的美和知识的今天,支配世界的不是希腊,而是犹太;不是希腊人卓越的艺术和科学,而是犹太人非凡的正义。” 
  为了改革中国,就必须探明中国人性格的来龙去脉;使之净化,就必须在实际上推崇人的良心,而不能像历代的日本天皇,整日被关在宫中。现代哲学的一位领袖说得好:“铅的本能炼不出金的品行。”中国需要的是正义,为了获得正义,中国人必须了解上帝,必须更新人的概念,并确立人与上帝之间的关系。他们需要全新的灵魂,全新的家庭,全新的社会。总之,中国人的各种需要化为一种迫切的需要,即她应该永久地。彻底地接受基督教文明。 





%这里空一行

\end{common-format}
\end{document}



